\chapter{Making Tables}
\label{chp:chapter2}
\graphicspath{{figures/}{figures/chapter2/}}

\section{Making a Table}
An example \LaTeX\ table is shown below in Table~\ref{tab:comparison}.
You make a table by starting a table environment with a caption
and label.  You can specify the text that shows up in the Table of
Contents using the optional parameter box, [], that's at the
beginning of the $\backslash$caption command. You tell the table how
many columns in the beginning of the tabular environment using a
command like this: $|$ l $|$ c $|$ r $|$. That would create a table with 3 columns
that are left-aligned, centered and right-aligned, in that order. The $|$'s tell \LaTeX\ that
you want bars separating the columns. Of course, you can also make tables without the $|$ characters, in which case no lines will be added between the fields, which often looks better. You can also add horizontal
lines using the $\backslash$hline command. This example is also centered in the page using the
$\backslash$centering command.
%
\begin{table}[b]
\centering
\caption[Example table]{Description of the table, where the caption is long enough\\ to go onto more than one line. You should put line breaks in to\\ make the caption not extend beyond the edges of the table,\\ and to make an ``inverted pyramid.''}
\label{tab:comparison}
%
\begin{tabular}{|l|c|c|r|}
\hline

Table Name  & Column 2 & Column 3   & Column 4 \\
\hline
First Row   & 4780286  & 72.941376  & A \\
Second Row  & 4069335  & 62.093124  & B \\
Third Row   & 4074900  & 62.178040  & C \\
\hline
Fourth Row  & 4000000  & 60.000000  & Z \\
\hline

\end{tabular}
\end{table}
%

Landscape tables can also be inserted, if desired, using the \verb-sidewaystable- environment, which is defined in the \verb!rotating! package. An example is found on the next page, in Table~\ref{tab:landscape}. Note that landscape tables and figures should be the exception rather than the rule, since they are more awkward to read. However, if you have an especially wide table or figure that cannot be reduced in size without losing required resolution, placing it in landscape may be a good idea.
\begin{sidewaystable}
\caption{A landscape table}\vspace*{6pt}
\label{tab:landscape}
\setlength{\tabcolsep}{6mm}
\begin{tabular}{lcccc}
Year & Method of CVD Deposition & Thickness ($\mu$m) & Hardness (MPa) & Resistivity ($\Omega\cdot$m)\\ 
\hline\\[-11pt]
1999 & LPCVD & 5.1 & 175 & 2.3$\times10^{-5}$\\
2000 & PECVD & 17.2 & 101& 1.5$\times10^{-5}$\\
2002 & PECVD & 4.3 & 55 & 3.4$\times10^{-5}$\\
2004 & LPCVD & 1.1 & 225 & 5.1$\times10^{-5}$\\
\hline
\end{tabular} 
\end{sidewaystable}