% TEX root=../root.tex

\subsection{Planar Rotations}
\label{sec:planar_rotations}
% \subsection{2D Rotation Group}
% We use rotation matrices, the Lie group $SO(2)$, to represent 2D rotations.
% We parameterize the 2D rotation matrix by the angle, $\psi$, such that
% \begin{align}
  % \begin{array}{llll}
    % R &= \exp_R \left( \psi \right) &=
  % \begin{bmatrix}
    % \cos \psi & - \sin \psi \\
    % \sin \psi & \cos \psi
  % \end{bmatrix} & \in \mathbb{R}^{2 \times 2} \\
    % \psi &= \log_R \left( \psi \right) &= \mathrm{atan2} \left(r_{21},r_{11}
      % \right) & \in \mathbb{R}.
  % \end{array}
% \end{align}
% We also note that planar rotations are commutative, meaning
% \begin{equation}
  % R_b^c R_a^b = R_a^b R_b^c \qquad \forall R_a^b, R_b^c \in SO(2).
% \end{equation}

% We parameterize the 2D rotation group as an angle, $\psi$, which respresents
We parameterize planar rotations as an angle, $\psi$, which respresents
the angle of rotation about a given axis.
% We treat $\psi$ as a vector space
% that uses the common addition and subtraction operators such that
We treat $\psi \in \mathbb{R}^1$ so that common addition and subtraction
operators can be used such as
\begin{align}
  \psi_a^c &= \psi_a^b + \psi_b^c \\
  \psi_a^b &= \psi_a^c - \psi_b^c.
\end{align}
We note, however, that with this formulation, all addition and subtraction
operations must be wrapped such
that the resultant angle $\psi \in \left[ \right. -\pi, \pi \left. \right)$. The
passive 2D
rotation matrix can be created from any $\psi$ as
\begin{equation}
  R \left( \psi \right) = \begin{bmatrix} \cos \psi & -\sin \psi \\ \sin \psi &
  \cos \psi \end{bmatrix}.
\end{equation}
If $\psi$ represents the angle of rotation about the $z$ axis of a reference frame, then
the corresponding passive 3D rotation matrix is given by
\begin{equation}
  R \left( \psi \right) = \begin{bmatrix}
    \cos \psi & -\sin \psi & 0 \\
    \sin \psi & \cos \psi & 0 \\
    0 & 0 & 1
  \end{bmatrix}.
\end{equation}

% We treat S1 as a vector space but make sure to wrap residuals between -pi and
% pi. 
% Also a 2D rotation matrix can be made like this...
% And a 3D rotation matrix like this..
