% !TEX root=../root.tex

\subsection{Quaternion Representation}
% \MF{TODO what of this is necessary}
% A quaternion $\q$ is a hyper-complex number of rank four.  It is well known that
% a unit quaternion $\in S^3$ can be used to efficiently represent attitude, as
% $S^3$ is a double cover of $\SO(3)$.  Quaternions have the advantage over
% $\SO(3)$ of being more efficient to implement on modern hardware~\cite{casey2013AttitudeRepresentation}, therefore in the software implementation of the described algorithm, we use quaternions, rather than rotation matrices.

% We use the Hamiltonian notation for unit quaternions $\in S^3$
Unit quaternions $\in S^3$ throughout the paper follow the Hamiltonian notation
\begin{equation}
  % \q=\begin{pmatrix}q_{0} & q_{x}i & q_{y} j & q_{z} k \end{pmatrix}.%=\begin{pmatrix}q_{0} & \bar{\q} \end{pmatrix},
  \q= q_{0} + q_{x}i + q_{y} j + q_{z} k .%=\begin{pmatrix}q_{0} & \bar{\q} \end{pmatrix},
\end{equation}
where $i$, $j$, and $k$ are the fundamental quaternion units.
% We write the complex p
% For convenience, we sometimes refer to the complex portion of the quaternion as
% \begin{equation}
	% \bar{\q} = \begin{bmatrix} q_x & q_y & q_z \end{bmatrix} ^\transpose
% \end{equation}
% and write quaternions as the tuple of the real and complex portions
We write quaternions as the tuple
\begin{equation}
	\q = \begin{pmatrix} q_0 \\ \bar{\q} \end{pmatrix}
\end{equation}
where $q_0$ represents the real portion of the quaternion and
% and the complex
% portion of the quaternion is given by
\begin{equation}
	\bar{\q} = \begin{bmatrix} q_x & q_y & q_z \end{bmatrix} ^\transpose
\end{equation}
respresents the complex portion of the quaternion.

% Given our use of the Hamiltonian notation, the quaternion group operator
Given this representation of the quaternion, the quaternion group operator
$\otimes$ can be written as the matrix-like products
\begin{align}
  \q^a \otimes \q^b &= \begin{pmatrix} q_0^a & \left(-\bar{\q}^{a}\right)^\transpose \\ \bar{\q}^a & q^a_0 I + \skewmat{\bar{\q}^a} \end{pmatrix}
	\begin{pmatrix} q_0^b \\ \bar{\q}^b\end{pmatrix}
  \label{eq:quat_otimes_product1} \\
  &= \begin{pmatrix} q_0^b & \left(-\bar{\q}^{b}\right)^\transpose \\ \bar{\q}^b & q^b_0 I - \skewmat{\bar{\q}^b} \end{pmatrix}
	\begin{pmatrix} q_0^a \\ \bar{\q}^a \end{pmatrix}.
  \label{eq:quat_otimes_product2}
\end{align}
% It is often convenient to convert a quaternion $\q$ to its associated passive rotation matrix.  This is done with
We frequently convert a quaternion $\q$ to its associated passive rotation
matrix. This is done with
\begin{equation}
R\left(\q\right)=\left(2q_{0}^{2}-1\right)I-2q_{0}\skewmat{\bar{\q}}+2\bar{\q}\bar{\q}^{\top}\in SO\left(3\right).
\label{eq:est_paper_R_from_q}
\end{equation}
We also note that rotations are written equivalently as
$\q_a^b=R\left(\q_a^b\right)=R_a^b$ throughout the paper.
% where the choice of these is dictated by convenience.
% We use passive rotation matrices, meaning that the rotation matrix $R_a^b$ acts
% on a vector $\vect{r}^a$, expressed in frame $a$, and results in the same
% vector, now expressed in frame $b$ as
% \begin{equation}
% \vect{r}^b = R_a^b \vect{r}^a .
% \end{equation}

% We need to frequently convert between the Lie group, $S^3$, and the Lie
% algebra, $\mathbb{R}^3$, which enables us to operate in a vector space.
To operate in a vector space, we frequently convert between the Lie group,
$S^3$,
% and the Lie algebra, $\mathbb{R}^3$.
and the vector space, $\mathbb{R}^3$, that is isomorphic to the Lie algebra.
This is done with the
exponential and logarithmic mappings. The exponential mapping for a unit quaternion is defined as
\begin{align}
\exp_{\q} & :\mathbb{R}^{3}\rightarrow S^3\nonumber \\
\exp_{\q}\left(\vect{\delta}\right) & \triangleq\begin{bmatrix}\cos\left(\frac{\lVert\vect{\delta}\rVert}{2}\right)\\
\sin\left(\frac{\lVert\vect{\delta}\rVert}{2}\right)\frac{\vect{\delta}}{\lVert\vect{\delta}\rVert}
\end{bmatrix},\label{eqn:exponential_map}
\end{align}
with the corresponding logarithmic map defined as
\begin{align}
\log_{\q} & :S^3\rightarrow \mathbb{R}^{3}\nonumber \\
\log_{\q}\left(\q\right) & \triangleq2\;\mathrm{atan2}\left(\left\Vert \bar{\q}\right\Vert ,q_{0}\right)\frac{\bar{\q}}{\left\Vert \bar{\q}\right\Vert }.\label{eqn:log_map}
\end{align}
% To avoid numerical issues when $\norm{\vect{\delta}}\approx0$, we also employ the small-angle approximations of the quaternion exponential and logarithm
With this formulation, $\vect{\delta}$ also represents the axis-angle
representation of a rotation where the unit vector
$\frac{\vect{\delta}}{\norm{\vect{\delta}}}$ represents the axis of rotation,
and $\norm{\vect{\delta}}$ represents the angular magnitude of rotation.
When $\norm{\vect{\delta}}\approx0$,
we employ the small-angle approximations of the quaternion exponential and
logarithm given by
\begin{align}
\exp_{\q}\left(\vect{\delta}\right) & \approx\begin{bmatrix}1\\
\frac{\vect{\delta}}{2}
\end{bmatrix}\label{eq:est_paper_quaternion_exp_approx}\\
\log_{\q}\left(\q\right) & \approx2\;\textrm{sign}\left(q_{0}\right)\bar{\q}.\label{eq:quaternion_log_approx}
\end{align}


% The $\boxplus$/$\boxminus$ notation for unit quaternions is now defined by
% \begin{align*}
% \boxplus: & S^3\times\mathbb{R}^{3}\rightarrow S^3\\
%  & \q\boxplus\vect{\delta}\triangleq\q\otimes\exp_q\left(\vect{\delta}\right)\\
% \boxminus: & S^3\times S^3\rightarrow\mathbb{R}^{3}\\
%  & \q\boxminus\vect{p}\triangleq\log_q\left(\vect{p}^{-1}\otimes\q\right).
% \end{align*}
% Let's look at one application for clarification.
% With body angular rate $\vect{\omega}_{b/i}^b$, inertial to body rotation $\q_i^b$, and $\vect{\theta}=\vect{\omega}_{b/i}^b \Delta t$, we have
% \begin{align}
% \q_i^b\left(t+\Delta t\right) & =\q_i^b\left(t\right)\boxplus\vect{\theta}\\
% \vect{\theta} & =\q_i^b\left(t+\Delta t\right)\boxminus\q_i^b\left(t\right).
% \end{align}

%\subsection{Quadrotor Dynamics With Quaternion Attitude}
%In the associated software library, we employ the following dynamics for the quadrotor:
%\begin{align}
    %\dot{\vect{p}}_{b/I}^{I} &= 	R\left(\q_{I}^{b}\right)^\transpose \vect{v}_{b/I}^{b} \nonumber \\
	%\dot{\q}_{I}^{b} &= - \frac{1}{2} \begin{pmatrix} 0 \\ \omega_{b/I}^b\end{pmatrix} \otimes \q_I^b\nonumber\\
	%\dot{\vect{v}}_{b/I}^{b} 
	%&= 	g\frac{s}{s_e}\vect{e}_3 + gR\left(\q_I^b\right) \vect{e}_3- \drag\vect{v}_{b/I}^b - \skewmat{\vect{\omega}_{b/I}^b}\vect{v}_{b/I}^b.
	%\label{eq:quat_dynamics}
%\end{align}
%and we solve this using a fourth-order Runge-Kutta extension of \eqref{eq:quaternion_integration}.

% TODO maybe pull from here
%\subsection{Quaternion Error State Dynamics}
%As with the rotation matrix, we wish to find a minimal representation of the error about a quaternion attitude state.  Let us start with the differential equation for quaternion dynamics
%\begin{equation}
	%\dot{\q}_{I}^{b} = \frac{1}{2}\q_I^b \otimes \begin{pmatrix} 0 \\ \vect{\omega}_{b/I}^b\end{pmatrix}.
%\end{equation}
%This can be solved using the quaternion exponential with
%\begin{equation}
	%\q_I^b\left(t\right) = \q_I^b\left(0\right)\otimes\exp_{\q}\left(\int_0^t \vect{\omega}_{b/I}^b\left(\tau\right) d\tau \right).
%\end{equation}
%To reduce this to our minimal vector representation, as we did with rotation matrices in \eqref{eq:att_vector}, let us define the attitude vector $\vect{r}_I^b\left(t\right)=\vect{r}_I^b\left(t_0\right)+\int_{t_0}^t\boldsymbol{\omega}_{b/I}^b\left(\tau\right)d\tau$ with $\vect{r}_I^b\left(t_0\right)=0$ such that
%\begin{equation}
	%\q_I^b\left(t\right) = \q_I^b\left(0\right)\otimes\exp_{\q}\left(\vect{r}_I^b\left(t\right) \right).
%\end{equation}
%It then follows that the error state of our attitude vector is
%\begin{align}
	%\tilde{\vect{r}}_I^{b} 
		%&= \vect{r}_I^b - \hat{\vect{r}}_I^{\hat{b}}  \nonumber \\
		%&= \vect{r}_I^b - R\left(\q_I^b\right) R\left(\q_I^{\hat{b}}\right)^\transpose \hat{\vect{r}}_I^{\hat{b}}\nonumber \\
		%&= \vect{r}_I^b - R\left(\tilde{\q}_{\hat{b}}^b\right)  \hat{\vect{r}}_I^{\hat{b}}
%\end{align}
%and its time derivative is
%\begin{equation}
	%\dot{\tilde{\vect{r}}}_I^{b} = \dot{\vect{r}}_I^b - R\left(\tilde{\q}_{\hat{b}}^b\right)  \dot{\hat{\vect{r}}}_I^{\hat{b}}.
%\end{equation}
%\subsection{Quaternion Trajectory Generation}
%When creating trajectories using differential flatness for quaternion attitude states, the equivalent expression to \eqref{eq:des_attitude} is
%\begin{equation}
	%\des{q}_I^b = \exp_{\q}\left(\vect{e}_3\des{\psi}_{b/I}^I)\right) \otimes \exp_{\q}\left(\theta \vect{\delta}\right) \label{eq:des_attitude}
%\end{equation}
%where, as in \eqref{eq:des_attitude}, $\theta\vect{\delta}$ is the axis-angle error between the desired acceleration and the inertial z-axis.
