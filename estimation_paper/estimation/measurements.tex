% !TEX root=../../master.tex

\subsection{Measurement Models}
\label{sec:measurement_models}

When
updating the filter with measurements, we make use of the partial Kalman update
which has been shown to improve estimates of bias states and constant values~\cite{brink2017partial}. The partial Kalman update provides a
means to limit the effect of measurement updates to certain estimated
states by providing a tuning vector
\begin{equation}
  \vect{\lambda} =
\begin{bmatrix}
  \lambda_1 & \lambda_2 & \dots & \lambda_N
\end{bmatrix}
\end{equation}
where $\lambda_i \in \left[ 0, 1\right]$ determines the proportion of the
measurement update applied to state $i$.
In practice, we use $\lambda < 1$ only for the
bias states, $\vect{\beta}_a$ and $\vect{\beta}_\omega$, and the
constant-value states $\vect{r}_{1/g}^{g}, \dots \vect{r}_{n/g}^{g}$.

With this formulation,
when a measurement is received, we compute the Kalman gain
\begin{equation}
  K = P H^\transpose \left(H P H^\transpose + R \right)^{-1}
\end{equation}
where $H$ is the residual Jacobian for the measurement and $R$ is the
measurement covariance. We then follow~\cite{brink2017partial}, using $K$ to update the filter such that
\begin{align}
  \hat{\tilde{\x}}^{+} &= \vect{\lambda} \odot K \vect{r} \\
  P^{+} &= P + \Lambda \odot \left( \left(I - K H \right) P \left(I - K H
  \right)^\transpose + K R K^\transpose - P \right)
\end{align}
where $\odot$ is the Hadamard product, $\vect{r}$ is the residual of the
measurement and
\begin{align}
  \vect{1} &= \begin{bmatrix} 1 & 1 & \dots & 1 \end{bmatrix}^\transpose \\
  \Lambda &= \vect{1} \vect{\lambda}^\transpose + \vect{\lambda} \vect{1} -
  \vect{\lambda} \vect{\lambda}^\transpose.
\end{align}

As the estimate of the error state of the system, $\hat{\tilde{\x}}$, becomes
non-zero after this update, we use this estimate to correct the estimated state,
$\hat{\x}$. As the estimated state is not a vector, this correction is done
piecewise. The vector components of the estimated state are
updated as
\begin{equation}
  \hat{\vect{\x}}_{\vect{v}}^{+} = \hat{\vect{\x}}_{\vect{v}} + \tilde{\vect{\x}}_{\vect{v}}
\end{equation}
and the quaternion state is updated as
\begin{equation}
  \left( \hat{\q}_I^b \right)^{+}  = \hat{\q}_I^b \otimes \exp_{\q} \left(
  \tilde{\vect{\theta}}_I^b \right).
\end{equation}
After this correction, the estimate of the error state of the system resets to
zero.
% reset to zero.
% The estimate of the error state of the system is then reset to zero before any

The measurement model, residual model and residual
Jacobian are defined below for each type of measurement used in the filter.

\subsubsection{Global UAV Position Measurement}
We assume to receive a measurement of the position of the UAV with respect to
the inertial frame.
In our experiments, this measurement results from a motion capture system;
however, in other applications,
a sensor such as a real-time kinematic GPS unit could provide this measurement.
% a similar measurement can be be used from RTK
% GPS.
% This measurement may come from a sensor such as GPS or a motion capture system.
% The 
% The measurement model and its estimate are
The measurement and its model are
written as
\begin{align}
  \vect{z}_{\text{pos}} &= h_{\text{pos}} \left( \x \right) + \vect{\eta}_{\text{pos}} \\
  h_{\text{pos}} \left( \x \right) &= \vect{p}_{b/I}^I,
  % \vect{\eta}_{\text{pos}},
  % h_{\text{pos}} \left( \hat{\x} \right) &= \hat{\vect{p}}_{b/I}^I,
\end{align}
where $\vect{\eta}_{\text{pos}}$ is a zero-mean Gaussian process
describing the sensor noise.
For a given measurement of position, $\bar{\vect{z}}_{\text{pos}}$, the residual is
given by
\begin{equation}
  \vect{r}_{\text{pos}} = \bar{\vect{z}}_{\text{pos}} - h_{\text{pos}} \left( \hat{\x}
  \right).
\end{equation}
For the error-state Kalman filter, the residual is modeled as
\begin{align}
  % \vect{r}_{\text{pos}} &=  h_{\text{pos}} \left( \x \right) - h_{\text{pos}} \left( \hat{\x}
  % \right) \\
  \vect{r}_{\text{pos}} &=  \vect{z}_{\text{pos}} - h_{\text{pos}} \left( \hat{\x}
  \right) \\
                        &= \vect{p}_{b/I}^I + \eta_{\text{pos}} -
                        \hat{\vect{p}}_{b/I}^I \\
                        &= \tilde{\vect{p}}_{b/I}^I + \eta_{\text{pos}}.
\end{align}
This results in the residual Jacobian
\begin{align}
  H_{\text{pos}} &= \frac{\partial \vect{r}_{\text{pos}}}{\partial \tilde{\x}}\\
                 &=
   \begin{bmatrix}
     \cfrac{\partial \vect{r}_{\text{pos}}}{\partial \tilde{\vect{p}}_{b/I}^{I} } &
     \vect{0} &
     \vect{0} &
     \vect{0} &
     \vect{0} &
     \vect{0} &
     \vect{0} &
     \vect{0} &
     \vect{0} &
     \vect{0} &
     \dots &
     \vect{0}
     % \cfrac{\partial \vect{r}_{\text{pos}}}{\partial \tilde{\vect{\theta}}_I^{b} } &
     % \cfrac{\partial \vect{r}_{\text{pos}}}{\partial \tilde{\vect{v}}_{b/I}^b } &
     % \cfrac{\partial \vect{r}_{\text{pos}}}{\partial \tilde{\vect{\beta}}_a } &
     % \cfrac{\partial \vect{r}_{\text{pos}}}{\partial \tilde{\vect{\beta}}_{\omega} } &
     % \cfrac{\partial \vect{r}_{\text{pos}}}{\partial \tilde{\vect{p}}_{g/b}^{v} } &
     % \cfrac{\partial \vect{r}_{\text{pos}}}{\partial \tilde{\vect{v}}_{g/I}^{g} } &
     % \cfrac{\partial \vect{r}_{\text{pos}}}{\partial \tilde{\psi}_{I}^{g} } &
     % \cfrac{\partial \vect{r}_{\text{pos}}}{\partial \tilde{\omega}_{g/I}^{g} } &
     % \cfrac{\partial \vect{r}_{\text{pos}}}{\partial \tilde{\vect{r}}_{i/g}^{g} }
   \end{bmatrix} \\
                 &=
  \begin{bmatrix}
    I_{3 \times 3} & \vect{0} & \vect{0} & \vect{0} & \vect{0} & \vect{0} &
    \vect{0} & \vect{0} & \vect{0} & \vect{0} & \dots & \vect{0}
  \end{bmatrix}.
\end{align}

% and the non-zero component of the Jacobian of the measurement model as
% \begin{equation}
  % \frac{\partial \hat{\vect{p}}_{b/I}^I}{\partial \vect{p}_{b/I}^I} = I_{3
  % \times 3}.
% \end{equation}

\subsubsection{Global UAV Attitude Measurement}
Similar to the position measurement above, we assume to receive a measurement of
the attitude of the body frame of the UAV with respect to the inertial frame.
In our experiments, this measurement results from a motion capture system;
however, in other applications, a sensor such as an attitude and heading
reference system could provide this measurement.
% This measurement may
% come from an attitude and heading reference system or from a motion
% capture system. 
% The measurement model and its estimate are
The measurement and its model are 
written as
\begin{align}
  \vect{z}_{\text{att}} &= h_{\text{att}} \left( \x \right) \otimes \exp_{\q} \left(
  \vect{\eta}_{\text{att}} \right) \\
  h_{\text{att}} \left( \x \right) &= \vect{q}_{I}^b, 
  % \vect{\eta}_{\text{att}} \right) \\
    % h_{\text{att}} \left( \hat{\x} \right) &= \hat{\vect{q}}_{I}^b,
\end{align}
where $\vect{\eta}_{\text{att}}$ is a zero-mean Gaussian process
describing the sensor noise.
For a given measurement of attitude, $\bar{\vect{z}}_{\text{att}}$, the residual is
given by
\begin{equation}
  \vect{r}_{\text{att}} = \log_{\q} \left(  h_{\text{att}} \left(
  \hat{\x} \right)^{-1} \otimes \bar{\vect{z}}_{\text{att}} \right),
  % \vect{r}_{\text{att}} = \vect{z}_{\text{att}} - h_{\text{att}} \left( \hat{\x}
\end{equation}
which is modeled as
\begin{align}
  \vect{r}_{\text{att}} &= \log_{\q} \left(  h_{\text{att}} \left(
  \hat{\x} \right)^{-1} \otimes \vect{z}_{\text{att}} \right) \\
  % \hat{\x} \right)^{-1} \otimes h_{\text{att}} \left( \x \right) \right) \\
                        &= \log_{\q} \left(  \left(
  \hat{\vect{q}}_{I}^b \right)^{-1} \otimes \vect{q}_{I}^b \otimes \exp_{\q} \left(
  \vect{\eta}_{\text{att}} \right)\right).
\end{align}
This is expanded using~\eqref{eq:quat_true_state} and simplified to yield
\begin{align}
  \vect{r}_{\text{att}}
  % &= \log_{\q} \left(  h_{\text{att}} \left(
  % \hat{\x} \right)^{-1} \otimes h_{\text{att}} \left( \x \right) \right) \\
                        % &= \log_{\q} \left(  \left(
  % \hat{\vect{q}}_{I}^b \right)^{-1} \otimes \vect{q}_{I}^b \otimes \exp_{\q} \left(
  % \vect{\eta}_{\text{att}} \right)\right) \\
                        &= \log_{\q} \left(  \left(
                        \hat{\vect{q}}_{I}^b \right)^{-1} \otimes
                        \hat{\vect{q}}_{I}^b \otimes \exp_{\q} \left(
                      \tilde{\vect{\theta}}_I^b \right) \right)
                          + \vect{\eta}_{\text{att}}  \\
                        &= \tilde{\vect{\theta}}_I^b
                          + \vect{\eta}_{\text{att}}. 
\end{align}
This results in the residual Jacobian
\begin{align}
  H_{\text{att}} &= \frac{\partial \vect{r}_{\text{att}}}{\partial \tilde{\x}}\\
                 &=
   \begin{bmatrix}
     \vect{0} &
     \cfrac{\partial \vect{r}_{\text{att}}}{\partial \tilde{\vect{\theta}}_I^{b} } &
     \vect{0} &
     \vect{0} &
     \vect{0} &
     \vect{0} &
     \vect{0} &
     \vect{0} &
     \vect{0} &
     \vect{0} &
     \dots &
     \vect{0}
     % \cfrac{\partial \vect{r}_{\text{pos}}}{\partial \tilde{\vect{\theta}}_I^{b} } &
     % \cfrac{\partial \vect{r}_{\text{pos}}}{\partial \tilde{\vect{v}}_{b/I}^b } &
     % \cfrac{\partial \vect{r}_{\text{pos}}}{\partial \tilde{\vect{\beta}}_a } &
     % \cfrac{\partial \vect{r}_{\text{pos}}}{\partial \tilde{\vect{\beta}}_{\omega} } &
     % \cfrac{\partial \vect{r}_{\text{pos}}}{\partial \tilde{\vect{p}}_{g/b}^{v} } &
     % \cfrac{\partial \vect{r}_{\text{pos}}}{\partial \tilde{\vect{v}}_{g/I}^{g} } &
     % \cfrac{\partial \vect{r}_{\text{pos}}}{\partial \tilde{\psi}_{I}^{g} } &
     % \cfrac{\partial \vect{r}_{\text{pos}}}{\partial \tilde{\omega}_{g/I}^{g} } &
     % \cfrac{\partial \vect{r}_{\text{pos}}}{\partial \tilde{\vect{r}}_{i/g}^{g} }
   \end{bmatrix} \\
                 &=
  \begin{bmatrix}
    \vect{0} & I_{3 \times 3} &  \vect{0} & \vect{0} & \vect{0} & \vect{0} &
    \vect{0} & \vect{0} & \vect{0} & \vect{0} & \dots & \vect{0}
  \end{bmatrix}.
\end{align}
% \begin{equation}
  % H_{\text{att}} =
  % \begin{bmatrix}
    % \vect{0} & I_{3 \times 3} & \vect{0} & \vect{0} & \vect{0} & \vect{0} & \vect{0} & \vect{0} & \vect{0} & \vect{0}
  % \end{bmatrix}.
% \end{equation}

\subsubsection{Fiducial Translation Measurement}
We assume that a known fiducial marker serves as the desired landing position
for the multirotor UAV on the target vehicle. The goal frame is, therefore,
located at the center of the fiducial marker. In consequence, every detection of the fiducial
marker yields a measurement of the relative translation and rotation from the
camera frame to the goal frame.
% as noted in~\eqref{eq:fiducial_meas}.
% The measurement model and its estimate for this relative translation measurement
The measurement and its model for this relative translation measurement
are written as
\begin{align}
  \vect{z}_{\text{ft}} &=
  h_{\text{ft}} \left( \x \right) + \vect{\eta}_{\text{ft}} \\
  h_{\text{ft}} \left( \x \right) &=
  \vect{p}_{g/c}^c \\
  &= R_b^c \left( R_I^b \vect{p}_{g/b}^v -
  \vect{p}_{c/b}^b \right),
  % h_{\text{ft}} \left( \hat{\x} \right) &=
    % \hat{\vect{p}}_{g/c}^c \\
  % &= R_b^c \left( \hat{R}_I^b \hat{\vect{p}}_{g/b}^v -
    % \vect{p}_{c/b}^b \right),
  % \hat{\vect{q}}_{c}^a  &= R_g^a \hat{R}_I^g \hat{R}_b^I R_c^b
\end{align}
where $R_b^c$ and $\vect{p}_{c/b}^b$ are assumed to be known constants, and
$\vect{\eta}_{\text{ft}}$ is a zero-mean Gaussian process describing the
measurement noise.
For a given measurement of the relative translation to the fiducial marker,
$\bar{\vect{z}}_{\text{ft}}$, the residual is given by
\begin{equation}
  \vect{r}_{\text{ft}} = \bar{\vect{z}}_{\text{ft}} - h_{\text{ft}} \left( \hat{\x}
  \right)
\end{equation}
and modeled as
\begin{align}
  \vect{r}_{\text{ft}} &=  \vect{z}_{\text{ft}} - h_{\text{ft}} \left( \hat{\x}
  \right) \\
                       &= R_b^c \left( R_I^b \vect{p}_{g/b}^v -
                         \vect{p}_{c/b}^b \right)  +\eta_{\text{ft}} - R_b^c \left( \hat{R}_I^b \hat{\vect{p}}_{g/b}^v -
    \vect{p}_{c/b}^b \right)  \\
                       &= R_b^c R_I^b \vect{p}_{g/b}^v 
                          - R_b^c \hat{R}_I^b \hat{\vect{p}}_{g/b}^v +
                          \eta_{\text{ft}}.
                          \label{eq:rft_1}
\end{align}
% We note that from~\eqref{eq:quat_true_state},
% \begin{equation*}
  % R_I^b = R \left( \exp_{\q} \left( \tilde{\vect{\theta}}_I^b \right) \right)
  % \hat{R}_I^b,
% \end{equation*}
% which can be simplifed using~\eqref{eq:quaternion_exp_approx}
% and~\eqref{eq:R_from_q} to show that
% \begin{equation}
  % R_I^b \approx \left( I - \skewmat{\tilde{\vect{\theta}}_I^b }\right) \hat{R}_I^b.
  % \label{eq:expand_R}
% \end{equation}
% We can similarly show that 
% \begin{equation}
  % \left( R_I^b \right)^\transpose \approx \left( \hat{R}_I^b \right)^\transpose
  % \left( I + \skewmat{\tilde{\vect{\theta}}_I^b }\right).
% \end{equation}
Using~\eqref{eq:est_paper_Rapprox}, we expand~\eqref{eq:rft_1} and ignore
second-order terms to yield
\begin{align}
   \vect{r}_{\text{ft}} &\approx R_b^c \left( I - \skewmat{\tilde{\vect{\theta}}_I^b }\right) \hat{R}_I^b \left( \hat{\vect{p}}_{g/b}^v + \tilde{\vect{p}}_{g/b}^v \right) 
                          - R_b^c \hat{R}_I^b \hat{\vect{p}}_{g/b}^v +
                          \eta_{\text{ft}} \\
  &\approx R_b^c \hat{R}_I^b \tilde{\vect{p}}_{g/b}^v -
  R_b^c \skewmat{\tilde{\vect{\theta}}_I^b } \hat{R}_I^b \hat{\vect{p}}_{g/b}^v 
      + \eta_{\text{ft}} \\
&\approx R_b^c \hat{R}_I^b \tilde{\vect{p}}_{g/b}^v + R_b^c \skewmat{\hat{R}_I^b \hat{\vect{p}}_{g/b}^v} \tilde{\vect{\theta}}_I^b 
      + \eta_{\text{ft}}.
\end{align}
This results in the residual Jacobian
\begin{align}
  H_{\text{ft}} &= \frac{\partial \vect{r}_{\text{ft}}}{\partial \tilde{\x}}\\
                 &=
   \begin{bmatrix}
     \vect{0} &
     \cfrac{\partial \vect{r}_{\text{ft}}}{\partial \tilde{\vect{\theta}}_I^{b} } &
     \vect{0} &
     \vect{0} &
     \vect{0} &
     \cfrac{\partial \vect{r}_{\text{ft}}}{\partial \tilde{\vect{p}}_{g/b}^{v} } &
     \vect{0} &
     \vect{0} &
     \vect{0} &
     \vect{0} &
     \dots &
     \vect{0}
     % \cfrac{\partial \vect{r}_{\text{pos}}}{\partial \tilde{\vect{\theta}}_I^{b} } &
     % \cfrac{\partial \vect{r}_{\text{pos}}}{\partial \tilde{\vect{v}}_{b/I}^b } &
     % \cfrac{\partial \vect{r}_{\text{pos}}}{\partial \tilde{\vect{\beta}}_a } &
     % \cfrac{\partial \vect{r}_{\text{pos}}}{\partial \tilde{\vect{\beta}}_{\omega} } &
     % \cfrac{\partial \vect{r}_{\text{pos}}}{\partial \tilde{\vect{p}}_{g/b}^{v} } &
     % \cfrac{\partial \vect{r}_{\text{pos}}}{\partial \tilde{\vect{v}}_{g/I}^{g} } &
     % \cfrac{\partial \vect{r}_{\text{pos}}}{\partial \tilde{\psi}_{I}^{g} } &
     % \cfrac{\partial \vect{r}_{\text{pos}}}{\partial \tilde{\omega}_{g/I}^{g} } &
     % \cfrac{\partial \vect{r}_{\text{pos}}}{\partial \tilde{\vect{r}}_{i/g}^{g} }
   \end{bmatrix} \\
                 &=
  \begin{bmatrix}
    \vect{0} & R_b^c \skewmat{ \hat{R}_I^b \hat{\vect{p}}_{g/b}^v } &  \vect{0}
             & \vect{0} & \vect{0} & R_b^c \hat{R}_I^b &
    \vect{0} & \vect{0} & \vect{0} & \vect{0} & \dots & \vect{0}
  \end{bmatrix}.
\end{align}
% \begin{equation}
  % H_{\text{ft}} =
  % \begin{bmatrix}
    % \vect{0} & R_b^c \skewmat{ \hat{R}_I^b \hat{\vect{p}}_{g/b}^v } & \vect{0} &
    % \vect{0} & \vect{0} & R_b^c \hat{R}_I^b & \vect{0} & \vect{0} & \vect{0} & \vect{0}
  % \end{bmatrix}.
% \end{equation}


% The non-zero components of the Jacobian of the measurement model used for the
% Kalman filter update are

% \begin{align}
  % \frac{\partial \hat{\vect{p}}_{g/c}^c}{\partial \vect{q}_I^b} =& R_b^c 
  % \hat{R}_I^b \skewmat{\hat{\vect{p}}_{g/b}^v} \\
    % \frac{\partial \hat{\vect{p}}_{g/c}^c}{\partial \vect{p}_{g/b}^v} =& R_b^c
    % \hat{R}_I^b.
% \end{align}

\subsubsection{Fiducial Rotation Measurement}
We use the relative rotation measurement that results from a detection of the
fiducial marker
to create a pseudo measurement of the orientation of the goal frame. In theory,
a measurement model could be developed to use the entire measurement,
$\bar{R}_c^g$; however, in practice, this may cause complications if the
fiducial landing marker is not perfectly aligned with the plane of the target
vehicle's motion (i.e., the fiducial marker is slighly rolled or pitched with
respect to the goal frame).
The pseudo
measurement is created with
\begin{equation}
  \bar{\psi}_I^g = \text{yaw} \left( \bar{R}_c^g R_b^c \hat{R}_I^b \right)
\end{equation}
where $\text{yaw}\left(\right)$ is a function that extracts the yaw angle
from a rotation matrix
given by
\begin{equation}
  \text{yaw} \left(
    \begin{bmatrix}
      r_{11} & r_{12} & r_{13} \\
      r_{21} & r_{22} & r_{23} \\
      r_{31} & r_{32} & r_{33} 
    \end{bmatrix}
  \right)
  =
  \mathrm{atan2}\left( r_{12}, r_{11} \right).
\end{equation}
% The measurement model and its estimate are
The measurement and its model are
written as
\begin{align}
  \vect{z}_{\text{fr}} &= h_{\text{fr}} \left( \x \right) + \eta_{\text{fr}} \\
  h_{\text{fr}} \left( \x \right) &= \psi_I^g,
  % h_{\text{fr}} \left( \hat{\x} \right) &= \hat{\psi}_I^g
\end{align}
where $\eta_{\text{fr}}$ is a zero-mean Gaussian process describing the
measurement noise.
For a given (pseudo) measurement of the rotation of the fiducial marker,
$\bar{\psi}_I^g$,
% $z_{\text{fr}}$,
the residual is given by
\begin{equation}
  r_{\text{fr}} = \bar{\psi}_I^g - h_{\text{fr}} \left( \hat{\x} \right),
  % r_{\text{fr}} = z_{\text{fr}} - h_{\text{fr}} \left( \hat{\x} \right),
\end{equation}
which is modeled as
\begin{align}
  % r_{\text{fr}} &= h_{\text{fr}} \left( \x \right) - h_{\text{fr}} \left( \hat{\x} \right) \\
  r_{\text{fr}} &= \vect{z}_{\text{fr}} - h_{\text{fr}} \left( \hat{\x} \right) \\
                &= \psi_I^g + \eta_{\text{fr}} - \hat{\psi}_I^g \\
                &= \tilde{\psi}_I^g + \eta_{\text{fr}}.
\end{align}
This results in the residual Jacobian
\begin{align}
  H_{\text{fr}} &= \frac{\partial \vect{r}_{\text{fr}}}{\partial \tilde{\x}}\\
                 &=
   \begin{bmatrix}
     \vect{0} &
     \vect{0} &
     \vect{0} &
     \vect{0} &
     \vect{0} &
     \vect{0} &
     \vect{0} &
     \cfrac{\partial \vect{r}_{\text{fr}}}{\partial \tilde{\psi}_I^g } &
     0 &
     \vect{0} &
     \dots &
     \vect{0}
     % \cfrac{\partial \vect{r}_{\text{pos}}}{\partial \tilde{\vect{\theta}}_I^{b} } &
     % \cfrac{\partial \vect{r}_{\text{pos}}}{\partial \tilde{\vect{v}}_{b/I}^b } &
     % \cfrac{\partial \vect{r}_{\text{pos}}}{\partial \tilde{\vect{\beta}}_a } &
     % \cfrac{\partial \vect{r}_{\text{pos}}}{\partial \tilde{\vect{\beta}}_{\omega} } &
     % \cfrac{\partial \vect{r}_{\text{pos}}}{\partial \tilde{\vect{p}}_{g/b}^{v} } &
     % \cfrac{\partial \vect{r}_{\text{pos}}}{\partial \tilde{\vect{v}}_{g/I}^{g} } &
     % \cfrac{\partial \vect{r}_{\text{pos}}}{\partial \tilde{\psi}_{I}^{g} } &
     % \cfrac{\partial \vect{r}_{\text{pos}}}{\partial \tilde{\omega}_{g/I}^{g} } &
     % \cfrac{\partial \vect{r}_{\text{pos}}}{\partial \tilde{\vect{r}}_{i/g}^{g} }
   \end{bmatrix} \\
                 &=
  \begin{bmatrix}
    \vect{0} & \vect{0} &  \vect{0}
             & \vect{0} & \vect{0} & \vect{0} &
    \vect{0} &  1 & 0 & \vect{0} & \dots & \vect{0}
  \end{bmatrix}.
\end{align}
% resulting in the residual Jacobian
% \begin{equation}
  % H_{\text{fr}} =
  % \begin{bmatrix}
    % \vect{0} & \vect{0} & \vect{0} &
    % \vect{0} & \vect{0} & \vect{0} & \vect{0} & 1 & \vect{0} & \vect{0}
  % \end{bmatrix}.
% \end{equation}

% and the non-zero Jacobians of the measurement model as
% \begin{equation}
  % \frac{\partial \hat{\theta}_I^g}{\partial \theta_I^g} = 1.
% \end{equation}

\subsubsection{Visual Feature Pixel Measurement}
The estimator receives measurements of
the location of each tracked visual feature in the camera image. We assume that the
pixel locations received have already been corrected for lens distortion.
% and
We also assume to know the camera intrinsic matrix,
% that we know the camera intrinsic matrix,
% We also assume to know the camera intrinsic matrix
\begin{equation}
  K =
  \begin{bmatrix}
    f_x & 0 & c_x \\
    0 & f_y & c_y \\
    0 & 0 & 1
  \end{bmatrix},
\end{equation}
where $f_x$ and $f_y$ are the focal lengths of the camera and $c_x$ and $c_y$
are the coordinates of the principal point in the camera image.
Using the pinhole camera model, we express the pixel location of
visual feature $i$ as
\begin{align}
  % \hat{h} &=
  % \begin{bmatrix}
    % \hat{p}_x & \hat{p}_y
  % \end{bmatrix}^\transpose \\
  \begin{bmatrix}
    p_x \\ p_y \\ 1
  \end{bmatrix} &= \frac{1}{\e_3^\transpose \vect{p}_{i/c}^c} K
  \vect{p}_{i/c}^c
  \label{eq:pinhole_camera}
  % \vect{z} &=
  % \begin{bmatrix}
    % f_x \frac{\e_1 \vect{p}_{i/c}^c}{\e_3 \vect{p}_{i/c}^c} + c_x \\
    % f_y \frac{\e_2 \vect{p}_{i/c}^c}{\e_3 \vect{p}_{i/c}^c} + c_y
  % \end{bmatrix},
\end{align}
% where K is the camera intrisic matrix given by
% \begin{equation}
  % K =
  % \begin{bmatrix}
    % f_x & 0 & c_x \\
    % 0 & f_y & c_y \\
    % 0 & 0 & 1
  % \end{bmatrix}
% \end{equation}
% with $f_x$ and $f_y$ 
% and 
where
\begin{align}
  \vect{p}_{i/c}^c = R_b^c \left( R_I^b \left( R_I^g \right)^\transpose
  \vect{r}_{i/g}^g + R_I^b \vect{p}_{g/b}^v - \vect{p}_{c/b}^b \right).
  \label{eq:p_i_c_c}
\end{align}
% The measurement model and its estimate are, therefore, written as
The measurement and its model are, therefore, written as
\begin{align}
  \vect{z}_{\text{pix}} 
  &= h_{\text{pix}} \left( \x \right) + \vect{\eta}_{\text{pix}} \\
  h_{\text{pix}} \left( \x \right)
  &= \begin{bmatrix} p_x & p_y \end{bmatrix}^\transpose \\
  &= \frac{1}{\e_3^\transpose \vect{p}_{i/c}^c} I_{2 \times 3} K
  \vect{p}_{i/c}^c,
  % h_{\text{pix}} \left( \hat{\x} \right)
  % &= \begin{bmatrix} \hat{p}_x & \hat{p}_y \end{bmatrix}^\transpose \\
  % &= \frac{1}{\e_3^\transpose \hat{\vect{p}}_{i/c}^c} I_{2 \times 3} K
  % \hat{\vect{p}}_{i/c}^c
\end{align}
where $\vect{\eta}_{\text{pix}}$ is a zero-mean Gaussian process describing the
measurement noise.
% and
% \begin{align}
  % \hat{\vect{p}}_{i/c}^c = R_b^c \left( \hat{R}_I^b \left( \hat{R}_I^g \right)^\transpose
  % \hat{\vect{r}}_{i/g}^g + \hat{R}_I^b \hat{\vect{p}}_{g/b}^v - \vect{p}_{c/b}^b
  % \label{eq:p_i_c_c_hat}
% \right).
% \end{align}
For a given measurement of the pixel location of a feature,
$\bar{\vect{z}}_{\text{pix}}$, the residual is given by
\begin{equation}
  \vect{r}_{\text{pix}} = \bar{\vect{z}}_{\text{pix}} - h_{\text{pix}} \left( \hat{\x}
    \right),
\end{equation}
which is modeled as
\begin{align}
  \vect{r}_{\text{pix}} &= \vect{z}_{\text{pix}} - h_{\text{pix}} \left( \hat{\x}
    \right) \\
  &= \frac{1}{\e_3^\transpose \vect{p}_{i/c}^c} I_{2 \times 3} K
  \vect{p}_{i/c}^c + \vect{\eta}_{\text{pix}} - \frac{1}{\e_3^\transpose \hat{\vect{p}}_{i/c}^c} I_{2 \times 3} K
  \hat{\vect{p}}_{i/c}^c.
  \label{eq:rpix}
\end{align}
This results in the residual Jacobian
\begin{align}
  H_{\text{pix}} &= \frac{ \partial \vect{r}_{\text{pix}} }{ \partial \tilde{\x}} \\
  % H_{\text{pix}} 
                   % &= \frac{ \partial} \\
  &\approx
  \frac{1}{\e_3^\transpose \hat{\vect{p}}_{i/c}^c} I_{2 \times 3} K
 \frac{\partial}{\partial \tilde{\x}} \vect{p}_{i/c}^c 
 - \frac{\e_3^\transpose \frac{\partial}{\partial \tilde{\x}} 
 \vect{p}_{i/c}^c}{\left( \e_3^\transpose
 \hat{\vect{p}}_{i/c}^c \right)^2 } I_{2 \times 3} K
 \hat{\vect{p}}_{i/c}^c
\end{align}
where the non-zero components of $\frac{\partial}{\partial \tilde{\x}}
\vect{p}_{i/c}^c$ are given by
\begin{align}
  \frac{\partial}{\partial \tilde{\vect{\theta}}_I^b } \vect{p}_{i/c}^c
  &=
  R_b^c \skewmat{ \hat{R}_I^b \left( \left( \hat{R}_I^g \right)^\transpose
  \hat{\vect{r}}_{i/g}^g + \hat{\vect{p}}_{g/b}^v \right) } \\
  \frac{\partial}{\partial \tilde{\vect{p}}_{g/b}^v} \vect{p}_{i/c}^c
  &=
  R_b^c \hat{R}_I^b \\
  \frac{\partial}{\partial \tilde{\psi}_I^g} \vect{p}_{i/c}^c
  &=
  R_b^c \hat{R}_I^b \left( \hat{R}_I^g \right)^\transpose
  \skewmat{\begin{matrix} 0 \\ 0 \\ 1 \end{matrix}} \hat{\vect{r}}_{i/g}^g \\
  \frac{\partial}{\partial \tilde{\vect{r}}_{i/g}^g} \vect{p}_{i/c}^c
  &=
  R_b^c \hat{R}_I^b \left( \hat{R}_I^g \right)^\transpose .
\end{align}
The derivation of this residual Jacobian is found in
Appendix~\ref{apdx:estimation_pixel_meas_model}.



% where $f_x$, $f_y$, $c_x$, and $c_y$ are constant parameters of the camera. We
% can then express the position of the landmark $i$ with respect to the camera in the
% camera frame as
% \begin{align}
  % \vect{p}_{i/c}^c &= R_b^c R_v^b \left(\vect{p}_{i/v}^v -
    % \vect{p}_{c/v}^v\right) \\
    % \vect{p}_{i/c}^c &= R_b^c \left(R_v^b \vect{p}_{i/v}^v -
    % \vect{p}_{c/b}^b\right)
% \end{align}
% This is actually a little bit weird because of the state parameters, but
% \begin{equation}
  % \vect{p}_{i/v}^v =
  % \begin{bmatrix}
    % \e_1^\transpose \left( R_g^v \vect{p}_{i/g}^g + \vect{p}_{g/v}^v \right) \\
    % \e_2^\transpose \left( R_g^v \vect{p}_{i/g}^g + \vect{p}_{g/v}^v \right) \\
    % \e_3^\transpose \left( \vect{p}_{i/g}^g - \vect{p}_{b/I}^I \right)
  % \end{bmatrix}.
% \end{equation}

% Let us derive the measurement Jacobians.
% \begin{align}
  % \frac{\partial}{\partial \x} \hat{h} &= \frac{\e_3^\transpose \hat{\vect{p}}_{i/c}^c
  % \frac{\partial}{\partial \x} K \hat{\vect{p}}_{i/c}^c - K \hat{\vect{p}}_{i/c}^c
  % \frac{\partial}{\partial \x} \e_3^\transpose \hat{\vect{p}}_{i/c}^c}{\left(
% \e_3^\transpose \hat{\vect{p}}_{i/c}^c \right)^2 } \\
  % \frac{\partial}{\partial \x} \hat{h} &= \frac{\e_3^\transpose \hat{\vect{p}}_{i/c}^c
  % K \frac{\partial}{\partial \x} \hat{\vect{p}}_{i/c}^c - K \hat{\vect{p}}_{i/c}^c
  % \e_3^\transpose \frac{\partial}{\partial \x} \hat{\vect{p}}_{i/c}^c}{\left(
% \e_3^\transpose \hat{\vect{p}}_{i/c}^c \right)^2 }
% \end{align}
% \begin{align}
  % \frac{\partial}{\partial \vect{p}_{g/b}^v} \hat{\vect{p}}_{i/c}^c &=
  % R_b^c \hat{R}_I^b \\
  % \frac{\partial}{\partial \vect{q}_{I}^b} \hat{\vect{p}}_{i/c}^c &=
  % -R_b^c \hat{R}_I^b \skewmat{ \left( R_I^g \right)^\transpose
  % \hat{\vect{r}}_{i/g}^g + \hat{\vect{p}}_{g/b}^v } \\
  % \frac{\partial}{\partial \theta_I^g} \hat{\vect{p}}_{i/c}^c &=
  % R_b^c \hat{R}_I^b \\
% \end{align}

% % \subsubsection{Measurement Jacobian}
% For the purpose of deriving the Jacobians, we will just look at the jacobians
% with respect to $p_x$, the pixel location along the $x$ axis of the camera
% frame. From above we have that
% \begin{align}
  % p_x &= f_x \frac{\e_1^\transpose \vect{p}_{i/c}^c}{\e_3^\transpose \vect{p}_{i/c}^c} + c_x \\ 
  % p_x &= f_x \frac{\e_1^\transpose R_b^c \left(R_v^b \vect{p}_{i/v}^v -
      % \vect{p}_{c/b}^b\right) }{\e_3^\transpose R_b^c \left(R_v^b \vect{p}_{i/v}^v -
  % \vect{p}_{c/b}^b\right) } + c_x 
% \end{align}
% Note that all values are constants except for $\vect{p}_{i/v}^v$ and $R_v^b$.
% The individual parts of the jacobian are given here:
% \begin{equation}
  % \frac{\partial p_x}{\partial \vect{p}_{g/v}^v} =
  % \frac{f_x \e_1 R_b^c R_v^b}
    % {\left(\e_3 R_b^c \left(R_v^b \vect{p}_{i/v}^v -
    % \vect{p}_{c/b}^b\right)\right)}
    % - \frac{\left(\e_3 R_b^c R_v^b \right) f_x \left(\e_1 R_b^c \left(R_v^b \vect{p}_{i/v}^v -
        % \vect{p}_{c/b}^b\right)\right)} {\left(\e_3 R_b^c \left(R_v^b \vect{p}_{i/v}^v -
  % \vect{p}_{c/b}^b\right)\right)^2}
% \end{equation}
% where only the first two (of three) entries of the vector are used. The third
% entry is used for the jacobian w.r.t. $\vect{p}_{b/I}^I(2)$ as
% \begin{equation}
  % \frac{\partial p_x}{\partial \vect{p}_{b/I}^I(2)} = -\frac{\partial p_x}{\partial
  % \vect{p}_{g/v}^v}\left(2\right).
% \end{equation}
% The derivatives for the attitude representation are less straight forward. From
% "Micro Lie Theory" we use the fact that the jacobian of the rotation action can
% be shown to be
% \begin{equation}
  % J_R^{R \cdot v} = -R \skewmat{v}.
% \end{equation}
% \begin{equation}
  % \frac{\partial p_x}{\partial \q} =
  % \frac{-f_x \e_1 R_b^c R_v^b \skewmat{\vect{p}_{i/v}^v}}
    % {\left(\e_3 R_b^c \left(R_v^b \vect{p}_{i/v}^v -
    % \vect{p}_{c/b}^b\right)\right)}
    % - \frac{\left(-\e_3 R_b^c R_v^b \skewmat{\vect{p}_{i/v}^v} \right) f_x \left(\e_1 R_b^c \left(R_v^b \vect{p}_{i/v}^v -
        % \vect{p}_{c/b}^b\right)\right)} {\left(\e_3 R_b^c \left(R_v^b \vect{p}_{i/v}^v -
  % \vect{p}_{c/b}^b\right)\right)^2}
% \end{equation}
% where the jacobians for the other angles are similar, just substituting in.
% The derivative for the goal angle, $\theta_g$ is given by using
% \begin{equation}
  % \frac{\partial \vect{p}_{i/v}^v}{\partial \theta_g} =
  % \begin{bmatrix}
    % \e_1^\transpose \frac{\partial R_g^v}{\partial \theta_g} \vect{p}_{i/g}^g \\
    % \e_2^\transpose \frac{\partial R_g^v}{\partial \theta_g} \vect{p}_{i/g}^g \\
    % 0
  % \end{bmatrix}
% \end{equation}
% in the jacobian given by
% \begin{equation}
  % \frac{\partial p_x}{\partial \theta_g} =
  % \frac{f_x \e_1 R_b^c R_v^b \frac{\partial \vect{p}_{i/v}^v}{\partial \theta_g}}
    % {\left(\e_3 R_b^c \left(R_v^b \vect{p}_{i/v}^v -
    % \vect{p}_{c/b}^b\right)\right)}
    % - \frac{\left(\e_3 R_b^c R_v^b \frac{\partial \vect{p}_{i/v}^v}{\partial \theta_g} \right) f_x \left(\e_1 R_b^c \left(R_v^b \vect{p}_{i/v}^v -
        % \vect{p}_{c/b}^b\right)\right)} {\left(\e_3 R_b^c \left(R_v^b \vect{p}_{i/v}^v -
  % \vect{p}_{c/b}^b\right)\right)^2}
% \end{equation}
% Similarly, the derivative for the landmark offset, $\vect{r}_i$ or
% $\vect{p}_{i/g}^g$ is given by using
% \begin{equation}
  % \frac{\partial \vect{p}_{i/v}^v}{\partial \vect{p}_{i/g}^g} =
  % \begin{bmatrix}
    % \e_1^\transpose R_g^v \\
    % \e_2^\transpose R_g^v \\
    % \e_3^\transpose
  % \end{bmatrix}
% \end{equation}
% in the jacobian given by
% \begin{equation}
  % \frac{\partial p_x}{\partial \vect{p}_{i/g}^g} =
  % \frac{f_x \e_1 R_b^c R_v^b \frac{\partial \vect{p}_{i/v}^v}{\partial \vect{p}_{i/g}^g}}
    % {\left(\e_3 R_b^c \left(R_v^b \vect{p}_{i/v}^v -
    % \vect{p}_{c/b}^b\right)\right)}
    % - \frac{\left(\e_3 R_b^c R_v^b \frac{\partial \vect{p}_{i/v}^v}{\partial \vect{p}_{i/g}^g} \right) f_x \left(\e_1 R_b^c \left(R_v^b \vect{p}_{i/v}^v -
        % \vect{p}_{c/b}^b\right)\right)} {\left(\e_3 R_b^c \left(R_v^b \vect{p}_{i/v}^v -
  % \vect{p}_{c/b}^b\right)\right)^2}
% \end{equation}
% The jacobians for the y pixel measurement, $p_y$ are similar to the ones derived
% above, just changing $f_y$ in place of $f_x$ and $\e_2$ instead of $\e_1$.

