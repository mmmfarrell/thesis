% This is just to make sure that if it goes onto multiple pages that
% it will only be on odd pages.
\afterpage{\cleardoublepage}
% Some people have had problems with needing a little more space or a little less space right before the body of the abstract. If you have that problem, you can uncomment the next line, and either add or take away space manually (negative spaces are OK).
%\vspace*{-0.05in}
% The abstract is a summary of the dissertation, thesis, or selected project with emphasis on the findings of the study.  The abstract must not exceed 1 page in length.  It should be printed in the same font and size as the rest of the work.  The abstract precedes the acknowledgments page and the body of the work.
Though multirotor unmanned aerial vehicles (UAVs) have become widely used for a variety of
tasks during the past decade, challenges in autonomy have prevented their
widespread operation from moving vehicles. Emerging use cases including maritime
surveillance, package delivery and convoy support require UAVs to autonomously
take off and land from moving vehicles in potentially challenging weather
conditions. This thesis presents improved solutions to both the estimation and
control problems that must be solved for a multirotor UAV to robustly land
autonomously on a moving vehicle in challenging conditions.

Current state-of-the-art UAV landing systems depend on visual fiducial markers
which serve as the landing pad for the UAV on the moving vehicle. However, in challenging weather
conditions, these fiducial markers may be undetected by the UAV for significant
periods of time. This thesis details a state estimation algorithm that tracks
and estimates the locations of unknown visual features on the landing vehicle in
addition to the fiducial marker. Simulation and hardware experiments described
in this thesis demonstrate that the addition of these visual features results in
accurate and consistent state estimates of the landing vehicle even when the
fiducial marker is undetected for long durations.

This thesis also describes an improved control scheme that enables a multirotor
UAV to accurately track a time-dependent trajectory. Rooted in Lie theory, this
controller computes the optimal control signal based on an error-state
formulation of the UAV dynamics, similar to that used in state estimation.
Simulation and hardware experiements of this control scheme demonstrate its
accuracy and computational efficiency.
