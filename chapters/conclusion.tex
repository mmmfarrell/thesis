% !TEX root=../master.tex

\chapter{Conclusion}
\label{chp:conclusion}

The robust autonomous operation of multirotor UAVs from moving-vehicle base
stations could benefit a
variety of new industries.
In an effort to make this a possibility,
this
thesis presents improvements to state estimation and control methods used by UAVs
during landing on moving vehicles.
The estimation algorithm proposed in
Chapter~\ref{chp:estimation_paper}
allows a UAV to continue to operate reliably with respect to a target vehicle
even when a fiducial landing marker is not detected for significant
periods of time.
The control algorithm proposed in Chapter~\ref{chp:control_paper} provides a principled
and computationally efficient method for a UAV to track a feasible, time-dependent
trajectory.
Simulation and hardware tests demonstrated the accurate and robust performance
of these two algorithms.
These results suggest that a reliable system for UAVs landing
on moving vehicles could be developed by combining these proposed state
estimation and control algorithms with an adequate motion planning algorithm.

\section{Future Work}
\label{sec:future_work}
While the research in this thesis presents improvements to 
state-of-the-art methods, there remain many opportunities for future
improvements.
The following subsections outline
recommendations and directions for future work related to the autonomous
landing of multirotor UAVs on moving vehicles.
These recommendations are divided into the specific fields of state estimation,
motion planning, and control.

\subsection{State Estimation}
The estimation algorithm presented in Chapter~\ref{chp:estimation_paper} details
a method of tracking and estimating the locations of visual features 
to aid in estimation when a fiducial landing marker is
not detected. The results of this work were quite remarkable, showing that a UAV
can continue to operate accurately with respect to a target vehicle without
detecting a fiducial marker for more
than a minute. However, as mentioned in~\secref{sec:estimation}, this estimation algorithm
assumes that all tracked features are rigidly attached to the target vehicle.
Therefore, before the presented estimation algorithm can be used in a commercial
product, a feature tracker must be developed that can reliably filter out visual
features that are not part of the target vehicle.
Recently, there has been significant research done
in the fields of image and video segmentation that could be leveraged for such a
task~\cite{chen2018encoder}. Alternative methods that depend on
mathematical models of rigid body motion could also be explored as a possible
solution to this problem.

While the proposed estimation method is easily extensible to a variety of
target vehicles, experiments to test this algorithm were only conducted
with a simple ground vehicle.
In practice, many use cases require UAVs to land on
vehicles with much more complex motion models such as trucks driving on
mountainous terrain or boats on rough seas.
Further experiments should be conducted to test the performance of the presented
estimation algorithm with target vehicles of complex motion models.
To achieve better results in these scenarios, additional measurement updates
from sensors mounted to the target vehicle, such as GPS or IMU,
could be added  to the algorithm.

\subsection{Motion Planning}
\label{sec:future_motion_planning}
While not discussed in great detail in this thesis, the motion planning problem
associated with multirotor UAVs operating from moving vehicles is 
challenging.
Dynamically feasible trajectories must be planned for UAVs to avoid
obstacles during their approach to the landing target.
This problem is especially important when a UAV must land on a large boat
with significant superstructure or when a UAV must re-enter and land inside of a
package delivery truck. While planning trajectories for UAVs with respect to
static obstacles has been widely researched, planning trajectories to avoid
dynamic obstacles that are rigidly attached to the desired landing target is its
own unique problem to be solved.

When operating from vehicles with complex motion, 
it can also be important to precisely time the touchdown of the UAV.
For instance, when landing on a boat in rough
waters, it may be desireable that the UAV touches down when the heave motion of the
boat reaches a peak to minimize potential damage to the UAV.
Further research should be conducted to plan feasible, time-dependent
trajectories for these scenarios.


\subsection{Control}
As discussed in Chapter~\ref{chp:control_paper}, the Lie-theory-based controller
operates on the error state of the system.
The main advantage to this approach is its mathematically principled nature.
This, in
theory, allows for more accurate tracking during extreme maneuvers, where
less principled methods, such as typical Euler-angle formulations, struggle.
This controller should be tested with feasible trajectories of acrobatic
maneuvers to demonstrate this advantage.

One key assumption made by the controller is that the
error-state of the system is always small. This was
especially not true in the flight experiments during takeoff, when the UAV was far from the
desired sinusoidal trajectory.
This resulted in unsteady transient behavior when the error-state values were
not saturated to be small.
Instead of a static trajectory,
the controller should be tested with
dynamically planned trajectories which begin at the current state of the
UAV. With these trajectories, the error state of the system would be forced to
be small, satisfying this important assumption.
