% !TEX root=../master.tex

\chapter{Conclusion}
\label{chp:conclusion}

The ability of a multirotor UAV to autonomously operate from a moving vehicle
provides the potential for UAVs to benefit a variety of new industries. This
thesis presents improvements to the estimation and control algorithms of a UAV
during landing on a moving vehicle. The improved estimation algorithm presented
allows the UAV to continue to operate with respect to the landing vehicle
reliable even when a fiducial landing marker is not detected for significant
periods of time. The improved control algorithm presented provides a principled
and computationally efficient method for the UAV to track a time-dependent
trajectory to reach touchdown. The results of both simulation and hardware
experiments presented in this thesis show that the combination of these estimation and control
algorithms along with a motion planning algorithm would result in a robust
system for UAVs landing on moving vehicles.

\section{Future Work}
\label{sec:future_work}
While the research presented in this thesis presents improvements to current
state-of-the-art methods, there remain many opportunities for improvements to
continue to be made in the future. The following subsections outline
recommendations and directions for future improvements related to the autonomous
landing of multirotor UAVs on moving vehicles specific to the fields of state
estimation, motion planning, and control.

\subsection{State Estimation}
The estimation algorithm presented in Chapter~\ref{chp:estimation_paper} details
a method of tracking and estimating visual features that are rigidly attached to
a moving vehicle to aid in estimation when a visual fiducial landing marker is
not detected. The results of this work were quite remarkable, showing that a UAV
can continue to operate accurately with respect to a landing vehicle for more
than a minute. As mentioned in~\secref{sec:estimator}, the estimation algorithm
assumes that all tracked features are rigidly attached to the landing vehicle.
It is not a trivial task to determine which visual features in a given camera
image belong to a particular object. Recently, however, there has been much research done
in the fields of image and video segmentation that could be leveraged for such a
distinction~\cite{chen2018encoder}. 

While the proposed estimation method is easily extensible to a variety of
landing vehicles, experiments were only conducted testing the estimation
algorithm with a UAV attempting to land on a ground vehicle which moves on flat
ground. In practice, many use cases require UAVs to land on
vehicles with much more complex motion models such as trucks driving on
mountainous terrain or boats on rough seas.
It would be interesting to see if the presented estimation algorithm achieves
similar levels of success with more complex motion models, or if more
information about the motion of the landing vehicle is necessary in these cases.
If more information were necessary, this could easily be done by implementing
additional measurement updates of
measurements originating from sensors mounted to the landing vehicle including
GPS and IMU.

\subsection{Motion Planning}
While not discussed in great detail in this thesis, the motion planning problem
associated with multirotor UAVs operating from moving vehicles is very
challenging. When operating from vehicles with complex motion, such as a boat,
it can become important to precisely time the touchdown of the UAV to minimize
potential damage to the UAV. For instance, when landing on a boat in rough
waters, it may be desireable that the UAV touchdown when the heave motion of the
boat reaches a peak instead of a trough to avoid rough contact between the UAV
and the boat. While this type of motion planning also requires very detailed and
accurate estimation and prediction of the motion of the boat, the motion
planning problem itself presents many unsolved challenges.

Motion planning for a UAV operating from a moving vehicle also presents an
interesting problem when the UAV must navigate close to or around obstacles on
the moving vehicle. This scenario may arise when a UAV must land on a large boat
with significant superstructure or when a UAV must re-enter and land inside of a
package delivery truck. While planning trajectories for UAV's with respect to
static obstacles has been widely researched, planning trajectories to avoid
moving obstacles while attempting to land on a moving platform is its own unique
problem to be solved.

\subsection{Control}
As discussed in Chapter~\ref{chp:control_paper}, the controller presented is
based in Lie theory to accurately track a time-dependent trajectory. While
operating on the error-state of the system provides its own computational
advantages, one key advantage to this principled approach is that it should
perform better during more extreme maneuvers than a typical ZYX Euler angle
formulation. Simulation and hardware experiments should be conducted to show the
advantage this error-state controller presents over existing control schemes.

