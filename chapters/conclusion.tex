% !TEX root=../master.tex

\chapter{Conclusion}
\label{chp:conclusion}

The robust autonomous operation of multirotor UAVs from moving vehicles would benefit a
variety of new industries.
% The ability of multirotor UAVs to autonomously operate from a moving vehicle
% provides the potential for UAVs to benefit a variety of new industries.
This
thesis presents improvements to the estimation and control algorithms of a UAV
for use during landing on a moving vehicle. The estimation algorithm in
Chapter~\ref{chp:estimation_paper}
allows the UAV to continue to operate reliably with respect to the target vehicle
even when a fiducial landing marker is not detected for significant
periods of time.
The control algorithm in Chapter~\ref{chp:control_paper} provides a principled
and computationally efficient method for the UAV to track a time-dependent
trajectory to reach touchdown on the target vehicle. The results of simulation and hardware
experiments show that the combination of these estimation and control
algorithms, together with a motion planning algorithm, would result in a robust
system for UAVs landing on moving vehicles.

\section{Future Work}
\label{sec:future_work}
While the research in this thesis presents improvements to 
state-of-the-art methods, there remain many opportunities for future
improvements.
The following subsections outline
recommendations and directions for future work related to the autonomous
landing of multirotor UAVs on moving vehicles specific to the fields of state
estimation, motion planning, and control.

\subsection{State Estimation}
The estimation algorithm presented in Chapter~\ref{chp:estimation_paper} details
a method of tracking and estimating the locations of visual features 
to aid in estimation when a fiducial landing marker is
not detected. The results of this work were quite remarkable, showing that a UAV
can continue to operate accurately with respect to a target vehicle for more
than a minute. However, as mentioned in~\secref{sec:estimation}, this estimation algorithm
assumes that all tracked features are rigidly attached to the target vehicle.
Therefore, before the presented estimation algorithm can be used in a commercial
product, a feature tracker must be developed that can reliably filter out visual
features that are not part of the target vehicle.
Recently, there has been significant research done
in the fields of image and video segmentation that could be leveraged for such a
task~\cite{chen2018encoder}. Alternative methods that depend on
mathematical models of rigid body motion could also be explored as a possible
solution to this problem.

While the proposed estimation method is easily extensible to a variety of
target vehicles, experiments to test this algorithm were only conducted
with a simple ground vehicle.
% testing the estimation
% algorithm with a UAV attempting to land on a ground vehicle which moves on flat
% ground.
In practice, many use cases require UAVs to land on
vehicles with much more complex motion models such as trucks driving on
mountainous terrain or boats on rough seas.
Further experiments should be conducted to test the performance of the presented
estimation algorithm with target vehicles of complex motion models.
% It would be interesting to see if the presented estimation algorithm achieves
% similar levels of success with more complex motion models, or if more
% information about the motion of the landing vehicle is necessary in these cases.
To achieve better results in these scenarios, additional measurement updates
from sensors mounted to the target vehicle, such as GPS or IMU,
could be added  to the algorithm.
% These sensors could include a GPS unit or an
% IMU.
% If more information were necessary, this could easily be done by implementing
% additional measurement updates of
% measurements originating from sensors mounted to the landing vehicle such as a
% GPS unit or an IMU.

\subsection{Motion Planning}
\label{sec:future_motion_planning}
While not discussed in great detail in this thesis, the motion planning problem
associated with multirotor UAVs operating from moving vehicles is very
challenging.
% Motion planning for a UAV also presents a
% problem when the UAV must navigate close to or around obstacles on
% the moving vehicle.
Dynamically feasible trajectories must be planned to enable the UAV to avoid
obstacles during its approach to the landing target.
This problem is especially important when a UAV must land on a large boat
with significant superstructure or when a UAV must re-enter and land inside of a
package delivery truck. While planning trajectories for UAVs with respect to
static obstacles has been widely researched, planning trajectories to avoid
dynamic obstacles that are rigidly attached to the desired landing target is its
own, unique problem to be solved.

When operating from vehicles with complex motion, 
it can also be important to precisely time the touchdown of the UAV.
% potential damage to the UAV.
For instance, when landing on a boat in rough
waters, it may be desireable that the UAV touches down when the heave motion of the
boat reaches a peak to minimize potential damage to the UAV.
% avoid rough contact between the UAV
% and the boat.
Further research should be conducted to plan feasible, time-dependent
trajectories for these scenarios.

% This type of motion planning presents many unsolved challenges.
% While this type of motion planning also requires very detailed
% estimation and prediction of the motion of the boat, the motion
% planning problem itself presents many unsolved challenges.


\subsection{Control}
As discussed in Chapter~\ref{chp:control_paper}, the controller presented is
based in Lie theory, operating on the error-state of the system.
% to accurately track a time-dependent trajectory. While
While operating on the error-state provides other
advantages, the main advantage to this approach is that it should
outperform typical Euler angle formulations during extreme maneuvers.
% perform better during more extreme maneuvers than a typical ZYX Euler angle
% formulation.
This controller should be tested with feasible trajectories of acrobatic
maneuvers to demonstrate this advantage.
% Simulation and hardware experiments should be conducted to show the
% advantage this error-state controller presents over existing control schemes.

One key assumption made by the controller is that the
% One potential down side to the proposed controller is that it assumes the
error-state of the system is always small. This was
especially not true during takeoff, when the UAV was far from the sinusoidal
desired trajectory.
% always true as the UAV had to converge onto the sinusoidal trajectory from stationary
% takeoff.
This resulted in unsteady transient behavior when the error-state values were
not saturated.
% To satisfy this assumption,
% the true performance of the controller,
The controller should be tested with
dynamically planned trajectories which always begin at the current state of the
UAV. With these trajectories, the error-state of the system would always be forced to
be small, satisfying this key assumption.
% small
% always remain small, satisfying the key assumption of the controller.
% The experiments described in~\ref{chp:control_paper} used a simple sinusoidal
% trajectory to demonstrate the performance of the controller. However, the
% transient behavior from take off to convergence onto the trajectory was at times
% unstable. This is a result of the 

