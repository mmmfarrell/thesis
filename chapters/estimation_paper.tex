% !TEX root=../master.tex

\chapter{Improving State Estimation of a Landing Vehicle \\
From a Multirotor UAV Using Feature Tracking}
\label{chp:estimation_paper}

\graphicspath{{estimation_paper/}}

\section{Introduction} \label{sec:intro}
% !TEX root=../../master.tex

Small multirotor unmanned air vehicles (UAVs) have rapidly become popular platforms for
a variety of applications including
inspection, reconnaissance, and search and rescue.
% The ability of small multirotor UAVs to agily operate in confined spaces and to
% take off and land vertically give them a unique advantage over other robotic
% platforms.
For many of these use cases, UAVs are required to operate
autonomously, as skilled pilots are not feasible
since they are often unable to maintain direct line of sight to the UAV.
% due to
% limitations in line of sight. 
% A variety of emerging use cases
% require multirotor UAVs to operate from larger, mobile vehicles.
% a moving vehicle.
% instead of from the static world.
% These use
% cases include martitime surveillance, where the UAV must operate from
% a martitime vessel at sea,
% and package delivery, where the UAV must operate from a large truck in motion.
Newly emerging use cases such as maritime surveillance and package delivery
pose unique problems, requiring UAVs to operate autonomously from larger,
mobile vehicles instead of from a stationary base station.
% require UAVs to operate autonomously from larger, mobile vehicles.
% which carries the packages to be delivered.
% The ability to operate reliably from a
% moving vehicle is still a very active field of research, posing many unsolved
% problems.

% \cite{ling2014precision} AprilTag, kalman filter to predict
% \cite{araar2017vision} relies on a known map of many AprilTags fiducials.
% \cite{borowczyk2017autonomous} uses a single AprilTag, fast car
% \cite{baca2019autonomous} main MBZIRC 2017 paper (tag = square with x)
% \cite{falanga2017vision} MBZIRC, SVO + IMU
% \cite{beul2017fast} MBZIRC
% \cite{cantelli2017autonomous} MBZIRC
% \cite{marantos2018vision} helicopter, tag (Aruco/April)

% \cite{lee2012autonomous} IBVS
% \cite{wynn2019visual} IBVS - nested tag

Nearly all current approaches to the operation of multirotor UAVs with respect to moving
vehicles rely on the
% A variety of approaches to the operation of multirotor UAVs with respect to
% moving vehicles have been proposed in literature.
% Nearly all of these approaches rely on the
detection of a fiducial marker on the moving vehicle for relative pose
measurements. One of the earliest of these works
used a known configuration of infrared LEDs on the landing vehicle as a fiducial
marker~\cite{wenzel2011automatic}.
% relied on the detection of infrared LEDs that were
% illuminated in a known configuration on the landing
% vehicle~\cite{wenzel2011automatic}.
Since then, visual fiducial markers such as
AprilTags~\cite{olson2011tags} and ArUco markers~\cite{garrido2016generation}
have become more widely
used~\cite{ling2014precision,borowczyk2017autonomous,marantos2018vision,
araar2017vision}.
% In 2017, the Mohamed Bin Zayed International Robotics Challenge (MBZIRC)
% featured a stage that included landing a multirotor UAV on a visual fiducial
% marker atop a moving golf
% cart~\cite{baca2019autonomous,falanga2017vision,beul2017fast,cantelli2017autonomous}.
% While some landing methods employ image-based visual
% servoing~\cite{lee2012autonomous,wynn2019visual}, most methods
% require an estimate of the state of the target landing vehicle.
While some landing methods control the UAV entirely based on the detections of the
fiducial marker~\cite{lee2012autonomous,wynn2019visual}, more robust methods
compute control based on an estimate of the state of the target landing
vehicle~\cite{ling2014precision}.

% The Kalman filter~\cite{kalman} is a widely used algorithm 
% As we should not expect uninterrupted detection of the fiducial marker during
% landing, it is
% important to model the dynamics of the target vehicle and use them to propagate
% forward
As the UAV descends toward the landing target, it is common
that the fiducial marker remains undetected
% to not detect the fiducial marker
for periods of time due to poor lighting, occlusion, or extreme
motion.
% For this reason, it is important that a model of the dynamics of the target vehicle
For this reason, it is important that the dynamics of the target vehicle are
modeled and
used by the estimation algorithm to predict the state of the target vehicle
when measurements are not available.
The Kalman filter~\cite{kalman} has been frequently used
for this task, producing accurate estimates 
% showing good results 
when the fiducial marker is not detected for
short periods of time~\cite{baca2019autonomous}.
% n estimation method which models
% the dynamics of the target vehicle is employed to maintain accurate estimates in
% between measurements.
% Ling notes in~\cite{ling2014precision} that it is important to model the
% dynamics of the target vehicle so that they can be propagated forward for short
% periods of time when the fiducial marker is not
% detected.
Due to imperfect motion models, however,
all of the mentioned
approaches are likely to fail if the fiducial marker is not detected for
significant periods of time.
% due to imperfect motion models.
% because the target vehicle is likely to not
% perfectly follow the modeled motion.
% Even if the state of the target vehicle is estimated, the estimates are only as
% good as the
% motion model of the target vehicle when the fiducial marker is not detected.
% We propose an improvement to these methods: 
% To improve these methods, we propose an estimation algorithm that detects,
% tracks, and estimates the positions of unknown visual features on the target
% vehicle in addition to the fiducial marker.
To improve these methods, we propose an estimation algorithm that uses
measurements of unknown visual features on the target vehicle in addition to
measurements resulting from detections of a fiducial marker.
% to improve
% estimation accuracy when the fiducial marker is not detected for significant
% periods of time.

% Tracking and estimating the positions of visual features is
% a common technique used
% % similar to that commonly used
% in the field of visual odometry.
Many visual odometry methods such
as~\cite{qin2018vins,leutenegger2013keyframe,mourikis2007multi,mur2015orb} use
detected and tracked visual features to aid in camera motion estimation.
% as VINS-MONO, OKVIS, MSCKF, and ORB SLAM use an indirect approach
% \cite{qin2018vins} VINS-MONO, feature extraction, optical flow frame to frame
% \cite{leutenegger2013keyframe} OKVIS, BRISK features
% \cite{mourikis2007multi} MSCKF, uses SIFT features
% \cite{mur2015orb} ORB SLAM
In these methods, the tracked visual features are assumed to belong to the
static world.
Visual odometry techniques
% tracking static features
have also been previously applied to landing on
moving platforms~\cite{falanga2017vision}.
During landing, however, static features become more sparse as the dynamic target vehicle
occupies progessively more of the
field of view of the UAV's camera. This results in detoriating
estimates as the UAV approaches the landing target.
% During landing, however, it is common for the target vehicle to occupy the
% entire field of view of the UAV's camera, making it impossible to track static
% features throughout the duration of the flight.
% While visual odometry techniques have been
% previously applied to landing on
% moving platforms~\cite{falanga2017vision}, these methods still only tracked and
% estimated static features.
% During the landing phase, it is not uncommon, however, for the target vehicle
% to occupy the entire field of view of the UAV's camera, making it not possible
% to track static features throughout the duration of the flight.
For this reason, the proposed estimator, instead, tracks and estimates the
locations of
visual features that are rigidly attached to the target vehicle. These tracked
visual features provide information about the relative position of the target
vehicle
as long as the vehicle remains in the field of view of the camera.
% for the duration of the flight.
% movement of the UAV as well as
% information about the movement of the target vehicle.
We show in simulation and hardware experiments that the estimation of these tracked features allows
for accurate estimates of the state of the target vehicle even
when the fiducial marker is not detected for significant periods of time.


% !TEX root=../root.tex

The outline of this chapter is as follows.
\secref{sec:math_prelim} explains the mathematical notation and conventions used
throughout the chapter.
\secref{sec:estimation} presents the proposed estimation algorithm including the
state dynamics, state initialization and measurement models.
\secref{sec:est_paper_simulation} describes the simulation experiments conducted and
\secref{sec:est_paper_hardware} describes the hardware experiments conducted.
\secref{sec:est_paper_conclusion} provides concluding remarks.


\section{Mathematical Conventions} \label{sec:model}
% !TEX root=../root.tex

\subsection{Notation}

Throughout the paper, we represent vectors with a bold letter (e.g., $\bf v$)
and matrices with a captial letter (e.g., $A$). Other common notation used
throughout the paper is contained below.
\begin{center}
\begin{tabularx}{\columnwidth}{lX}
$R_a^b$ & Rotation matrix from reference frame $a$ to $b$ \\
$\vect{v}_{a/b}^c$ & Vector state $\vect{v}$ of frame $a$ w.r.t.~frame $b$, expressed in frame $c$ \\
$\hat{a}$ & Estimate of true variable $a$ \\
$\bar{a}$ & Measurement of $a$ \\
% $\des{a}$ & Desired value of $a$ \\
$\dot{a}$ & Time derivative of $a$ \\
$\tilde{a}$ & Error of variable $a$, i.e., $\tilde{a} \triangleq a - \hat{a}$
\end{tabularx}
\end{center}
%
We also make use of the following coordinate frames:
\begin{center}
\begin{tabularx}{\columnwidth}{lX}
$I$ & The inertial coordinate frame in north-east-down\\
% $\ell$ & The aircraft's vehicle-1 (body-level) coordinate frame \\
$v$ & The aircraft's vehicle-1 (body-level) coordinate frame \\
$b$ & The aircraft's body-fixed coordinate frame \\
$c$ & The camera frame \\
$g$ & The target landing vehicle's body-fixed coordinate frame located at the desired
landing location (goal) of the aircraft
\end{tabularx}
\end{center}
%
We use the standard basis vectors $\vect{e}_1, \vect{e}_2, \vect{e}_3$,
where $\vect{e}_1 = \begin{bmatrix} 1 & 0 & 0 \end{bmatrix}^\transpose$,
$\vect{e}_2 = \begin{bmatrix} 0 & 1 & 0 \end{bmatrix}^\transpose$,
and $\vect{e}_3 = \begin{bmatrix} 0 & 0 & 1 \end{bmatrix}^\transpose$.
We also use the skew-symmetric matrix operator
% We make frequent use of the skew-symmetric matrix operator defined by
\begin{equation}
  \skewmat{\vect{v}} \triangleq
  \begin{bmatrix}
  0 & -v_{3} & v_{2}\\
  v_{3} & 0 & -v_{1}\\
  -v_{2} & v_{1} & 0
  \end{bmatrix},
\end{equation}
which is related to the cross-product between two vectors as
\begin{equation}
  \vect{v}\times\vect{w}=\skewmat{\vect{v}}\vect{w},
\end{equation}
and the skew-symmetric identity
\begin{equation}
  \skewmat{\vect{v}} \vect{w} = -\skewmat{\vect{w}} \vect{v}.
\end{equation}
We also use the two-dimensional skew-symmetric operator which acts on a scalar
\begin{equation}
  \skewmat{a} \triangleq
  \begin{bmatrix}
  0 & -a \\
  a & 0
  \end{bmatrix}.
\end{equation}
We use the identity matrix, $I$, as well as submatricies of $I$ denoted with
subscripts such as
\begin{equation}
  I_{2 \times 3} =
  \begin{bmatrix}
    1 & 0 & 0 \\
    0 & 1 & 0
  \end{bmatrix}.
\end{equation}

% !TEX root=../root.tex

\subsection{Quaternion Representation}
% \MF{TODO what of this is necessary}
% A quaternion $\q$ is a hyper-complex number of rank four.  It is well known that
% a unit quaternion $\in S^3$ can be used to efficiently represent attitude, as
% $S^3$ is a double cover of $\SO(3)$.  Quaternions have the advantage over
% $\SO(3)$ of being more efficient to implement on modern hardware~\cite{casey2013AttitudeRepresentation}, therefore in the software implementation of the described algorithm, we use quaternions, rather than rotation matrices.

% We use the Hamiltonian notation for unit quaternions $\in S^3$
Unit quaternions $\in S^3$ throughout the paper follow the Hamiltonian notation
\begin{equation}
  % \q=\begin{pmatrix}q_{0} & q_{x}i & q_{y} j & q_{z} k \end{pmatrix}.%=\begin{pmatrix}q_{0} & \bar{\q} \end{pmatrix},
  \q= q_{0} + q_{x}i + q_{y} j + q_{z} k .%=\begin{pmatrix}q_{0} & \bar{\q} \end{pmatrix},
\end{equation}
where $i$, $j$, and $k$ are the fundamental quaternion units.
% We write the complex p
% For convenience, we sometimes refer to the complex portion of the quaternion as
% \begin{equation}
	% \bar{\q} = \begin{bmatrix} q_x & q_y & q_z \end{bmatrix} ^\transpose
% \end{equation}
% and write quaternions as the tuple of the real and complex portions
We write quaternions as the tuple
\begin{equation}
	\q = \begin{pmatrix} q_0 \\ \bar{\q} \end{pmatrix}
\end{equation}
where $q_0$ represents the real portion of the quaternion and
% and the complex
% portion of the quaternion is given by
\begin{equation}
	\bar{\q} = \begin{bmatrix} q_x & q_y & q_z \end{bmatrix} ^\transpose
\end{equation}
respresents the complex portion of the quaternion.

% Given our use of the Hamiltonian notation, the quaternion group operator
Given this representation of the quaternion, the quaternion group operator
$\otimes$ can be written as the matrix-like products
\begin{align}
  \q^a \otimes \q^b &= \begin{pmatrix} q_0^a & \left(-\bar{\q}^{a}\right)^\transpose \\ \bar{\q}^a & q^a_0 I + \skewmat{\bar{\q}^a} \end{pmatrix}
	\begin{pmatrix} q_0^b \\ \bar{\q}^b\end{pmatrix}
  \label{eq:quat_otimes_product1} \\
  &= \begin{pmatrix} q_0^b & \left(-\bar{\q}^{b}\right)^\transpose \\ \bar{\q}^b & q^b_0 I - \skewmat{\bar{\q}^b} \end{pmatrix}
	\begin{pmatrix} q_0^a \\ \bar{\q}^a \end{pmatrix}.
  \label{eq:quat_otimes_product2}
\end{align}
% It is often convenient to convert a quaternion $\q$ to its associated passive rotation matrix.  This is done with
We frequently convert a quaternion $\q$ to its associated passive rotation
matrix. This is done with
\begin{equation}
R\left(\q\right)=\left(2q_{0}^{2}-1\right)I-2q_{0}\skewmat{\bar{\q}}+2\bar{\q}\bar{\q}^{\top}\in SO\left(3\right).
\label{eq:est_paper_R_from_q}
\end{equation}
We also note that rotations are written equivalently as
$\q_a^b=R\left(\q_a^b\right)=R_a^b$ throughout the paper.
% where the choice of these is dictated by convenience.
% We use passive rotation matrices, meaning that the rotation matrix $R_a^b$ acts
% on a vector $\vect{r}^a$, expressed in frame $a$, and results in the same
% vector, now expressed in frame $b$ as
% \begin{equation}
% \vect{r}^b = R_a^b \vect{r}^a .
% \end{equation}

% We need to frequently convert between the Lie group, $S^3$, and the Lie
% algebra, $\mathbb{R}^3$, which enables us to operate in a vector space.
To operate in a vector space, we frequently convert between the Lie group,
$S^3$,
% and the Lie algebra, $\mathbb{R}^3$.
and the vector space, $\mathbb{R}^3$, that is isomorphic to the Lie algebra.
This is done with the
exponential and logarithmic mappings. The exponential mapping for a unit quaternion is defined as
\begin{align}
\exp_{\q} & :\mathbb{R}^{3}\rightarrow S^3\nonumber \\
\exp_{\q}\left(\vect{\delta}\right) & \triangleq\begin{bmatrix}\cos\left(\frac{\lVert\vect{\delta}\rVert}{2}\right)\\
\sin\left(\frac{\lVert\vect{\delta}\rVert}{2}\right)\frac{\vect{\delta}}{\lVert\vect{\delta}\rVert}
\end{bmatrix},\label{eqn:exponential_map}
\end{align}
with the corresponding logarithmic map defined as
\begin{align}
\log_{\q} & :S^3\rightarrow \mathbb{R}^{3}\nonumber \\
\log_{\q}\left(\q\right) & \triangleq2\;\mathrm{atan2}\left(\left\Vert \bar{\q}\right\Vert ,q_{0}\right)\frac{\bar{\q}}{\left\Vert \bar{\q}\right\Vert }.\label{eqn:log_map}
\end{align}
% To avoid numerical issues when $\norm{\vect{\delta}}\approx0$, we also employ the small-angle approximations of the quaternion exponential and logarithm
With this formulation, $\vect{\delta}$ also represents the axis-angle
representation of a rotation where the unit vector
$\frac{\vect{\delta}}{\norm{\vect{\delta}}}$ represents the axis of rotation,
and $\norm{\vect{\delta}}$ represents the angular magnitude of rotation.
When $\norm{\vect{\delta}}\approx0$,
we employ the small-angle approximations of the quaternion exponential and
logarithm given by
\begin{align}
\exp_{\q}\left(\vect{\delta}\right) & \approx\begin{bmatrix}1\\
\frac{\vect{\delta}}{2}
\end{bmatrix}\label{eq:est_paper_quaternion_exp_approx}\\
\log_{\q}\left(\q\right) & \approx2\;\textrm{sign}\left(q_{0}\right)\bar{\q}.\label{eq:quaternion_log_approx}
\end{align}


% The $\boxplus$/$\boxminus$ notation for unit quaternions is now defined by
% \begin{align*}
% \boxplus: & S^3\times\mathbb{R}^{3}\rightarrow S^3\\
%  & \q\boxplus\vect{\delta}\triangleq\q\otimes\exp_q\left(\vect{\delta}\right)\\
% \boxminus: & S^3\times S^3\rightarrow\mathbb{R}^{3}\\
%  & \q\boxminus\vect{p}\triangleq\log_q\left(\vect{p}^{-1}\otimes\q\right).
% \end{align*}
% Let's look at one application for clarification.
% With body angular rate $\vect{\omega}_{b/i}^b$, inertial to body rotation $\q_i^b$, and $\vect{\theta}=\vect{\omega}_{b/i}^b \Delta t$, we have
% \begin{align}
% \q_i^b\left(t+\Delta t\right) & =\q_i^b\left(t\right)\boxplus\vect{\theta}\\
% \vect{\theta} & =\q_i^b\left(t+\Delta t\right)\boxminus\q_i^b\left(t\right).
% \end{align}

%\subsection{Quadrotor Dynamics With Quaternion Attitude}
%In the associated software library, we employ the following dynamics for the quadrotor:
%\begin{align}
    %\dot{\vect{p}}_{b/I}^{I} &= 	R\left(\q_{I}^{b}\right)^\transpose \vect{v}_{b/I}^{b} \nonumber \\
	%\dot{\q}_{I}^{b} &= - \frac{1}{2} \begin{pmatrix} 0 \\ \omega_{b/I}^b\end{pmatrix} \otimes \q_I^b\nonumber\\
	%\dot{\vect{v}}_{b/I}^{b} 
	%&= 	g\frac{s}{s_e}\vect{e}_3 + gR\left(\q_I^b\right) \vect{e}_3- \drag\vect{v}_{b/I}^b - \skewmat{\vect{\omega}_{b/I}^b}\vect{v}_{b/I}^b.
	%\label{eq:quat_dynamics}
%\end{align}
%and we solve this using a fourth-order Runge-Kutta extension of \eqref{eq:quaternion_integration}.

% TODO maybe pull from here
%\subsection{Quaternion Error State Dynamics}
%As with the rotation matrix, we wish to find a minimal representation of the error about a quaternion attitude state.  Let us start with the differential equation for quaternion dynamics
%\begin{equation}
	%\dot{\q}_{I}^{b} = \frac{1}{2}\q_I^b \otimes \begin{pmatrix} 0 \\ \vect{\omega}_{b/I}^b\end{pmatrix}.
%\end{equation}
%This can be solved using the quaternion exponential with
%\begin{equation}
	%\q_I^b\left(t\right) = \q_I^b\left(0\right)\otimes\exp_{\q}\left(\int_0^t \vect{\omega}_{b/I}^b\left(\tau\right) d\tau \right).
%\end{equation}
%To reduce this to our minimal vector representation, as we did with rotation matrices in \eqref{eq:att_vector}, let us define the attitude vector $\vect{r}_I^b\left(t\right)=\vect{r}_I^b\left(t_0\right)+\int_{t_0}^t\boldsymbol{\omega}_{b/I}^b\left(\tau\right)d\tau$ with $\vect{r}_I^b\left(t_0\right)=0$ such that
%\begin{equation}
	%\q_I^b\left(t\right) = \q_I^b\left(0\right)\otimes\exp_{\q}\left(\vect{r}_I^b\left(t\right) \right).
%\end{equation}
%It then follows that the error state of our attitude vector is
%\begin{align}
	%\tilde{\vect{r}}_I^{b} 
		%&= \vect{r}_I^b - \hat{\vect{r}}_I^{\hat{b}}  \nonumber \\
		%&= \vect{r}_I^b - R\left(\q_I^b\right) R\left(\q_I^{\hat{b}}\right)^\transpose \hat{\vect{r}}_I^{\hat{b}}\nonumber \\
		%&= \vect{r}_I^b - R\left(\tilde{\q}_{\hat{b}}^b\right)  \hat{\vect{r}}_I^{\hat{b}}
%\end{align}
%and its time derivative is
%\begin{equation}
	%\dot{\tilde{\vect{r}}}_I^{b} = \dot{\vect{r}}_I^b - R\left(\tilde{\q}_{\hat{b}}^b\right)  \dot{\hat{\vect{r}}}_I^{\hat{b}}.
%\end{equation}
%\subsection{Quaternion Trajectory Generation}
%When creating trajectories using differential flatness for quaternion attitude states, the equivalent expression to \eqref{eq:des_attitude} is
%\begin{equation}
	%\des{q}_I^b = \exp_{\q}\left(\vect{e}_3\des{\psi}_{b/I}^I)\right) \otimes \exp_{\q}\left(\theta \vect{\delta}\right) \label{eq:des_attitude}
%\end{equation}
%where, as in \eqref{eq:des_attitude}, $\theta\vect{\delta}$ is the axis-angle error between the desired acceleration and the inertial z-axis.

% TEX root=../root.tex

\subsection{Planar Rotations}
\label{sec:planar_rotations}
% \subsection{2D Rotation Group}
% We use rotation matrices, the Lie group $SO(2)$, to represent 2D rotations.
% We parameterize the 2D rotation matrix by the angle, $\psi$, such that
% \begin{align}
  % \begin{array}{llll}
    % R &= \exp_R \left( \psi \right) &=
  % \begin{bmatrix}
    % \cos \psi & - \sin \psi \\
    % \sin \psi & \cos \psi
  % \end{bmatrix} & \in \mathbb{R}^{2 \times 2} \\
    % \psi &= \log_R \left( \psi \right) &= \mathrm{atan2} \left(r_{21},r_{11}
      % \right) & \in \mathbb{R}.
  % \end{array}
% \end{align}
% We also note that planar rotations are commutative, meaning
% \begin{equation}
  % R_b^c R_a^b = R_a^b R_b^c \qquad \forall R_a^b, R_b^c \in SO(2).
% \end{equation}

% We parameterize the 2D rotation group as an angle, $\psi$, which respresents
We parameterize planar rotations as an angle, $\psi$, which respresents
the angle of rotation about a given axis.
% We treat $\psi$ as a vector space
% that uses the common addition and subtraction operators such that
We treat $\psi \in \mathbb{R}^1$ so that common addition and subtraction
operators can be used such as
\begin{align}
  \psi_a^c &= \psi_a^b + \psi_b^c \\
  \psi_a^b &= \psi_a^c - \psi_b^c.
\end{align}
We note, however, that with this formulation, all addition and subtraction
operations must be wrapped such
that the resultant angle $\psi \in \left[ \right. -\pi, \pi \left. \right)$. The
passive 2D
rotation matrix can be created from any $\psi$ as
\begin{equation}
  R \left( \psi \right) = \begin{bmatrix} \cos \psi & -\sin \psi \\ \sin \psi &
  \cos \psi \end{bmatrix}.
\end{equation}
If $\psi$ represents the angle of rotation about the $z$ axis of a reference frame, then
the corresponding passive 3D rotation matrix is given by
\begin{equation}
  R \left( \psi \right) = \begin{bmatrix}
    \cos \psi & -\sin \psi & 0 \\
    \sin \psi & \cos \psi & 0 \\
    0 & 0 & 1
  \end{bmatrix}.
\end{equation}

% We treat S1 as a vector space but make sure to wrap residuals between -pi and
% pi. 
% Also a 2D rotation matrix can be made like this...
% And a 3D rotation matrix like this..


\section{Estimation} \label{sec:estimation}
% !TEX root=./root.tex

% The propsed estimator estimates both the state of
% the UAV and the state of the target vehicle as well as the positions of visual
% landmarks on the target vehicle.
% in the same Error-State Kalman
% filter~\cite{sola2017quaternion}.
% Though
% This work focuses on the estimation of the state of a landing target
% vehicle. However, we also estimate the state of the UAV to properly account for the
% uncertainty in the UAV states that appear in the motion model of the target
% vehicle and the measurement
% models used. states upon which many of the measurement models depend.
This work focuses on the estimation of the state of the landing target vehicle.
However, as several of the measurement models employed depend on the state of the
UAV, we also estimate the state of the UAV in the same filter to properly
account for the uncertainty in its state.
We, therefore, estimate the position, attitude, and velocity of the UAV given by
$\hat{\vect{p}}_{b/I}^{I}$, $\hat{\vect{q}}_I^{b}$, and
$\hat{\vect{v}}_{b/I}^b$.
In addition, we estimate bias states for the acclerometer and gyroscope sensors that are
used as inputs to the filter. These estimated states are given by
$\hat{\vect{\beta}}_a$ and $\hat{\vect{\beta}}_{\omega}$.

The estimated state of the target vehicle is defined as the position, velocity,
attitude, and angular rate of the target vehicle.
% denoted by
% $\hat{\vect{p}}_{g/b}^{v}$, $\hat{\vect{v}}_{g/I}^{g}$, $\hat{\psi}_{I}^{g}$,
% and $\hat{\omega}_{g/I}^{g}$.
We note that the
estimated position of the target vehicle, $\hat{\vect{p}}_{g/b}^v$, is relative to the position of the
UAV.
We formulate this state relatively,
as this relative state is observable even with poor estimates of the UAV's global
position, $\hat{\vect{p}}_{b/I}^I$, due to the relative information provided
by the measurements as described in~\secref{sec:measurement_models}.
For this work, we assume that the target vehicle's
motion is constrained to a two-dimensional plane. This means that the estimated
target vehicle velocity, $\hat{\vect{v}}_{g/I}^{g}$, is only of two dimensions and that the
estimated attitude, $\hat{\psi}_{I}^g$, represents a planar rotation as
described in~\secref{sec:planar_rotations}, implying that the estimated angular
rate, $\hat{\omega}_{g/I}^g$ is of one dimension.

As previously mentioned, we also estimate the locations of unknown visual
features on the target vehicle.
The vectors $\hat{\vect{r}}_{1/g}^{g}, \dots \hat{\vect{r}}_{n/g}^{g}$ represent the
estimated locations of visual features $1, \dots n$ with respect to the goal frame and
expressed in the goal frame. As we assume these features are rigidly attached to the
target vehicle, these vectors remain constant as the vehicle moves and rotates.
We show in the experiments described in~\secref{sec:est_paper_simulation}
and~\secref{sec:est_paper_hardware} that the
addition of only ten visual features to the estimated state significantly
improves the estimates of the state of the target vehicle
% $\hat{\x}_{\text{Goal}}$
while the fiducial landing marker is not detected. 

% We, therefore, estimate the position,
% attitude, and velocity of the UAV together with the position, attitude, velocity,
% and angular rate of the target vehicle and the positions of visual features on
% the target vehicle. We also estimate bias states for the
% accelerometer and gyroscope sensors that are used as inputs to the filter.
% to properly
% account for the uncertainty, as several of the 
We express the full state of the estimated system as the tuple
\begin{equation}
  \hat{\x} =
  \begin{pmatrix}
    \hat{\x}_{\text{UAV}}, \hat{\x}_{\text{Goal}}, \hat{\x}_{\text{Features}}
  \end{pmatrix}
\end{equation}
with the components defined as
\begin{align}
  \hat{\vect{x}}_{\text{UAV}} &=
  \begin{pmatrix}
    \hat{\vect{p}}_{b/I}^{I}, \hat{\vect{q}}_I^{b}, \hat{\vect{v}}_{b/I}^b,
    \hat{\vect{\beta}}_a,
    \hat{\vect{\beta}}_{\omega}
  \end{pmatrix}
    \in \mathbb{R}^3 \times S^3 \times \mathbb{R}^3 \times \mathbb{R}^3 \times
    \mathbb{R}^3  \\
  % \x_{\text{UAV}} &=
  % \begin{bmatrix}
    % \vect{p}_{b/I}^I &
    % \phi & \theta & \psi &
    % \vect{v}_{b/I}^b &
    % \mu & \vect{\beta}_a & \vect{\beta}_\omega
  % \end{bmatrix}^\transpose \\
    \hat{\x}_{\text{Goal}} & =
    \begin{pmatrix}
      \hat{\vect{p}}_{g/b}^{v}, \hat{\vect{v}}_{g/I}^{g}, \hat{\psi}_{I}^{g},
      \hat{\omega}_{g/I}^{g}
    \end{pmatrix}
    \in \mathbb{R}^3 \times \mathbb{R}^2 \times \mathbb{R}^1 \times \mathbb{R}^1
    \\
    \hat{\x}_{\text{Features}} & =
    \begin{pmatrix}
      \hat{\vect{r}}_{1/g}^{g}, \dots \hat{\vect{r}}_{n/g}^{g}
    \end{pmatrix}
    \in \mathbb{R}^3 \times \dots \mathbb{R}^3.
\end{align}
The inputs to the estimated system are given by
\begin{equation}
  \vect{u} = \begin{pmatrix} \bar{\vect{a}}_{b/I}^b, \bar{\vect{\omega}}_{b/I}^b \end{pmatrix} \in
        \mathbb{R}^3 \times \mathbb{R}^3,
\end{equation}
which are directly measured from an inertial measurement unit mounted on the UAV.
% Though
% this work focuses on the estimation of the state of the landing target
% vehicle, we also estimate the state of the UAV to properly account for the
% uncertainty in these states upon which the motion model for
% $\x_{\text{Goal}}$ and many of the measurement models depend.

% The estimated states associated with the UAV, $\hat{\vect{x}}_{\text{UAV}}$,
% contain the traditionally estimated states of position, attitude, and velocity
% in addition to bias states for the accelerometer and gyroscope sensors.

% The estimated states associated with the target vehicle,
% $\hat{\vect{x}}_{\text{Goal}}$, contain the position, attitude, velocity, and
% angular velocity of the target vehicle.
% We note that the
% estimated position of the target vehicle, $\hat{\vect{p}}_{g/b}^v$, is relative to the position of the
% UAV.
% We formulate this state relatively,
% as this relative state is observable even with poor estimates of the UAV's global
% position, $\hat{\vect{p}}_{g/I}^I$, due to the relative information provided
% by the measurements as described in~\secref{sec:measurement_models}.
% Note that $\hat{\vect{x}}_{\text{UAV}}$ contains the same states mentioned previously
% in~\secref{sec:UAV_dynamics} with the addition of $\hat{\vect{\beta}}_a$ and
% $\hat{\vect{\beta}}_\omega$, the estimated bias vectors for the acclerometer and
% gyroscope sensors. On the
% other hand, $\hat{\vect{x}}_{\text{Goal}}$ varies 
% from the target vehicle states mentioned in~\secref{sec:landing_veh_dynamics}
% by containing $\hat{\vect{p}}_{g/b}^v$ instead of $\hat{\vect{p}}_{b/I}^I$.
% % as well as $\hat{\vect{r}}_{1/g}^{g} \dots \hat{\vect{r}}_{n/g}^{g}$.
% We estimate the relative state, $\hat{\vect{p}}_{g/b}^v$, instead
% of the global state, $\hat{\vect{p}}_{g/I}^I$, as the relative state is observable
% even with poor estimates of the UAV's global position, $\hat{\vect{p}}_{b/I}^I$.

% The estimated vectors $\hat{\vect{r}}_{1/g}^{g}, \dots \hat{\vect{r}}_{n/g}^{g}$ represent the
% locations of visual features $1, \dots n$ with respect to the goal frame and
% expressed in the goal frame. As we assume these features are rigidly attached to the
% target vehicle, these vectors remain constant as the vehicle moves and rotates.
% We show in the experiments described in~\secref{sec:est_paper_simulation}
% and~\secref{sec:est_paper_hardware} that the
% addition of only ten visual features to the estimated state significantly
% improves the estimates of the state of the target vehicle
% % $\hat{\x}_{\text{Goal}}$
% while the fiducial landing marker is not detected. 

We also note
that the estimated state is of dynamic size. As visual features are
detected they are added to the estimated state until a maximum size of the state
is reached. As visual features leave the field of view of the camera,
or are otherwise no longer tracked, they are removed from the estimated state
to make room for new visual features to be added.

% In the following
% subsections, we define the error-state of the estimated system,  thdescribe the dynamic equations used to model the state of the
% estimated system, the
% initialization of certain states, and the specific
% measurement models used to update the filter.

% As mentioned in~\secref{sec:intro??} the estimation of these
% visual landmarks allows the estimator to maintain accurate and consistent
% estimates of the target vehicle even while the fiducial landing marker is not
% detected for long periods of time.

\subsection{Error-State Definition}

As the estimated state is not a vector, but rather a tuple of Lie groups, we
employ the error-state Kalman filter (ESKF) as described in~\cite{koch2017relative}.
% The error-state of the estimated system is defined by
We define the error-state of the estimated system as
\begin{align}
  \tilde{\vect{x}} =&
  \left[ \begin{matrix}
    \tilde{\vect{p}}_{b/I}^{I}, \tilde{\vect{\theta}}_I^{b}, \tilde{\vect{v}}_{b/I}^b,
    \tilde{\vect{\beta}}_a,
    \tilde{\vect{\beta}}_{\omega},
    \tilde{\vect{p}}_{g/b}^{v}, \tilde{\vect{v}}_{g/I}^{g}, \tilde{\psi}_{I}^{g},
    \tilde{\omega}_{g/I}^{g},
      \tilde{\vect{r}}_{1/g}^{g}, \dots \tilde{\vect{r}}_{n/g}^{g}
  \end{matrix} \right]
  \in \mathbb{R}^{22 + 3n}
\end{align}
with the error-state components related to the vector states, $\x_{\vect{v}}$, defined with
the vector subtraction operator as
\begin{equation}
\tilde{\x}_{\vect{v}} \triangleq \x_{\vect{v}} - \hat{\x}_{\vect{v}}
\label{eq:est_paper_vector_error_state}
\end{equation}
such that
\begin{equation}
  \tilde{\vect{p}}_{b/I}^I = \vect{p}_{b/I}^I - \hat{\vect{p}}_{b/I}^I.
  \label{eq:est_paper_uav_pos_err_state}
\end{equation}
We reiterate that we treat $\psi_I^g \in \mathbb{R}^1$
% as a vector space
such that
\begin{equation}
  \tilde{\psi}_I^g = \psi_I^g - \hat{\psi}_I^g.
  \label{eq:2d_att_err_state}
\end{equation}

We follow~\cite{koch2017relative}, defining the error-state of the quaternion,
$\q_I^b$,
as the minimal representation
\begin{equation}
  \tilde{\vect{\theta}}_I^b \triangleq \log_{\q} \left( \left( \hat{\q}_I^b \right)^{-1}
  \otimes \q_I^b \right)
  \label{eq:quat_error_state}
\end{equation}
which implies
\begin{equation}
  \q_I^b  = \hat{\q}_I^b \otimes \exp_{\q} \left( \tilde{\vect{\theta}}_I^b
  \right).
  \label{eq:quat_true_state}
\end{equation}
As rotation matrices concatenate in the order opposite to
quaternions,~\eqref{eq:quat_true_state} can also be expressed as
\begin{equation}
  R_I^b  = R \left( \exp_{\q} \left( \tilde{\vect{\theta}}_I^b \right) \right)
  \hat{R}_I^b.
  \label{eq:rot_true_state}
\end{equation}
% Similarly, we define the error-state of the 2D rotation matrix state, $R_I^g$,
% as
% \begin{equation}
  % \tilde{\psi}_I^g = \log_R \left(  
  % R_I^g \left( \hat{R}_I^g \right)^\transpose \right)
  % \label{eq:2d_rot_err_state}
% \end{equation}

To derive the error-state dynamics and the measurement residual Jacobians in the
following sections, we
use an approximation for~\eqref{eq:rot_true_state} developed by first expanding
the
quaternion exponential using~\eqref{eq:est_paper_quaternion_exp_approx} as
$\tilde{\vect{\theta}}_I^b$ is assumed to be small
\begin{equation}
  R_I^b  \approx R \left( \begin{bmatrix}
      1 \\
    \frac{1}{2} \tilde{\vect{\theta}}_I^b
  \end{bmatrix}\right)
  \hat{R}_I^b.
\end{equation}
We then employ~\eqref{eq:est_paper_R_from_q}, neglecting higher-order terms, to
yield the approximation
\begin{equation}
  R_I^b  \approx 
  \left( I - \skewmat{\tilde{\vect{\theta}}_I^b} \right)
  \hat{R}_I^b.
  \label{eq:est_paper_Rapprox}
\end{equation}
It can similarly be shown that
\begin{equation}
  \left( R_I^b \right)^\transpose  \approx 
  \left( \hat{R}_I^b \right)^\transpose
  \left( I + \skewmat{\tilde{\vect{\theta}}_I^b} \right).
  \label{eq:est_paper_RTapprox}
\end{equation}




% !TEX root=./root.tex

\subsection{Propagation Model}
To model the motion of the UAV, we use common rigid-body kinematics given by
% We use common rigid body kinematics to model the dynamics of
% the UAV given by
\begin{align}
  \dot{\vect{p}}_{b/I}^I
  &=
  \left( R_I^b \right)^\transpose \vect{v}_{b/I}^b
  \label{eq:uav_dynamics}
  \\
  \dot{\vect{q}}_{I}^{b} 
	&= 	
  \q_I^b \otimes \begin{pmatrix} 0 \\ \frac{1}{2}
    \left( \bar{\vect{\omega}}_{b/I}^b - \vect{\beta}_\omega - \vect{\upsilon}_\omega \right)
\end{pmatrix} \nonumber \\
  \dot{\vect{v}}_{b/I}^b 
  &=
  R_I^b \vect{g}^I
  +
  \skewmat{\vect{v}_{b/I}^b}
  \left( \bar{\vect{\omega}}_{b/I}^b - \vect{\beta}_\omega -
  \vect{\upsilon}_\omega \right)
  +
  \left( \bar{\vect{a}}_{b/I}^b - \vect{\beta}_a - \vect{\upsilon}_a \right) \nonumber
  % \vect{\eta}_v 
  \\
  \dot{\vect{\beta}}_a &= \vect{\eta}_{\beta_a} \nonumber
  \\
  \dot{\vect{\beta}}_\omega &= \vect{\eta}_{\beta_\omega}, \nonumber
\end{align}
% $\vect{\eta}_v$, 
where $\vect{g}^I$ represents the gravity vector expressed in the inertial
frame, $\vect{\eta}_{\beta_a}$ and $\vect{\eta}_{\beta_\omega}$ are zero-mean
Gaussian noise processes corresponding to the state dynamics, and
$\vect{\upsilon}_\omega$ and $\vect{\upsilon}_a$ are zero-mean Gaussian noise
processes corresponding to the noise in the inputs to the system.

We model the motion of the landing target vehicle
with a constant-velocity and constant-angular-velocity motion model such that
% Though this is a very simplified model, we show
% in~\secref{sec:simulation} and~\secref{sec:hardware} that it is sufficient for
% cases that do not perfectly match this model. The dynamics of $\x_{\text{Goal}}$ are expressed as
% \begin{align}
  % \dot{\hat{\vect{p}}}_{g/b}^{v} &= \left( \hat{R}_{I}^{g} \right)^\transpose
  % \hat{\vect{v}}_{g/I}^{g} - \left( \hat{R}_{I}^{b} \right)^\transpose
  % \hat{\vect{v}}_{b/I}^{b} \label{eq:goal_dynamics} \\
  % \dot{\hat{\vect{v}}}_{g/I}^{g} &= \vect{0} \nonumber \\
  % \dot{\hat{\theta}}_{I}^{g} &= \hat{\omega}_{g/I}^g \nonumber \\
  % \dot{\hat{\omega}}_{g/I}^{g} &= 0. \nonumber
% \end{align}
\begin{align}
  \dot{\vect{p}}_{g/b}^{v} &= I_{3 \times 2} \left( R_{I}^{g} \right)^\transpose
   \vect{v}_{g/I}^{g} - \left( R_{I}^{b} \right)^\transpose
  \vect{v}_{b/I}^{b} \label{eq:goal_dynamics} \\
  \dot{\vect{v}}_{g/I}^{g} &= \vect{\eta}_{gv} \nonumber \\
  \dot{\psi}_{I}^{g} &= \omega_{g/I}^g \nonumber \\
  % \dot{R}_{I}^{g} &= -\skewmat{ \omega_{g/I}^g } R_I^g \nonumber \\
  \dot{\omega}_{g/I}^{g} &= \eta_{g\omega}, \nonumber
\end{align}
where $\vect{\eta}_{gv}$ and $\eta_{g\omega}$ are zero-mean Gaussian noise
processes. Though this is a simplified motion model for the target
vehicle, we show in~\secref{sec:est_paper_simulation}
and~\secref{sec:est_paper_hardware} that it is
satisfactory for our experiments. We intend the motion model of the target
vehicle to be easily modified for vehicles with more complex motion such
as a boat at sea.

As mentioned previously, we assume the tracked visual features are rigidly
attached to the landing vehicle such that
% there are no dynamics 
% of their associated states
\begin{equation}
  \dot{\vect{r}}_{i/g}^g = \vect{0}. \label{eq:feature_dynamics}
  % \dot{\hat{\vect{r}}}_{i/g}^g = \vect{0}. \label{eq:feature_dynamics}
\end{equation}

In the ESKF, the estimated state is propagated independently of the filter using
the expected value of the modeled dynamics. We use the expected values
of~\eqref{eq:uav_dynamics}, \eqref{eq:goal_dynamics}, and
\eqref{eq:feature_dynamics} given by
% The estimated state is propagated
\begin{align}
  \dot{\hat{\vect{p}}}_{b/I}^I
  &=
  \left( \hat{R}_I^b \right)^\transpose \hat{\vect{v}}_{b/I}^b
  \label{eq:estimated_dynamics}
  \\
  \dot{\hat{\vect{q}}}_{I}^{b} 
  &= 	
  \hat{\q}_I^b \otimes \begin{pmatrix} 0 \\ \frac{1}{2}
  \left( \bar{\vect{\omega}}_{b/I}^b - \hat{\vect{\beta}}_\omega \right)
\end{pmatrix} \nonumber \\
  \dot{\hat{\vect{v}}}_{b/I}^b 
  &=
  \hat{R}_I^b \vect{g}^I
  +
  \skewmat{\hat{\vect{v}}_{b/I}^b}
  \left( \bar{\vect{\omega}}_{b/I}^b - \hat{\vect{\beta}}_\omega \right)
  +
  \left( \bar{\vect{a}}_{b/I}^b - \hat{\vect{\beta}}_a \right) \nonumber
  \\
  \dot{\hat{\vect{\beta}}}_a &= \vect{0} \nonumber
  \\
  \dot{\hat{\vect{\beta}}}_\omega &= \vect{0} \nonumber
  \\
  \dot{\hat{\vect{p}}}_{g/b}^{v} &= I_{3 \times 2} \left( \hat{R}_{I}^{g} \right)^\transpose
   \hat{\vect{v}}_{g/I}^{g} - \left( \hat{R}_{I}^{b} \right)^\transpose
  \hat{\vect{v}}_{b/I}^{b} \nonumber \\
  \dot{\hat{\vect{v}}}_{g/I}^{g} &= \vect{0} \nonumber \\
  \dot{\hat{\psi}}_{I}^{g} &= \hat{\omega}_{g/I}^g \nonumber \\
  % \dot{\hat{R}}_{I}^{g} &= -\skewmat{ \hat{\omega}_{g/I}^g } \hat{R}_I^g \nonumber \\
  \dot{\hat{\omega}}_{g/I}^{g} &= 0 \nonumber \\
  \dot{\hat{\vect{r}}}_{i/g}^g &= \vect{0}. \nonumber
\end{align}

The error-state dynamics used to propagate the filter are found by relating
the modeled true-state dynamics from~\eqref{eq:uav_dynamics},
\eqref{eq:goal_dynamics}, and \eqref{eq:feature_dynamics}
with~\eqref{eq:estimated_dynamics} using the error-state definitions
from~\eqref{eq:est_paper_vector_error_state} and~\eqref{eq:quat_error_state}. The
first-order approximation of the error-state dynamics is given by 
\begin{align}
  \dot{\tilde{\vect{p}}}_{b/I}^I
  &\approx
  \left( \hat{R}_I^b \right)^\transpose \tilde{\vect{v}}_{b/I}^b
  - \left( \hat{R}_I^b \right)^\transpose \skewmat{\hat{\vect{v}}_{b/I}^b}
  \tilde{\vect{\theta}}_I^b
  \\
  \dot{\tilde{\vect{\theta}}}_{I}^{b} 
  &\approx 	
  -\skewmat{ \bar{\vect{\omega}}_{b/I}^b - \hat{\vect{\beta}}_\omega}
    \tilde{\vect{\theta}}_I^b
    - \tilde{\vect{\beta}}_\omega -
    \vect{\upsilon}_\omega
  \nonumber \\
  \dot{\tilde{\vect{v}}}_{b/I}^b 
  &\approx
  \skewmat{ \hat{R}_I^b \vect{g}^I } \tilde{\vect{\theta}}_I^b 
  -
  \skewmat{ \hat{\vect{v}}_{b/I}^b } \tilde{\vect{\beta}}_\omega
  -
  \skewmat{ \hat{\vect{v}}_{b/I}^b } \vect{\upsilon}_\omega
  -
  \skewmat{ \bar{\omega}_{b/I}^b - \hat{\vect{\beta}}_\omega }
  \tilde{\vect{v}}_{b/I}^b
  -
  \tilde{\vect{\beta}}_a
  -
  \vect{\upsilon}_a \nonumber
  \\
  \dot{\tilde{\vect{\beta}}}_a &= \vect{\eta}_{\beta_a} \nonumber
  \\
  \dot{\tilde{\vect{\beta}}}_\omega &= \vect{\eta}_{\beta_\omega} \nonumber
  \\
  \dot{\tilde{\vect{p}}}_{g/b}^{v}
                                  &\approx
  I_{3 \times 2} \left( \hat{R}_{I}^{g} \right)^\transpose
  \tilde{\vect{v}}_{g/I}^g
  +
  I_{3 \times 2} \left( \hat{R}_{I}^{g} \right)^\transpose
  \skewmat{ \tilde{\psi}_I^g } \hat{\vect{v}}_{g/I}^g
  +
  \left( \hat{R}_{I}^{b} \right)^\transpose \skewmat{ \hat{\vect{v}}_{b/I}^{b} } 
  \tilde{\vect{\theta}}_I^b
  -
  \left( \hat{R}_{I}^{b} \right)^\transpose \tilde{\vect{v}}_{b/I}^{b} \nonumber \\
  \dot{\tilde{\vect{v}}}_{g/I}^{g} &= \vect{\eta}_{gv} \nonumber \\
  \dot{\tilde{\psi}}_{I}^{g} &= \tilde{\omega}_{g/I}^g \nonumber \\
  \dot{\tilde{\omega}}_{g/I}^{g} &= \eta_{g\omega} \nonumber \\
  \dot{\tilde{\vect{r}}}_{i/g}^g &= \vect{0}, \nonumber
\end{align}
or succinctly,
\begin{equation}
  \dot{\tilde{\x}} = f\left(\x, \tilde{\x}, \vect{u}, \tilde{\vect{u}}\right).
\end{equation}
The derivation of these error-state dynamics can be found in
Appendix~\ref{apdx:estimation_err_state_derivation}. In practice, the expected
value of the error state remains zero over the propagation window, and only the
error covariance, $P$, is propagated.
The continous-time derivative of the error covariance is given by
% forward using a first-order approximation
% of the continous-time derivative of covariance given by
\begin{equation}
  \dot{P} = FP + PF^\transpose + G Q_{\vect{u}} G^\transpose + Q_{\vect{x}}
  % P^{+} = FP + PF^\transpose + G Q_{\vect{u}} G^\transpose + Q_{\vect{x}}
\end{equation}
where $Q_{\vect{u}}$ is the input noise covariance, $Q_{\vect{x}}$ is the
process noise covariance,
% \MF{TODO define this matrix by vertical slices}
% \begin{align}
  % F &= \frac{ \partial \dot{\tilde{\x}} }{ \partial \tilde{\x} } \\
    % &=
    % \scriptsize
    % \begin{bmatrix}
    % % \begin{smallbmatrix}
      % \vect{0} & - \left( \hat{R}_I^b \right)^\transpose
      % \skewmat{\hat{\vect{v}}_{b/I}^b} & \left( \hat{R}_I^b \right)^\transpose &
      % \vect{0} & \vect{0} & \vect{0} & \vect{0}
               % & 0 & 0 & \vect{0} \\
      % \vect{0} & -\skewmat{ \bar{\vect{\omega}}_{b/I}^b
      % - \hat{\vect{\beta}}_\omega} & \vect{0} & \vect{0} & -I & \vect{0} & \vect{0}
               % & 0 & 0 & \vect{0} \\
      % \vect{0} & \skewmat{ \hat{R}_I^b \vect{g} } &
      % -\skewmat{ \bar{\omega}_{b/I}^b - \hat{\vect{\beta}}_\omega } & -I &
      % -\skewmat{ \hat{\vect{v}}_{b/I}^b } & \vect{0} & \vect{0}
               % & 0 & 0 & \vect{0} \\
      % \vect{0} & \vect{0} & \vect{0} & \vect{0} & \vect{0} & \vect{0} & \vect{0}
               % & 0 & 0 & \vect{0} \\
      % \vect{0} & \vect{0} & \vect{0} & \vect{0} & \vect{0} & \vect{0} & \vect{0}
               % & 0 & 0 & \vect{0} \\
      % \vect{0} & \left( \hat{R}_{I}^{b} \right)^\transpose
      % \skewmat{ \hat{\vect{v}}_{b/I}^{b} } & 
      % -\left( \hat{R}_{I}^{b} \right)^\transpose & \vect{0} & \vect{0} & \vect{0} & 
      % I_{3 \times 2} \left( \hat{R}_{I}^{g} \right)^\transpose
               % & I_{3 \times 2} \left( \hat{R}_{I}^{g} \right)^\transpose
               % \skewmat{ 1 } \hat{\vect{v}}_{g/I}^g
               % & 0 & \vect{0} \\
      % \vect{0} & \vect{0} & \vect{0} & \vect{0} & \vect{0} & \vect{0} & \vect{0}
               % & 0 & 0 & \vect{0} \\
      % \vect{0} & \vect{0} & \vect{0} & \vect{0} & \vect{0} & \vect{0} & \vect{0}
               % & 0 & 1 & \vect{0} \\
      % \vect{0} & \vect{0} & \vect{0} & \vect{0} & \vect{0} & \vect{0} & \vect{0}
               % & 0 & 0 & \vect{0} \\
      % \vect{0} & \vect{0} & \vect{0} & \vect{0} & \vect{0} & \vect{0} & \vect{0}
               % & 0 & 0 & \vect{0}
    % \end{bmatrix},
    % % \end{smallbmatrix},
% \end{align}
\begin{align}
  F &= \frac{ \partial \dot{\tilde{\x}} }{ \partial \tilde{\x} } \\
    &=
    \begin{bmatrix}
      \vect{0} &
    \cfrac{ \partial \dot{\tilde{\vect{p}}}_{b/I}^{I}}{ \partial \tilde{\vect{\theta}}_I^{b}}
      % - \left( \hat{R}_I^b \right)^\transpose
      % \skewmat{\hat{\vect{v}}_{b/I}^b}
               &
      % \left( \hat{R}_I^b \right)^\transpose
               \cfrac{ \partial \dot{\tilde{\vect{p}}}_{b/I}^{I} }{ \partial \tilde{\vect{v}}_{b/I}^b }
               &
      \vect{0} & \vect{0} & \vect{0} & \vect{0}
               & \vect{0} & \vect{0} & \vect{0} & \dots & \vect{0} \\
      \vect{0} &
      % -\skewmat{ \bar{\vect{\omega}}_{b/I}^b
      % - \hat{\vect{\beta}}_\omega}
      \cfrac{ \partial \dot{\tilde{\vect{\theta}}}_I^{b} }{ \partial \tilde{\vect{\theta}}_I^{b} }
               & \vect{0} & \vect{0} &
      \cfrac{ \partial \dot{\tilde{\vect{\theta}}}_I^{b} }{ \partial \tilde{\vect{\beta}}_{\omega} }
      % -I
               & \vect{0} & \vect{0}
               & \vect{0} & \vect{0} & \vect{0} & \dots & \vect{0} \\
      \vect{0} &
      % \skewmat{ \hat{R}_I^b \vect{g} }
      \cfrac{ \partial \dot{\tilde{\vect{v}}}_{b/I}^b }{ \partial \tilde{\vect{\theta}}_I^{b} }
               &
      % -\skewmat{ \bar{\omega}_{b/I}^b - \hat{\vect{\beta}}_\omega }
      \cfrac{ \partial \dot{\tilde{\vect{v}}}_{b/I}^b }{ \partial \tilde{\vect{v}}_{b/I}^b }
               &
      \cfrac{ \partial \dot{\tilde{\vect{v}}}_{b/I}^b }{ \partial \tilde{\vect{\beta}}_a }
      % -I
               &
      \cfrac{ \partial \dot{\tilde{\vect{v}}}_{b/I}^b }{ \partial \tilde{\vect{\beta}}_{\omega} }
      % -\skewmat{ \hat{\vect{v}}_{b/I}^b }
               & \vect{0} & \vect{0}
               & \vect{0} & \vect{0} & \vect{0} & \dots & \vect{0} \\
      \vect{0} & \vect{0} & \vect{0} & \vect{0} & \vect{0} & \vect{0} & \vect{0}
               & \vect{0} & \vect{0} & \vect{0} & \dots & \vect{0} \\
      \vect{0} & \vect{0} & \vect{0} & \vect{0} & \vect{0} & \vect{0} & \vect{0}
               & \vect{0} & \vect{0} & \vect{0} & \dots & \vect{0} \\
      \vect{0} &
      % \left( \hat{R}_{I}^{b} \right)^\transpose
      % \skewmat{ \hat{\vect{v}}_{b/I}^{b} }
      \cfrac{ \partial \dot{\tilde{\vect{p}}}_{g/b}^{v} }{ \partial \tilde{\vect{\theta}}_I^{b} }
               & 
      % -\left( \hat{R}_{I}^{b} \right)^\transpose
      \cfrac{ \partial \dot{\tilde{\vect{p}}}_{g/b}^{v} }{ \partial \tilde{\vect{v}}_{b/I}^b }
               & \vect{0} & \vect{0} & \vect{0} & 
      % I_{3 \times 2} \left( \hat{R}_{I}^{g} \right)^\transpose
      \cfrac{ \partial \dot{\tilde{\vect{p}}}_{g/b}^{v} }{ \partial \tilde{\vect{v}}_{g/I}^{g} }
               &
               % I_{3 \times 2} \left( \hat{R}_{I}^{g} \right)^\transpose
               % \skewmat{ 1 } \hat{\vect{v}}_{g/I}^g
      \cfrac{ \partial \dot{\tilde{\vect{p}}}_{g/b}^{v} }{ \partial \tilde{\psi}_{I}^{g} }
               & \vect{0} & \vect{0} & \dots & \vect{0} \\
      \vect{0} & \vect{0} & \vect{0} & \vect{0} & \vect{0} & \vect{0} & \vect{0}
               & \vect{0} & \vect{0} & \vect{0} & \dots & \vect{0} \\
      \vect{0} & \vect{0} & \vect{0} & \vect{0} & \vect{0} & \vect{0} & \vect{0}
               & 0 & 
      % 1 
               \cfrac{ \partial \dot{\tilde{\psi}}_{I}^{g} }{ \partial \tilde{\omega}_{g/I}^{g} }
               & \vect{0} & \dots & \vect{0} \\
      \vect{0} & \vect{0} & \vect{0} & \vect{0} & \vect{0} & \vect{0} & \vect{0}
               & 0 & 0 & \vect{0} & \dots & \vect{0} \\
      \vect{0} & \vect{0} & \vect{0} & \vect{0} & \vect{0} & \vect{0} & \vect{0}
               & \vect{0} & \vect{0} & \vect{0} & \dots & \vect{0} \\
      \vdots & \vdots & \vdots & \vdots & \vdots & \vdots & \vdots
             & \vdots & \vdots & \vdots & \ddots & \vdots \\
      \vect{0} & \vect{0} & \vect{0} & \vect{0} & \vect{0} & \vect{0} & \vect{0}
               & \vect{0} & \vect{0} & \vect{0} & \dots & \vect{0}
    \end{bmatrix},
\end{align}
with
\begin{align}
    \cfrac{ \partial \dot{\tilde{\vect{p}}}_{b/I}^{I}}{ \partial \tilde{\vect{\theta}}_I^{b}}
    &=
      - \left( \hat{R}_I^b \right)^\transpose
      \skewmat{\hat{\vect{v}}_{b/I}^b} \\
               \cfrac{ \partial \dot{\tilde{\vect{p}}}_{b/I}^{I} }{ \partial \tilde{\vect{v}}_{b/I}^b }
    &=
      \left( \hat{R}_I^b \right)^\transpose \\
      \cfrac{ \partial \dot{\tilde{\vect{\theta}}}_I^{b} }{ \partial \tilde{\vect{\theta}}_I^{b} }
    &=
      -\skewmat{ \bar{\vect{\omega}}_{b/I}^b
      - \hat{\vect{\beta}}_\omega}
\end{align}
\begin{align}
      \cfrac{ \partial \dot{\tilde{\vect{\theta}}}_I^{b} }{ \partial \tilde{\vect{\beta}}_{\omega} }
    &=
      -I \\
      \cfrac{ \partial \dot{\tilde{\vect{v}}}_{b/I}^b }{ \partial \tilde{\vect{\theta}}_I^{b} }
    &=
      \skewmat{ \hat{R}_I^b \vect{g} } \\
      \cfrac{ \partial \dot{\tilde{\vect{v}}}_{b/I}^b }{ \partial \tilde{\vect{v}}_{b/I}^b }
    &=
      -\skewmat{ \bar{\omega}_{b/I}^b - \hat{\vect{\beta}}_\omega } \\
      \cfrac{ \partial \dot{\tilde{\vect{v}}}_{b/I}^b }{ \partial \tilde{\vect{\beta}}_a }
    &=
      -I \\
      \cfrac{ \partial \dot{\tilde{\vect{v}}}_{b/I}^b }{ \partial \tilde{\vect{\beta}}_{\omega} }
    &=
      -\skewmat{ \hat{\vect{v}}_{b/I}^b } \\
      \cfrac{ \partial \dot{\tilde{\vect{p}}}_{g/b}^{v} }{ \partial \tilde{\vect{\theta}}_I^{b} }
    &=
      \left( \hat{R}_{I}^{b} \right)^\transpose
      \skewmat{ \hat{\vect{v}}_{b/I}^{b} } \\
      \cfrac{ \partial \dot{\tilde{\vect{p}}}_{g/b}^{v} }{ \partial \tilde{\vect{v}}_{b/I}^b }
    &=
      -\left( \hat{R}_{I}^{b} \right)^\transpose \\
      \cfrac{ \partial \dot{\tilde{\vect{p}}}_{g/b}^{v} }{ \partial \tilde{\vect{v}}_{g/I}^{g} }
    &=
      I_{3 \times 2} \left( \hat{R}_{I}^{g} \right)^\transpose \\
      \cfrac{ \partial \dot{\tilde{\vect{p}}}_{g/b}^{v} }{ \partial \tilde{\psi}_{I}^{g} }
    &=
               I_{3 \times 2} \left( \hat{R}_{I}^{g} \right)^\transpose
               \skewmat{ 1 } \hat{\vect{v}}_{g/I}^g \\
               \cfrac{ \partial \dot{\tilde{\psi}}_{I}^{g} }{ \partial \tilde{\omega}_{g/I}^{g} }
    &=
      1 ,
\end{align}
and
\begin{align}
  G &= \frac{ \partial \dot{\tilde{\x}} }{ \partial \vect{\upsilon} } \\
    &=
    \begin{bmatrix}
      \vect{0} & \vect{0} \\
      \vect{0} & -I \\
      -I & -\skewmat{\hat{\vect{v}}_{b/I}^b} \\
      \vect{0} & \vect{0} \\
      \vect{0} & \vect{0} \\
      \vect{0} & \vect{0} \\
      \vect{0} & \vect{0} \\
      \vect{0} & \vect{0} \\
      \vect{0} & \vect{0} \\
      \vect{0} & \vect{0} \\
      \vdots & \vdots \\
      \vect{0} & \vect{0} 
    \end{bmatrix}.
\end{align}
However, to ensure numerical stability, we propagate the covariance using a first-order discrete approximation
defined by
\begin{equation}
  P_{k+1} = F_{k} P_{k} F_{k}^\transpose + G_k Q_u G_k^\transpose + Q_x \Delta
  t^2
\end{equation}
where
\begin{align}
  F_k &\approx I + F \Delta t \\
  G_k &\approx G \Delta t.
\end{align}



% !TEX root=../../master.tex

\subsection{Measurement Models}
\label{sec:measurement_models}

When
updating the filter with measurements, we make use of the partial Kalman update
which has been shown to improve estimates of bias states and constant values~\cite{brink2017partial}. The partial Kalman update provides a
means to limit the effect of measurement updates to certain estimated
states by providing a tuning vector
\begin{equation}
  \vect{\lambda} =
\begin{bmatrix}
  \lambda_1 & \lambda_2 & \dots & \lambda_N
\end{bmatrix}
\end{equation}
where $\lambda_i \in \left[ 0, 1\right]$ determines the proportion of the
measurement update applied to state $i$.
In practice, we use $\lambda < 1$ only for the
bias states, $\vect{\beta}_a$ and $\vect{\beta}_\omega$, and the
constant-value states $\vect{r}_{1/g}^{g}, \dots \vect{r}_{n/g}^{g}$.

With this formulation,
when a measurement is received, we compute the Kalman gain
\begin{equation}
  K = P H^\transpose \left(H P H^\transpose + R \right)^{-1}
\end{equation}
where $H$ is the residual Jacobian for the measurement and $R$ is the
measurement covariance. We then follow~\cite{brink2017partial}, using $K$ to update the filter such that
\begin{align}
  \hat{\tilde{\x}}^{+} &= \vect{\lambda} \odot K \vect{r} \\
  P^{+} &= P + \Lambda \odot \left( \left(I - K H \right) P \left(I - K H
  \right)^\transpose + K R K^\transpose - P \right)
\end{align}
where $\odot$ is the Hadamard product, $\vect{r}$ is the residual of the
measurement and
\begin{align}
  \vect{1} &= \begin{bmatrix} 1 & 1 & \dots & 1 \end{bmatrix}^\transpose \\
  \Lambda &= \vect{1} \vect{\lambda}^\transpose + \vect{\lambda} \vect{1} -
  \vect{\lambda} \vect{\lambda}^\transpose.
\end{align}

As the estimate of the error state of the system, $\hat{\tilde{\x}}$, becomes
non-zero after this update, we use this estimate to correct the estimated state,
$\hat{\x}$. As the estimated state is not a vector, this correction is done
piecewise. The vector components of the estimated state are
updated as
\begin{equation}
  \hat{\vect{\x}}_{\vect{v}}^{+} = \hat{\vect{\x}}_{\vect{v}} + \tilde{\vect{\x}}_{\vect{v}}
\end{equation}
and the quaternion state is updated as
\begin{equation}
  \left( \hat{\q}_I^b \right)^{+}  = \hat{\q}_I^b \otimes \exp_{\q} \left(
  \tilde{\vect{\theta}}_I^b \right).
\end{equation}
After this correction, the estimate of the error state of the system resets to
zero.
% reset to zero.
% The estimate of the error state of the system is then reset to zero before any

The measurement model, residual model and residual
Jacobian are defined below for each type of measurement used in the filter.

\subsubsection{Global UAV Position Measurement}
We assume to receive a measurement of the position of the UAV with respect to
the inertial frame.
In our experiments, this measurement results from a motion capture system;
however, in other applications,
a sensor such as a real-time kinematic GPS unit could provide this measurement.
% a similar measurement can be be used from RTK
% GPS.
% This measurement may come from a sensor such as GPS or a motion capture system.
% The 
% The measurement model and its estimate are
The measurement and its model are
written as
\begin{align}
  \vect{z}_{\text{pos}} &= h_{\text{pos}} \left( \x \right) + \vect{\eta}_{\text{pos}} \\
  h_{\text{pos}} \left( \x \right) &= \vect{p}_{b/I}^I,
  % \vect{\eta}_{\text{pos}},
  % h_{\text{pos}} \left( \hat{\x} \right) &= \hat{\vect{p}}_{b/I}^I,
\end{align}
where $\vect{\eta}_{\text{pos}}$ is a zero-mean Gaussian process
describing the sensor noise.
For a given measurement of position, $\bar{\vect{z}}_{\text{pos}}$, the residual is
given by
\begin{equation}
  \vect{r}_{\text{pos}} = \bar{\vect{z}}_{\text{pos}} - h_{\text{pos}} \left( \hat{\x}
  \right).
\end{equation}
For the error-state Kalman filter, the residual is modeled as
\begin{align}
  % \vect{r}_{\text{pos}} &=  h_{\text{pos}} \left( \x \right) - h_{\text{pos}} \left( \hat{\x}
  % \right) \\
  \vect{r}_{\text{pos}} &=  \vect{z}_{\text{pos}} - h_{\text{pos}} \left( \hat{\x}
  \right) \\
                        &= \vect{p}_{b/I}^I + \eta_{\text{pos}} -
                        \hat{\vect{p}}_{b/I}^I \\
                        &= \tilde{\vect{p}}_{b/I}^I + \eta_{\text{pos}}.
\end{align}
This results in the residual Jacobian
\begin{align}
  H_{\text{pos}} &= \frac{\partial \vect{r}_{\text{pos}}}{\partial \tilde{\x}}\\
                 &=
   \begin{bmatrix}
     \cfrac{\partial \vect{r}_{\text{pos}}}{\partial \tilde{\vect{p}}_{b/I}^{I} } &
     \vect{0} &
     \vect{0} &
     \vect{0} &
     \vect{0} &
     \vect{0} &
     \vect{0} &
     \vect{0} &
     \vect{0} &
     \vect{0} &
     \dots &
     \vect{0}
     % \cfrac{\partial \vect{r}_{\text{pos}}}{\partial \tilde{\vect{\theta}}_I^{b} } &
     % \cfrac{\partial \vect{r}_{\text{pos}}}{\partial \tilde{\vect{v}}_{b/I}^b } &
     % \cfrac{\partial \vect{r}_{\text{pos}}}{\partial \tilde{\vect{\beta}}_a } &
     % \cfrac{\partial \vect{r}_{\text{pos}}}{\partial \tilde{\vect{\beta}}_{\omega} } &
     % \cfrac{\partial \vect{r}_{\text{pos}}}{\partial \tilde{\vect{p}}_{g/b}^{v} } &
     % \cfrac{\partial \vect{r}_{\text{pos}}}{\partial \tilde{\vect{v}}_{g/I}^{g} } &
     % \cfrac{\partial \vect{r}_{\text{pos}}}{\partial \tilde{\psi}_{I}^{g} } &
     % \cfrac{\partial \vect{r}_{\text{pos}}}{\partial \tilde{\omega}_{g/I}^{g} } &
     % \cfrac{\partial \vect{r}_{\text{pos}}}{\partial \tilde{\vect{r}}_{i/g}^{g} }
   \end{bmatrix} \\
                 &=
  \begin{bmatrix}
    I_{3 \times 3} & \vect{0} & \vect{0} & \vect{0} & \vect{0} & \vect{0} &
    \vect{0} & \vect{0} & \vect{0} & \vect{0} & \dots & \vect{0}
  \end{bmatrix}.
\end{align}

% and the non-zero component of the Jacobian of the measurement model as
% \begin{equation}
  % \frac{\partial \hat{\vect{p}}_{b/I}^I}{\partial \vect{p}_{b/I}^I} = I_{3
  % \times 3}.
% \end{equation}

\subsubsection{Global UAV Attitude Measurement}
Similar to the position measurement above, we assume to receive a measurement of
the attitude of the body frame of the UAV with respect to the inertial frame.
In our experiments, this measurement results from a motion capture system;
however, in other applications, a sensor such as an attitude and heading
reference system could provide this measurement.
% This measurement may
% come from an attitude and heading reference system or from a motion
% capture system. 
% The measurement model and its estimate are
The measurement and its model are 
written as
\begin{align}
  \vect{z}_{\text{att}} &= h_{\text{att}} \left( \x \right) \otimes \exp_{\q} \left(
  \vect{\eta}_{\text{att}} \right) \\
  h_{\text{att}} \left( \x \right) &= \vect{q}_{I}^b, 
  % \vect{\eta}_{\text{att}} \right) \\
    % h_{\text{att}} \left( \hat{\x} \right) &= \hat{\vect{q}}_{I}^b,
\end{align}
where $\vect{\eta}_{\text{att}}$ is a zero-mean Gaussian process
describing the sensor noise.
For a given measurement of attitude, $\bar{\vect{z}}_{\text{att}}$, the residual is
given by
\begin{equation}
  \vect{r}_{\text{att}} = \log_{\q} \left(  h_{\text{att}} \left(
  \hat{\x} \right)^{-1} \otimes \bar{\vect{z}}_{\text{att}} \right),
  % \vect{r}_{\text{att}} = \vect{z}_{\text{att}} - h_{\text{att}} \left( \hat{\x}
\end{equation}
which is modeled as
\begin{align}
  \vect{r}_{\text{att}} &= \log_{\q} \left(  h_{\text{att}} \left(
  \hat{\x} \right)^{-1} \otimes \vect{z}_{\text{att}} \right) \\
  % \hat{\x} \right)^{-1} \otimes h_{\text{att}} \left( \x \right) \right) \\
                        &= \log_{\q} \left(  \left(
  \hat{\vect{q}}_{I}^b \right)^{-1} \otimes \vect{q}_{I}^b \otimes \exp_{\q} \left(
  \vect{\eta}_{\text{att}} \right)\right).
\end{align}
This is expanded using~\eqref{eq:quat_true_state} and simplified to yield
\begin{align}
  \vect{r}_{\text{att}}
  % &= \log_{\q} \left(  h_{\text{att}} \left(
  % \hat{\x} \right)^{-1} \otimes h_{\text{att}} \left( \x \right) \right) \\
                        % &= \log_{\q} \left(  \left(
  % \hat{\vect{q}}_{I}^b \right)^{-1} \otimes \vect{q}_{I}^b \otimes \exp_{\q} \left(
  % \vect{\eta}_{\text{att}} \right)\right) \\
                        &= \log_{\q} \left(  \left(
                        \hat{\vect{q}}_{I}^b \right)^{-1} \otimes
                        \hat{\vect{q}}_{I}^b \otimes \exp_{\q} \left(
                      \tilde{\vect{\theta}}_I^b \right) \right)
                          + \vect{\eta}_{\text{att}}  \\
                        &= \tilde{\vect{\theta}}_I^b
                          + \vect{\eta}_{\text{att}}. 
\end{align}
This results in the residual Jacobian
\begin{align}
  H_{\text{att}} &= \frac{\partial \vect{r}_{\text{att}}}{\partial \tilde{\x}}\\
                 &=
   \begin{bmatrix}
     \vect{0} &
     \cfrac{\partial \vect{r}_{\text{att}}}{\partial \tilde{\vect{\theta}}_I^{b} } &
     \vect{0} &
     \vect{0} &
     \vect{0} &
     \vect{0} &
     \vect{0} &
     \vect{0} &
     \vect{0} &
     \vect{0} &
     \dots &
     \vect{0}
     % \cfrac{\partial \vect{r}_{\text{pos}}}{\partial \tilde{\vect{\theta}}_I^{b} } &
     % \cfrac{\partial \vect{r}_{\text{pos}}}{\partial \tilde{\vect{v}}_{b/I}^b } &
     % \cfrac{\partial \vect{r}_{\text{pos}}}{\partial \tilde{\vect{\beta}}_a } &
     % \cfrac{\partial \vect{r}_{\text{pos}}}{\partial \tilde{\vect{\beta}}_{\omega} } &
     % \cfrac{\partial \vect{r}_{\text{pos}}}{\partial \tilde{\vect{p}}_{g/b}^{v} } &
     % \cfrac{\partial \vect{r}_{\text{pos}}}{\partial \tilde{\vect{v}}_{g/I}^{g} } &
     % \cfrac{\partial \vect{r}_{\text{pos}}}{\partial \tilde{\psi}_{I}^{g} } &
     % \cfrac{\partial \vect{r}_{\text{pos}}}{\partial \tilde{\omega}_{g/I}^{g} } &
     % \cfrac{\partial \vect{r}_{\text{pos}}}{\partial \tilde{\vect{r}}_{i/g}^{g} }
   \end{bmatrix} \\
                 &=
  \begin{bmatrix}
    \vect{0} & I_{3 \times 3} &  \vect{0} & \vect{0} & \vect{0} & \vect{0} &
    \vect{0} & \vect{0} & \vect{0} & \vect{0} & \dots & \vect{0}
  \end{bmatrix}.
\end{align}
% \begin{equation}
  % H_{\text{att}} =
  % \begin{bmatrix}
    % \vect{0} & I_{3 \times 3} & \vect{0} & \vect{0} & \vect{0} & \vect{0} & \vect{0} & \vect{0} & \vect{0} & \vect{0}
  % \end{bmatrix}.
% \end{equation}

\subsubsection{Fiducial Translation Measurement}
We assume that a known fiducial marker serves as the desired landing position
for the multirotor UAV on the target vehicle. The goal frame is, therefore,
located at the center of the fiducial marker. In consequence, every detection of the fiducial
marker yields a measurement of the relative translation and rotation from the
camera frame to the goal frame.
% as noted in~\eqref{eq:fiducial_meas}.
% The measurement model and its estimate for this relative translation measurement
The measurement and its model for this relative translation measurement
are written as
\begin{align}
  \vect{z}_{\text{ft}} &=
  h_{\text{ft}} \left( \x \right) + \vect{\eta}_{\text{ft}} \\
  h_{\text{ft}} \left( \x \right) &=
  \vect{p}_{g/c}^c \\
  &= R_b^c \left( R_I^b \vect{p}_{g/b}^v -
  \vect{p}_{c/b}^b \right),
  % h_{\text{ft}} \left( \hat{\x} \right) &=
    % \hat{\vect{p}}_{g/c}^c \\
  % &= R_b^c \left( \hat{R}_I^b \hat{\vect{p}}_{g/b}^v -
    % \vect{p}_{c/b}^b \right),
  % \hat{\vect{q}}_{c}^a  &= R_g^a \hat{R}_I^g \hat{R}_b^I R_c^b
\end{align}
where $R_b^c$ and $\vect{p}_{c/b}^b$ are assumed to be known constants, and
$\vect{\eta}_{\text{ft}}$ is a zero-mean Gaussian process describing the
measurement noise.
For a given measurement of the relative translation to the fiducial marker,
$\bar{\vect{z}}_{\text{ft}}$, the residual is given by
\begin{equation}
  \vect{r}_{\text{ft}} = \bar{\vect{z}}_{\text{ft}} - h_{\text{ft}} \left( \hat{\x}
  \right)
\end{equation}
and modeled as
\begin{align}
  \vect{r}_{\text{ft}} &=  \vect{z}_{\text{ft}} - h_{\text{ft}} \left( \hat{\x}
  \right) \\
                       &= R_b^c \left( R_I^b \vect{p}_{g/b}^v -
                         \vect{p}_{c/b}^b \right)  +\eta_{\text{ft}} - R_b^c \left( \hat{R}_I^b \hat{\vect{p}}_{g/b}^v -
    \vect{p}_{c/b}^b \right)  \\
                       &= R_b^c R_I^b \vect{p}_{g/b}^v 
                          - R_b^c \hat{R}_I^b \hat{\vect{p}}_{g/b}^v +
                          \eta_{\text{ft}}.
                          \label{eq:rft_1}
\end{align}
% We note that from~\eqref{eq:quat_true_state},
% \begin{equation*}
  % R_I^b = R \left( \exp_{\q} \left( \tilde{\vect{\theta}}_I^b \right) \right)
  % \hat{R}_I^b,
% \end{equation*}
% which can be simplifed using~\eqref{eq:quaternion_exp_approx}
% and~\eqref{eq:R_from_q} to show that
% \begin{equation}
  % R_I^b \approx \left( I - \skewmat{\tilde{\vect{\theta}}_I^b }\right) \hat{R}_I^b.
  % \label{eq:expand_R}
% \end{equation}
% We can similarly show that 
% \begin{equation}
  % \left( R_I^b \right)^\transpose \approx \left( \hat{R}_I^b \right)^\transpose
  % \left( I + \skewmat{\tilde{\vect{\theta}}_I^b }\right).
% \end{equation}
Using~\eqref{eq:est_paper_Rapprox}, we expand~\eqref{eq:rft_1} and ignore
second-order terms to yield
\begin{align}
   \vect{r}_{\text{ft}} &\approx R_b^c \left( I - \skewmat{\tilde{\vect{\theta}}_I^b }\right) \hat{R}_I^b \left( \hat{\vect{p}}_{g/b}^v + \tilde{\vect{p}}_{g/b}^v \right) 
                          - R_b^c \hat{R}_I^b \hat{\vect{p}}_{g/b}^v +
                          \eta_{\text{ft}} \\
  &\approx R_b^c \hat{R}_I^b \tilde{\vect{p}}_{g/b}^v -
  R_b^c \skewmat{\tilde{\vect{\theta}}_I^b } \hat{R}_I^b \hat{\vect{p}}_{g/b}^v 
      + \eta_{\text{ft}} \\
&\approx R_b^c \hat{R}_I^b \tilde{\vect{p}}_{g/b}^v + R_b^c \skewmat{\hat{R}_I^b \hat{\vect{p}}_{g/b}^v} \tilde{\vect{\theta}}_I^b 
      + \eta_{\text{ft}}.
\end{align}
This results in the residual Jacobian
\begin{align}
  H_{\text{ft}} &= \frac{\partial \vect{r}_{\text{ft}}}{\partial \tilde{\x}}\\
                 &=
   \begin{bmatrix}
     \vect{0} &
     \cfrac{\partial \vect{r}_{\text{ft}}}{\partial \tilde{\vect{\theta}}_I^{b} } &
     \vect{0} &
     \vect{0} &
     \vect{0} &
     \cfrac{\partial \vect{r}_{\text{ft}}}{\partial \tilde{\vect{p}}_{g/b}^{v} } &
     \vect{0} &
     \vect{0} &
     \vect{0} &
     \vect{0} &
     \dots &
     \vect{0}
     % \cfrac{\partial \vect{r}_{\text{pos}}}{\partial \tilde{\vect{\theta}}_I^{b} } &
     % \cfrac{\partial \vect{r}_{\text{pos}}}{\partial \tilde{\vect{v}}_{b/I}^b } &
     % \cfrac{\partial \vect{r}_{\text{pos}}}{\partial \tilde{\vect{\beta}}_a } &
     % \cfrac{\partial \vect{r}_{\text{pos}}}{\partial \tilde{\vect{\beta}}_{\omega} } &
     % \cfrac{\partial \vect{r}_{\text{pos}}}{\partial \tilde{\vect{p}}_{g/b}^{v} } &
     % \cfrac{\partial \vect{r}_{\text{pos}}}{\partial \tilde{\vect{v}}_{g/I}^{g} } &
     % \cfrac{\partial \vect{r}_{\text{pos}}}{\partial \tilde{\psi}_{I}^{g} } &
     % \cfrac{\partial \vect{r}_{\text{pos}}}{\partial \tilde{\omega}_{g/I}^{g} } &
     % \cfrac{\partial \vect{r}_{\text{pos}}}{\partial \tilde{\vect{r}}_{i/g}^{g} }
   \end{bmatrix} \\
                 &=
  \begin{bmatrix}
    \vect{0} & R_b^c \skewmat{ \hat{R}_I^b \hat{\vect{p}}_{g/b}^v } &  \vect{0}
             & \vect{0} & \vect{0} & R_b^c \hat{R}_I^b &
    \vect{0} & \vect{0} & \vect{0} & \vect{0} & \dots & \vect{0}
  \end{bmatrix}.
\end{align}
% \begin{equation}
  % H_{\text{ft}} =
  % \begin{bmatrix}
    % \vect{0} & R_b^c \skewmat{ \hat{R}_I^b \hat{\vect{p}}_{g/b}^v } & \vect{0} &
    % \vect{0} & \vect{0} & R_b^c \hat{R}_I^b & \vect{0} & \vect{0} & \vect{0} & \vect{0}
  % \end{bmatrix}.
% \end{equation}


% The non-zero components of the Jacobian of the measurement model used for the
% Kalman filter update are

% \begin{align}
  % \frac{\partial \hat{\vect{p}}_{g/c}^c}{\partial \vect{q}_I^b} =& R_b^c 
  % \hat{R}_I^b \skewmat{\hat{\vect{p}}_{g/b}^v} \\
    % \frac{\partial \hat{\vect{p}}_{g/c}^c}{\partial \vect{p}_{g/b}^v} =& R_b^c
    % \hat{R}_I^b.
% \end{align}

\subsubsection{Fiducial Rotation Measurement}
We use the relative rotation measurement that results from a detection of the
fiducial marker
to create a pseudo measurement of the orientation of the goal frame. In theory,
a measurement model could be developed to use the entire measurement,
$\bar{R}_c^g$; however, in practice, this may cause complications if the
fiducial landing marker is not perfectly aligned with the plane of the target
vehicle's motion (i.e., the fiducial marker is slighly rolled or pitched with
respect to the goal frame).
The pseudo
measurement is created with
\begin{equation}
  \bar{\psi}_I^g = \text{yaw} \left( \bar{R}_c^g R_b^c \hat{R}_I^b \right)
\end{equation}
where $\text{yaw}\left(\right)$ is a function that extracts the yaw angle
from a rotation matrix
given by
\begin{equation}
  \text{yaw} \left(
    \begin{bmatrix}
      r_{11} & r_{12} & r_{13} \\
      r_{21} & r_{22} & r_{23} \\
      r_{31} & r_{32} & r_{33} 
    \end{bmatrix}
  \right)
  =
  \mathrm{atan2}\left( r_{12}, r_{11} \right).
\end{equation}
% The measurement model and its estimate are
The measurement and its model are
written as
\begin{align}
  \vect{z}_{\text{fr}} &= h_{\text{fr}} \left( \x \right) + \eta_{\text{fr}} \\
  h_{\text{fr}} \left( \x \right) &= \psi_I^g,
  % h_{\text{fr}} \left( \hat{\x} \right) &= \hat{\psi}_I^g
\end{align}
where $\eta_{\text{fr}}$ is a zero-mean Gaussian process describing the
measurement noise.
For a given (pseudo) measurement of the rotation of the fiducial marker,
$\bar{\psi}_I^g$,
% $z_{\text{fr}}$,
the residual is given by
\begin{equation}
  r_{\text{fr}} = \bar{\psi}_I^g - h_{\text{fr}} \left( \hat{\x} \right),
  % r_{\text{fr}} = z_{\text{fr}} - h_{\text{fr}} \left( \hat{\x} \right),
\end{equation}
which is modeled as
\begin{align}
  % r_{\text{fr}} &= h_{\text{fr}} \left( \x \right) - h_{\text{fr}} \left( \hat{\x} \right) \\
  r_{\text{fr}} &= \vect{z}_{\text{fr}} - h_{\text{fr}} \left( \hat{\x} \right) \\
                &= \psi_I^g + \eta_{\text{fr}} - \hat{\psi}_I^g \\
                &= \tilde{\psi}_I^g + \eta_{\text{fr}}.
\end{align}
This results in the residual Jacobian
\begin{align}
  H_{\text{fr}} &= \frac{\partial \vect{r}_{\text{fr}}}{\partial \tilde{\x}}\\
                 &=
   \begin{bmatrix}
     \vect{0} &
     \vect{0} &
     \vect{0} &
     \vect{0} &
     \vect{0} &
     \vect{0} &
     \vect{0} &
     \cfrac{\partial \vect{r}_{\text{fr}}}{\partial \tilde{\psi}_I^g } &
     0 &
     \vect{0} &
     \dots &
     \vect{0}
     % \cfrac{\partial \vect{r}_{\text{pos}}}{\partial \tilde{\vect{\theta}}_I^{b} } &
     % \cfrac{\partial \vect{r}_{\text{pos}}}{\partial \tilde{\vect{v}}_{b/I}^b } &
     % \cfrac{\partial \vect{r}_{\text{pos}}}{\partial \tilde{\vect{\beta}}_a } &
     % \cfrac{\partial \vect{r}_{\text{pos}}}{\partial \tilde{\vect{\beta}}_{\omega} } &
     % \cfrac{\partial \vect{r}_{\text{pos}}}{\partial \tilde{\vect{p}}_{g/b}^{v} } &
     % \cfrac{\partial \vect{r}_{\text{pos}}}{\partial \tilde{\vect{v}}_{g/I}^{g} } &
     % \cfrac{\partial \vect{r}_{\text{pos}}}{\partial \tilde{\psi}_{I}^{g} } &
     % \cfrac{\partial \vect{r}_{\text{pos}}}{\partial \tilde{\omega}_{g/I}^{g} } &
     % \cfrac{\partial \vect{r}_{\text{pos}}}{\partial \tilde{\vect{r}}_{i/g}^{g} }
   \end{bmatrix} \\
                 &=
  \begin{bmatrix}
    \vect{0} & \vect{0} &  \vect{0}
             & \vect{0} & \vect{0} & \vect{0} &
    \vect{0} &  1 & 0 & \vect{0} & \dots & \vect{0}
  \end{bmatrix}.
\end{align}
% resulting in the residual Jacobian
% \begin{equation}
  % H_{\text{fr}} =
  % \begin{bmatrix}
    % \vect{0} & \vect{0} & \vect{0} &
    % \vect{0} & \vect{0} & \vect{0} & \vect{0} & 1 & \vect{0} & \vect{0}
  % \end{bmatrix}.
% \end{equation}

% and the non-zero Jacobians of the measurement model as
% \begin{equation}
  % \frac{\partial \hat{\theta}_I^g}{\partial \theta_I^g} = 1.
% \end{equation}

\subsubsection{Visual Feature Pixel Measurement}
The estimator receives measurements of
the location of each tracked visual feature in the camera image. We assume that the
pixel locations received have already been corrected for lens distortion.
% and
We also assume to know the camera intrinsic matrix,
% that we know the camera intrinsic matrix,
% We also assume to know the camera intrinsic matrix
\begin{equation}
  K =
  \begin{bmatrix}
    f_x & 0 & c_x \\
    0 & f_y & c_y \\
    0 & 0 & 1
  \end{bmatrix},
\end{equation}
where $f_x$ and $f_y$ are the focal lengths of the camera and $c_x$ and $c_y$
are the coordinates of the principal point in the camera image.
Using the pinhole camera model, we express the pixel location of
visual feature $i$ as
\begin{align}
  % \hat{h} &=
  % \begin{bmatrix}
    % \hat{p}_x & \hat{p}_y
  % \end{bmatrix}^\transpose \\
  \begin{bmatrix}
    p_x \\ p_y \\ 1
  \end{bmatrix} &= \frac{1}{\e_3^\transpose \vect{p}_{i/c}^c} K
  \vect{p}_{i/c}^c
  \label{eq:pinhole_camera}
  % \vect{z} &=
  % \begin{bmatrix}
    % f_x \frac{\e_1 \vect{p}_{i/c}^c}{\e_3 \vect{p}_{i/c}^c} + c_x \\
    % f_y \frac{\e_2 \vect{p}_{i/c}^c}{\e_3 \vect{p}_{i/c}^c} + c_y
  % \end{bmatrix},
\end{align}
% where K is the camera intrisic matrix given by
% \begin{equation}
  % K =
  % \begin{bmatrix}
    % f_x & 0 & c_x \\
    % 0 & f_y & c_y \\
    % 0 & 0 & 1
  % \end{bmatrix}
% \end{equation}
% with $f_x$ and $f_y$ 
% and 
where
\begin{align}
  \vect{p}_{i/c}^c = R_b^c \left( R_I^b \left( R_I^g \right)^\transpose
  \vect{r}_{i/g}^g + R_I^b \vect{p}_{g/b}^v - \vect{p}_{c/b}^b \right).
  \label{eq:p_i_c_c}
\end{align}
% The measurement model and its estimate are, therefore, written as
The measurement and its model are, therefore, written as
\begin{align}
  \vect{z}_{\text{pix}} 
  &= h_{\text{pix}} \left( \x \right) + \vect{\eta}_{\text{pix}} \\
  h_{\text{pix}} \left( \x \right)
  &= \begin{bmatrix} p_x & p_y \end{bmatrix}^\transpose \\
  &= \frac{1}{\e_3^\transpose \vect{p}_{i/c}^c} I_{2 \times 3} K
  \vect{p}_{i/c}^c,
  % h_{\text{pix}} \left( \hat{\x} \right)
  % &= \begin{bmatrix} \hat{p}_x & \hat{p}_y \end{bmatrix}^\transpose \\
  % &= \frac{1}{\e_3^\transpose \hat{\vect{p}}_{i/c}^c} I_{2 \times 3} K
  % \hat{\vect{p}}_{i/c}^c
\end{align}
where $\vect{\eta}_{\text{pix}}$ is a zero-mean Gaussian process describing the
measurement noise.
% and
% \begin{align}
  % \hat{\vect{p}}_{i/c}^c = R_b^c \left( \hat{R}_I^b \left( \hat{R}_I^g \right)^\transpose
  % \hat{\vect{r}}_{i/g}^g + \hat{R}_I^b \hat{\vect{p}}_{g/b}^v - \vect{p}_{c/b}^b
  % \label{eq:p_i_c_c_hat}
% \right).
% \end{align}
For a given measurement of the pixel location of a feature,
$\bar{\vect{z}}_{\text{pix}}$, the residual is given by
\begin{equation}
  \vect{r}_{\text{pix}} = \bar{\vect{z}}_{\text{pix}} - h_{\text{pix}} \left( \hat{\x}
    \right),
\end{equation}
which is modeled as
\begin{align}
  \vect{r}_{\text{pix}} &= \vect{z}_{\text{pix}} - h_{\text{pix}} \left( \hat{\x}
    \right) \\
  &= \frac{1}{\e_3^\transpose \vect{p}_{i/c}^c} I_{2 \times 3} K
  \vect{p}_{i/c}^c + \vect{\eta}_{\text{pix}} - \frac{1}{\e_3^\transpose \hat{\vect{p}}_{i/c}^c} I_{2 \times 3} K
  \hat{\vect{p}}_{i/c}^c.
  \label{eq:rpix}
\end{align}
This results in the residual Jacobian
\begin{align}
  H_{\text{pix}} &= \frac{ \partial \vect{r}_{\text{pix}} }{ \partial \tilde{\x}} \\
  % H_{\text{pix}} 
                   % &= \frac{ \partial} \\
  &\approx
  \frac{1}{\e_3^\transpose \hat{\vect{p}}_{i/c}^c} I_{2 \times 3} K
 \frac{\partial}{\partial \tilde{\x}} \vect{p}_{i/c}^c 
 - \frac{\e_3^\transpose \frac{\partial}{\partial \tilde{\x}} 
 \vect{p}_{i/c}^c}{\left( \e_3^\transpose
 \hat{\vect{p}}_{i/c}^c \right)^2 } I_{2 \times 3} K
 \hat{\vect{p}}_{i/c}^c
\end{align}
where the non-zero components of $\frac{\partial}{\partial \tilde{\x}}
\vect{p}_{i/c}^c$ are given by
\begin{align}
  \frac{\partial}{\partial \tilde{\vect{\theta}}_I^b } \vect{p}_{i/c}^c
  &=
  R_b^c \skewmat{ \hat{R}_I^b \left( \left( \hat{R}_I^g \right)^\transpose
  \hat{\vect{r}}_{i/g}^g + \hat{\vect{p}}_{g/b}^v \right) } \\
  \frac{\partial}{\partial \tilde{\vect{p}}_{g/b}^v} \vect{p}_{i/c}^c
  &=
  R_b^c \hat{R}_I^b \\
  \frac{\partial}{\partial \tilde{\psi}_I^g} \vect{p}_{i/c}^c
  &=
  R_b^c \hat{R}_I^b \left( \hat{R}_I^g \right)^\transpose
  \skewmat{\begin{matrix} 0 \\ 0 \\ 1 \end{matrix}} \hat{\vect{r}}_{i/g}^g \\
  \frac{\partial}{\partial \tilde{\vect{r}}_{i/g}^g} \vect{p}_{i/c}^c
  &=
  R_b^c \hat{R}_I^b \left( \hat{R}_I^g \right)^\transpose .
\end{align}
The derivation of this residual Jacobian is found in
Appendix~\ref{apdx:estimation_pixel_meas_model}.



% where $f_x$, $f_y$, $c_x$, and $c_y$ are constant parameters of the camera. We
% can then express the position of the landmark $i$ with respect to the camera in the
% camera frame as
% \begin{align}
  % \vect{p}_{i/c}^c &= R_b^c R_v^b \left(\vect{p}_{i/v}^v -
    % \vect{p}_{c/v}^v\right) \\
    % \vect{p}_{i/c}^c &= R_b^c \left(R_v^b \vect{p}_{i/v}^v -
    % \vect{p}_{c/b}^b\right)
% \end{align}
% This is actually a little bit weird because of the state parameters, but
% \begin{equation}
  % \vect{p}_{i/v}^v =
  % \begin{bmatrix}
    % \e_1^\transpose \left( R_g^v \vect{p}_{i/g}^g + \vect{p}_{g/v}^v \right) \\
    % \e_2^\transpose \left( R_g^v \vect{p}_{i/g}^g + \vect{p}_{g/v}^v \right) \\
    % \e_3^\transpose \left( \vect{p}_{i/g}^g - \vect{p}_{b/I}^I \right)
  % \end{bmatrix}.
% \end{equation}

% Let us derive the measurement Jacobians.
% \begin{align}
  % \frac{\partial}{\partial \x} \hat{h} &= \frac{\e_3^\transpose \hat{\vect{p}}_{i/c}^c
  % \frac{\partial}{\partial \x} K \hat{\vect{p}}_{i/c}^c - K \hat{\vect{p}}_{i/c}^c
  % \frac{\partial}{\partial \x} \e_3^\transpose \hat{\vect{p}}_{i/c}^c}{\left(
% \e_3^\transpose \hat{\vect{p}}_{i/c}^c \right)^2 } \\
  % \frac{\partial}{\partial \x} \hat{h} &= \frac{\e_3^\transpose \hat{\vect{p}}_{i/c}^c
  % K \frac{\partial}{\partial \x} \hat{\vect{p}}_{i/c}^c - K \hat{\vect{p}}_{i/c}^c
  % \e_3^\transpose \frac{\partial}{\partial \x} \hat{\vect{p}}_{i/c}^c}{\left(
% \e_3^\transpose \hat{\vect{p}}_{i/c}^c \right)^2 }
% \end{align}
% \begin{align}
  % \frac{\partial}{\partial \vect{p}_{g/b}^v} \hat{\vect{p}}_{i/c}^c &=
  % R_b^c \hat{R}_I^b \\
  % \frac{\partial}{\partial \vect{q}_{I}^b} \hat{\vect{p}}_{i/c}^c &=
  % -R_b^c \hat{R}_I^b \skewmat{ \left( R_I^g \right)^\transpose
  % \hat{\vect{r}}_{i/g}^g + \hat{\vect{p}}_{g/b}^v } \\
  % \frac{\partial}{\partial \theta_I^g} \hat{\vect{p}}_{i/c}^c &=
  % R_b^c \hat{R}_I^b \\
% \end{align}

% % \subsubsection{Measurement Jacobian}
% For the purpose of deriving the Jacobians, we will just look at the jacobians
% with respect to $p_x$, the pixel location along the $x$ axis of the camera
% frame. From above we have that
% \begin{align}
  % p_x &= f_x \frac{\e_1^\transpose \vect{p}_{i/c}^c}{\e_3^\transpose \vect{p}_{i/c}^c} + c_x \\ 
  % p_x &= f_x \frac{\e_1^\transpose R_b^c \left(R_v^b \vect{p}_{i/v}^v -
      % \vect{p}_{c/b}^b\right) }{\e_3^\transpose R_b^c \left(R_v^b \vect{p}_{i/v}^v -
  % \vect{p}_{c/b}^b\right) } + c_x 
% \end{align}
% Note that all values are constants except for $\vect{p}_{i/v}^v$ and $R_v^b$.
% The individual parts of the jacobian are given here:
% \begin{equation}
  % \frac{\partial p_x}{\partial \vect{p}_{g/v}^v} =
  % \frac{f_x \e_1 R_b^c R_v^b}
    % {\left(\e_3 R_b^c \left(R_v^b \vect{p}_{i/v}^v -
    % \vect{p}_{c/b}^b\right)\right)}
    % - \frac{\left(\e_3 R_b^c R_v^b \right) f_x \left(\e_1 R_b^c \left(R_v^b \vect{p}_{i/v}^v -
        % \vect{p}_{c/b}^b\right)\right)} {\left(\e_3 R_b^c \left(R_v^b \vect{p}_{i/v}^v -
  % \vect{p}_{c/b}^b\right)\right)^2}
% \end{equation}
% where only the first two (of three) entries of the vector are used. The third
% entry is used for the jacobian w.r.t. $\vect{p}_{b/I}^I(2)$ as
% \begin{equation}
  % \frac{\partial p_x}{\partial \vect{p}_{b/I}^I(2)} = -\frac{\partial p_x}{\partial
  % \vect{p}_{g/v}^v}\left(2\right).
% \end{equation}
% The derivatives for the attitude representation are less straight forward. From
% "Micro Lie Theory" we use the fact that the jacobian of the rotation action can
% be shown to be
% \begin{equation}
  % J_R^{R \cdot v} = -R \skewmat{v}.
% \end{equation}
% \begin{equation}
  % \frac{\partial p_x}{\partial \q} =
  % \frac{-f_x \e_1 R_b^c R_v^b \skewmat{\vect{p}_{i/v}^v}}
    % {\left(\e_3 R_b^c \left(R_v^b \vect{p}_{i/v}^v -
    % \vect{p}_{c/b}^b\right)\right)}
    % - \frac{\left(-\e_3 R_b^c R_v^b \skewmat{\vect{p}_{i/v}^v} \right) f_x \left(\e_1 R_b^c \left(R_v^b \vect{p}_{i/v}^v -
        % \vect{p}_{c/b}^b\right)\right)} {\left(\e_3 R_b^c \left(R_v^b \vect{p}_{i/v}^v -
  % \vect{p}_{c/b}^b\right)\right)^2}
% \end{equation}
% where the jacobians for the other angles are similar, just substituting in.
% The derivative for the goal angle, $\theta_g$ is given by using
% \begin{equation}
  % \frac{\partial \vect{p}_{i/v}^v}{\partial \theta_g} =
  % \begin{bmatrix}
    % \e_1^\transpose \frac{\partial R_g^v}{\partial \theta_g} \vect{p}_{i/g}^g \\
    % \e_2^\transpose \frac{\partial R_g^v}{\partial \theta_g} \vect{p}_{i/g}^g \\
    % 0
  % \end{bmatrix}
% \end{equation}
% in the jacobian given by
% \begin{equation}
  % \frac{\partial p_x}{\partial \theta_g} =
  % \frac{f_x \e_1 R_b^c R_v^b \frac{\partial \vect{p}_{i/v}^v}{\partial \theta_g}}
    % {\left(\e_3 R_b^c \left(R_v^b \vect{p}_{i/v}^v -
    % \vect{p}_{c/b}^b\right)\right)}
    % - \frac{\left(\e_3 R_b^c R_v^b \frac{\partial \vect{p}_{i/v}^v}{\partial \theta_g} \right) f_x \left(\e_1 R_b^c \left(R_v^b \vect{p}_{i/v}^v -
        % \vect{p}_{c/b}^b\right)\right)} {\left(\e_3 R_b^c \left(R_v^b \vect{p}_{i/v}^v -
  % \vect{p}_{c/b}^b\right)\right)^2}
% \end{equation}
% Similarly, the derivative for the landmark offset, $\vect{r}_i$ or
% $\vect{p}_{i/g}^g$ is given by using
% \begin{equation}
  % \frac{\partial \vect{p}_{i/v}^v}{\partial \vect{p}_{i/g}^g} =
  % \begin{bmatrix}
    % \e_1^\transpose R_g^v \\
    % \e_2^\transpose R_g^v \\
    % \e_3^\transpose
  % \end{bmatrix}
% \end{equation}
% in the jacobian given by
% \begin{equation}
  % \frac{\partial p_x}{\partial \vect{p}_{i/g}^g} =
  % \frac{f_x \e_1 R_b^c R_v^b \frac{\partial \vect{p}_{i/v}^v}{\partial \vect{p}_{i/g}^g}}
    % {\left(\e_3 R_b^c \left(R_v^b \vect{p}_{i/v}^v -
    % \vect{p}_{c/b}^b\right)\right)}
    % - \frac{\left(\e_3 R_b^c R_v^b \frac{\partial \vect{p}_{i/v}^v}{\partial \vect{p}_{i/g}^g} \right) f_x \left(\e_1 R_b^c \left(R_v^b \vect{p}_{i/v}^v -
        % \vect{p}_{c/b}^b\right)\right)} {\left(\e_3 R_b^c \left(R_v^b \vect{p}_{i/v}^v -
  % \vect{p}_{c/b}^b\right)\right)^2}
% \end{equation}
% The jacobians for the y pixel measurement, $p_y$ are similar to the ones derived
% above, just changing $f_y$ in place of $f_x$ and $\e_2$ instead of $\e_1$.


% !TEX root=./root.tex

\subsection{State Initialization}
% Unlike $\hat{\vect{x}}_{\text{UAV}}$ which is initialized based on an initial
% guess of the true state, the estimated states associated with the landing
% vehicle, $\hat{\vect{x}}_{\text{Goal}}$, and the estimated states associated
% with the visual features, $\hat{\vect{x}}_{\text{Features}}$, are initialized
% upon receiving the first measurements corresponding measurements.

While the states associated with the UAV and sensor biases,
$\hat{\x}_{\text{UAV}}$, are initialized at startup,
the states associated with the target vehicle, $\hat{\x}_{\text{Goal}}$ are
initialized upon receiving the first measurement from
the detection of the fiducial marker given by
% We assume that we can measure the
% relative translation and rotation from the camera to the fiducial marker
% at each detection. This measurement,
\begin{equation}
  \bar{\vect{z}} =
  \begin{bmatrix}
    \bar{\vect{p}}_{g/c}^c & \bar{R}_c^g
  \end{bmatrix}.
  \label{eq:fiducial_meas}
\end{equation}
We use this measurement to initialize the estimates of the states with
% is used to initialize the goal states with
\begin{align}
  \hat{\vect{p}}_{g/b}^v &= \left( \hat{R}_I^b \right)^\transpose \left( \left ( R_b^c
  \right)^\transpose \bar{\vect{p}}_{g/c}^c + \vect{p}_{c/b}^b
\right)   \\
      \hat{\vect{v}}_{g/I}^I &= \vect{0} \\
      \hat{\psi_I^g} &= \text{yaw} \left( \bar{R}_c^g R_b^c \hat{R}_I^b \right)
      \\
      \hat{\omega_{g/I}^g} &= 0.
  % quat::Quatd q_I2g_meas = x().q * q_b2c_ * z.q_c2a * q_a2g;
  % const double yaw_meas = q_I2g_meas.euler()(2);
\end{align}
% where $R_b^c$ and $\vect{p}_{c/b}^b$ are assumed to be known constants and
% $yaw()$ is a function that extracts the yaw Euler angle from a rotation matrix.

% The linear and angular velocity of the landing vehicle are initialized to
% zero with large enough covariances for the specific use case.

Similar to the manner in which the target vehicle states are intialized,
the visual feature states are initialized based on the first corresponding measurement received.
% each time a new
% landmark is added to the estimated vector, we initialize its state based on the
% first measurement received.
As we only receive a measurement of the pixel
location of the visual feature in the camera image,
\begin{equation}
  \bar{\vect{z}} = \begin{bmatrix} \bar{p}_x & \bar{p}_y \end{bmatrix}
  \label{eq:pixel_meas},
\end{equation}
the estimated state, $\hat{\vect{r}}_{i/g}^g$, is not entirely observered.
Therefore, to initialize
the feature state, we assume the feature lies in the $xy$ plane of the
goal frame, initializing the $z$ component of $\hat{\vect{r}}_{i/g}^g$ to zero.
% with a large enough covariance for the specific landing vehicle.
To compute the
$x$ and $y$ components of $\hat{\vect{r}}_{i/g}^g$,
% we start with the pinhole camera
% model.
% We assume that the pixel measurements in~\eqref{eq:pixel_meas} have been
% rectified to compensate for lens distortion such that
% \begin{equation}
  % \begin{bmatrix}
    % p_x \\ p_y \\ 1
  % \end{bmatrix} = \frac{1}{\e_3^\transpose \vect{p}_{i/c}^c} K \vect{p}_{i/c}^c \\
% \end{equation}
% where $K$ is the camera intrisic matrix.
we start with~\eqref{eq:pinhole_camera}
% If we invert $K$ and multiply by both
% sides, we get
% \begin{equation}
 % \frac{1}{\e_3^\transpose \vect{p}_{i/c}^c} \vect{p}_{i/c}^c
  % =
  % K^{-1} \begin{bmatrix}
    % p_x \\ p_y \\ 1
  % \end{bmatrix}
% \end{equation}
which can be rotated and solved to yield
\begin{equation}
 \vect{p}_{i/c}^v
  =
  \left( \e_3^\transpose \vect{p}_{i/c}^c \right) \left( R_I^b
  \right)^\transpose \left( R_b^c \right)^\transpose K^{-1} \begin{bmatrix}
    \bar{p}_x \\ \bar{p}_y \\ 1
  \end{bmatrix}.
\end{equation}
As $\e_3^\transpose \vect{p}_{i/c}^c$ is unknown, we define 
\begin{equation}
  \scaled{\vect{p}}_{i/c}^v
  % =
  \triangleq
   \left( R_I^b
  \right)^\transpose \left( R_b^c \right)^\transpose K^{-1} \begin{bmatrix}
    \bar{p}_x \\ \bar{p}_y \\ 1
  \end{bmatrix},
\end{equation}
which is equivalent to $\vect{p}_{i/c}^v$ up to a scale factor. As previously
mentioned, we estimate this scale factor by assuming $\e_3^\transpose
\vect{r}_{i/g}^g = 0$ such that
\begin{equation}
  \e_3^\transpose \hat{\vect{p}}_{i/c}^v = \e_3^\transpose \hat{\vect{p}}_{g/b}^v -
  \e_3^\transpose \left( \hat{R}_I^b \right)^\transpose \vect{p}_{c/b}^b.
\end{equation}
We therefore initialize
\begin{align}
    \hat{\vect{r}}_{i/g}^g &= \hat{R}_I^g \left( \hat{\vect{p}}_{i/b}^v -
    \hat{\vect{p}}_{g/b}^v \right)
\end{align}
where
% This is used to solve for $\hat{\vect{r}}_{i/g}^g$ with
\begin{align}
  % \hat{\vect{p}}_{i/c}^v &= \frac{\e_3^\transpose
  % \hat{\vect{p}}_{i/c}^v}{\e_3^\transpose \scaled{\vect{p}}_{i/c}^v} \scaled{\vect{p}}_{i/c}^v
  % \\
    \hat{\vect{p}}_{i/b}^v &= \frac{\e_3^\transpose
  \hat{\vect{p}}_{i/c}^v}{\e_3^\transpose \scaled{\vect{p}}_{i/c}^v}
  \scaled{\vect{p}}_{i/c}^v+ \left( \hat{R}_I^b \right)^\transpose
  \vect{p}_{c/b}^b.
    % \hat{\vect{p}}_{i/b}^v &= \hat{\vect{p}}_{i/c}^v + \left( R_I^b \right)^\transpose \vect{p}_{c/b}^b \\
    % \hat{\vect{r}}_{i/g}^g &= \hat{R}_I^g \left( \hat{\vect{p}}_{i/b}^v -
    % \hat{\vect{p}}_{g/b}^v \right).
\end{align}

% \begin{align}
  % \begin{bmatrix}
    % p_x \\ p_y \\ 1
  % \end{bmatrix} &= \frac{1}{\e_3^\transpose \vect{p}_{i/c}^c} K \vect{p}_{i/c}^c \\
  % \begin{bmatrix}
    % X^c / Z^c \\
    % Y^c / Z^c \\
    % 1
  % \end{bmatrix}
   % &= K^{-1}
  % \begin{bmatrix}
    % p_x \\ p_y \\ 1
  % \end{bmatrix} \\
  % \begin{bmatrix}
    % X^c / Z^c \\
    % Y^c / Z^c \\
    % 1
  % \end{bmatrix}^v
   % &= R_b^v R_c^b K^{-1}
  % \begin{bmatrix}
    % p_x \\ p_y \\ 1
  % \end{bmatrix} \\
% \end{align}
% The resulting vector, $
  % \begin{bmatrix}
    % X^c / Z^c \\
    % Y^c / Z^c \\
    % 1
  % \end{bmatrix}^v$
  % is the vector $\vect{p}_{i/c}^v$ up to a scale factor. The vector can be scaled by assuming that the altitude of the landmark is equal to the
% altitude of the goal. To get the expected altitude, we solve for
% \begin{align}
  % \e_3^\transpose \vect{p}_{g/c}^v &= \e_3^\transpose \vect{p}_{g/b}^v - \e_3^\transpose \vect{p}_{c/b}^v \\
  % \e_3^\transpose \vect{p}_{g/c}^v &= \e_3^\transpose \vect{p}_{g/b}^v -
  % \e_3^\transpose R_b^I \vect{p}_{c/b}^b \\
  % \e_3^\transpose \vect{p}_{g/c}^v &= \frac{1}{\rho_g} -
  % \e_3^\transpose R_b^I \vect{p}_{c/b}^b \\
% \end{align}

% This gives us the vector, $\vect{p}_{i/c}^v$.
% With this vector, we can then reach the estimated state vector,
% $\vect{p}_{i/g}^g$ with the following
% \begin{align}
  % \vect{p}_{i/v}^v &= \vect{p}_{i/c}^v + R_b^I \vect{p}_{c/b}^b \\
  % \vect{p}_{i/g}^g &= R_v^g \left( \vect{p}_{i/v}^v - \vect{p}_{g/v}^v \right).
% \end{align}



\section{Simulation} \label{sec:est_paper_simulation}
% !TEX root=../root.tex

To demonstrate the effectiveness of the proposed estimation algorithm, we first
present results from simulation.
% We present simulation results to demonstrate the effectiveness of the proposed estimation
% algorithm.
A multirotor UAV and a landing target vehicle were simulated using the dynamics presented
in~\eqref{eq:uav_dynamics} and~\eqref{eq:goal_dynamics} with the zero-mean
Gaussian noise processes as described
% The standard deviations used for the zero-mean
% Guassian noise processes in the dynamics are found
in~\tabref{tab:sim_process_noises}.
% As the estimator depends on
% tracking and estimating the positions of visual features that are rigidly
% attached to the landing vehicle, feature positions on the landing vehicle
Positions of visual features on the target vehicle
were also simulated by randomly sampling from the uniform distribution
% Positions for each feature in the goal frame are
% randomly sampled from the
% uniform distribution
% \begin{equation}
  % \vect{r}_{i/g}^g = \mathcal{U}
  % \left( \begin{bmatrix} -2 \\ -2 \\ -1 \end{bmatrix},
  % \begin{bmatrix} 2 \\ 2 \\ 1 \end{bmatrix} \right).
  % \label{eq:uniform_dist}
% \end{equation}
\begin{equation}
  \vect{r}_{i/g}^g = \mathcal{U}
  \left( \begin{matrix}
    \left[ -2, 2 \right] \\
    \left[ -2, 2 \right] \\
    \left[ -1, 1 \right] 
  \end{matrix}
  \right).
  \label{eq:uniform_dist}
\end{equation}
To emulate a visual feature tracker losing track of features, simulated features
randomly disappeared at each time step of the simulation.
% were randomly eliminated from the estimated state with a one percent probabilty 
% persisted to the next time step with a 99\% probability
When a feature disappeared, a new feature was generated by sampling
from~\eqref{eq:uniform_dist}
such that several
hundred, different features were used during a 30 second simulation.
% When a
% simulated feature was no longer tracked, a new feature was randomly generated by
% sampling from~\eqref{eq:uniform_dist}.
% one percent chance of 
% At each time step of the simulation, features are given a one percent chance
% of disappearing. When a feature disappears, a new feature is randomly
% generated by sampling from~\eqref{eq:uniform_dist}.

\begin{table}[htb!]
  \begin{center}
    \caption{Simulated Motion Model Parameters.}
    \label{tab:sim_process_noises}
    \begin{tabular}{l|l}
      \textbf{Parameter} & \textbf{Std. Deviation} \\
      \hline
      $\vect{\eta}_{\beta_a}$ & 0.05 m/s$^2$ \\
      $\vect{\eta}_{\beta_\omega}$ & 0.01 rad/s \\
      $\vect{\eta}_{gv}$ & 5 m/s \\
      $\vect{\eta}_{g\omega}$ & 5 rad/s \\
      % v & 5 m/s \\
      % landmark $x_{\text{min}}$ & -2 m \\
      % landmark $x_{\text{max}}$ & 2 m \\
      % landmark $y_{\text{min}}$ & -2 m \\
      % landmark $y_{\text{max}}$ & 2 m \\
      % landmark $z_{\text{min}}$ & -1 m \\
      % landmark $z_{\text{max}}$ & 1 m \\
      % landmark disappear prob. & 1\% \\
    \end{tabular}
  \end{center}
\end{table}

The initial state of the landing target vehicle was given by
\begin{equation}
  \begin{bmatrix}
    \vect{p}_{g/b}^v \\
    \vect{v}_{g/I}^g \\
    \psi_I^g \\
    \omega_{g/I}^g
  \end{bmatrix}
  =
  \begin{bmatrix}
    \begin{bmatrix} 0 & 0 & 5 \end{bmatrix}^\transpose \text{m} \\[1mm]
    % 0.5 \hphantom{\cdot} m/s \\
    \begin{bmatrix} 0.5 & 0.0 \end{bmatrix}^\transpose \text{m/s} \\
    1.5 \hphantom{\cdot} \text{rad} \\
    0.5 \hphantom{\cdot} \text{rad/s}
  \end{bmatrix}.
\end{equation}
The simulated multirotor UAV was controlled to orbit around the true position of
the target vehicle such that the target vehicle remained in the field of view of
the simulated camera for the duration of the simulation.
% Additionally, measurements from the fiducial landing marker were not

The sensor measurements described in~\secref{sec:measurement_models} were
simulated using the true state of the simulation. The rate at which each
simulated sensor was sampled and the standard deviation of the zero-mean
Guassian noise added to each measurement are found
in~\tabref{tab:sim_meas_noise}.
% The sensor measurements provided to the ESKF were simulated using the true
% state of the simulation. As described 
% in~\secref{sec:measurement_models}, these sensors included accelerometer,
% gyroscope, UAV position, UAV attitude,
% relative translation to a fiducial marker, relative rotation to a
% fiducial marker, and
% visual feature locations in a camera image.
The 
simulated 640~$\times$~480 pixel camera, described by its intrinsic matrix
\begin{equation}
  K =
  \begin{bmatrix}
    410 & 0 & 320 \\
    0 & 420 & 240 \\
    0& 0 & 1
  \end{bmatrix},
\end{equation}
was oriented at a yaw angle of $\pi/2$ rad with respect to the body
frame, such that
\begin{equation}
  R_b^c =
  \begin{bmatrix}
    0 & -1 & 0 \\
    1 & 0 & 0 \\
    0 & 0 & 1
  \end{bmatrix},
\end{equation}
and was positioned such that $\vect{p}_{c/b}^b = \begin{bmatrix}0.25, & -0.20, &
0.40 \end{bmatrix}^\transpose$ m.
% The rate at which each simulated sensor is sampled and the standard deviation of
% the zero-mean Gaussian noise added to each measurement are found
% % The sensor noise parameters along with the
% % rate at which each sensor is sampled are found
% in~\tabref{tab:sim_meas_noise}.

\begin{table}[h!]
  \begin{center}
    \caption{Simulated Sensor Characteristics.}
    \label{tab:sim_meas_noise}
    \begin{tabular}{l|l|r}
      \textbf{Measurement Type} & \textbf{Std. Deviation} & \textbf{Rate} \\
      \hline
      Accelerometer & 0.2 m/s$^2$ & 250 Hz \\
      % walk & 0.05 $m/s^2$  \\
      % init & 0.01 $m/s^2$ \\
      Gyroscope & 0.1 rad/s & 250 Hz \\
      % walk & 0.01 $rad/s$  \\
      % init & 0.01 $rad/s$ \\
      UAV global position & 0.1 m & 10 Hz \\
      UAV global attitude & 0.1 rad & 10 Hz \\
      Fiducial marker translation & 0.1 m & 30 Hz \\
      Fiducial marker rotation & 0.1 rad & 30 Hz \\
      Visual feature image point & 2.0 pixels & 30 Hz \\
    \end{tabular}
  \end{center}
\end{table}

We present the results of two simulation experiments:
\figref{fig:no_lms} shows the results of a simulation experiment in which the ESKF
does not estimate the locations of any visual features ($n = 0$);
\figref{fig:with_lms} shows the results of a simulation experiment in which
the ESKF estimates the location of ten visual features ($n = 10$).
To demonstrate the performance of the proposed estimation algorithm when the
fiducial landing marker is not detected, the measurements from the fiducial
landing marker were not used in these experiments after $t = 5$
seconds.
Note that we
do not include plots for the estimated UAV states, $\hat{\x}_{\text{UAV}}$,
as it is well known that these states can be accurately estimated with the given
measurements.

\figref{fig:no_lms} clearly shows that the covariance
% (denoted by $\pm 2 \sigma$ bounds in grey) 
of the estimated
% target vehicle
states began to grow unbounded after $t = 5$ s when $n = 0$.
This matches our intuition, as during that time, the filter received no measurements to contrain
these states.
We can also see significant error in the estimated state beginning at $t = 5$ s.
% Starting at $t = 5$ s, significant error in the estimated state is also seen.
This error resulted from an imperfect model of the target vehicle's motion.
% can also be seen when comparing the estimated
% states and the true states.
% as no measurements are
% received to constrain these states.
% Th
% As the covariance grew
% During this time, the estimates (blue lines)
% can also be seen to have significant error compared to the true states (orange
% lines).
\figref{fig:with_lms}, however, shows that when $n = 10$,
the estimates remained accurate and the covariance of the estimates remained
small for the duration of the simulation.
Note the scale differences between~\figref{fig:no_lms} and~\figref{fig:with_lms}
required to properly depict these results.
% the covariance
% of the estimated states remained small and the estimates remained
% accurate for the duration of the simulation.

\begin{figure}
  \centering
  \includegraphics[width=6.5in]{plots/single_run_no_lms}
  \caption[ESKF Simulation Results Using No Visual Features]{Simulation results when the ESKF estimated the positions of \emph{no} visual
  features. The blue line represents the true state while the orange line
  represents the estimated state. The two grey lines show $\pm 2 \sigma$ bounds for
  the estimate based on the estimated covariance. Measurements from the fiducial
  marker were not used after $t = 5$ s
  to demonstrate the performance of the estimator.}
  \label{fig:no_lms}
\end{figure}

\begin{figure}
  \centering
  \includegraphics[width=6.5in]{plots/single_run_with_lms}
  \caption[ESKF Simulation Results Using Ten Visual Features]{Simulation results
    when the ESKF estimated the positions of \emph{ten} visual
  features. The blue line represents the true state while the orange line
  represents the estimated state. The two grey lines show $\pm 2 \sigma$ bounds for
  the estimate based on the estimated covariance. Measurements from the fiducial
  marker were not used after $t = 5$ s
  to demonstrate the performance of the estimator.}
  \label{fig:with_lms}
\end{figure}


% when the ESKF does not estimate
% the locations of any visual features ($n = 0$) in~\figref{fig:no_lms}, and 
% when the ESKF estimates the location of 10 visual features ($n =
% 10$) in~\figref{fig:with_lms}.
% Simulation experiments were performed with the ESKF
% The estimated state of the target vehicle, $\hat{\x}_{\text{Goal}}$, is seen in Figs.~\ref{fig:no_lms}
% and~\ref{fig:with_lms}, where~\figref{fig:no_lms} shows the results when the
% ESKF does not estimate any visual features ($n = 0$) and~\figref{fig:with_lms}
% shows the results when the ESKF estimates the locations of 10 visual features
% ($n = 10$).

% The simulation results for a 30 second simulation in which the estimator does
% not use any visual features are seen in
% Figures~\ref{fig:no_lms_gp}, \ref{fig:no_lms_gv}, \ref{fig:no_lms_gatt}. In the
% experiment shown, the measurements from the fiducial landing marker are not used
% after $t = 5 \hphantom{\cdot} s$ to demonstrate the performance of the proposed
% estimation algorithm when the fiducial marker is not detected for significant
% periods of time. The figures clearly show that when measurements from the
% fiducial marker are not available, the covariance of the estimate grows rapidly
% and the estimated state of the landing vehicle quickly becomes inaccurate. This
% matches our intuition as the estimator receives no information about the landing
% vehicle after $t = 5 \hphantom{\cdot} s$ and therefore is only able to propagate
% the dynamics of the system. We note that the $z$ direction of the estimated
% relative goal position is the exception as the landing vehicle is constrained to
% move only in the $xy$ plane. We do not include the plots of the estimated UAV states,
% $\hat{\x}_{\text{UAV}}$, as it is well known that these states are easily
% estimated with the given measurements.

% Simulation results for the same scenario, but in which the estimator uses a
% maximum of 10 visual features, are seen in
% Figures~\ref{fig:with_lms_gp}, \ref{fig:with_lms_gv}, \ref{fig:with_lms_gatt}.
% These figures clearly show that even after $t = 5 \hphantom{\cdot} s$ when the
% fiducial marker measurements are no longer available, the estimated states of
% the landing vehicle remain accurate with relatively tight covariance bounds. We
% reiterate that the only information the estimator receives during this time is
% the measured pixel locations of unknown visual features on the landing vehicle.

To further demonstrate performance, the two previous experiments were each
repeated 100 times.
The error in the estimated position of the target vehicle in the $xy$ plane of
the inertial frame is
plotted with respect to time for each of these experiments
in~\figref{fig:mc_xy_err}. While the error when $n = 10$
remained under one meter for all 100 simulations, the error when $n = 0$
grew quickly, reaching an error of over ten meters in many cases.
% runs of the same two
% experiments were performed.
% mentioned above were performed
% both using ten visual features and using zero visual features.
% The L2 norm of
% the $xy$ error of the estimated goal position state are plotted
% with respect to time in
% Figures~\ref{fig:mc_no_lms_xy_err}~and~\ref{fig:mc_with_lms_xy_err}. While the
% error when using 10 features
% almost entirely remains under 1 $m$ for all 100 simulations, the error when using
% no features quickly grows very large, reaching an error of over 10 $m$ in many
% of the runs.

\begin{figure}
  \centering
  \includegraphics[width=5.5in]{plots/mc_both_xy_err}
  \caption[Estimation Error for 100 Simulations]{Error of the estimated position of the target vehicle in the $xy$
    plane. The top plot shows the error with respect to time for 100 simulations in which the ESKF
    estimates the positions of zero visual features. The bottom plot shows the
    error with respect to time for 100 simulations in which the ESKF estimates
    the positions of ten
    visual features.
    Measurements from the fiducial marker are not used after $t = 5$ s to
    demonstrate the performance of the estimator.
    % Simulation results with estimator using a maximum of ten visual
  % features. Measurements from the fiducial marker are not used after $t$ = 5
% $s$ to demonstrate the performance of the estimator. The L2 norm of the error in
% the x and y directions of the goal position is seen for 100 different simulation
% runs.
}
  \label{fig:mc_xy_err}
\end{figure}



\section{Hardware} \label{sec:est_paper_hardware}
% !TEX root=../root.tex

\subsection{Platform}
The proposed estimation algorithm was implemented and flown in
hardware to validate the performance observed in the simulation
results. The multirotor used for the experiments was built on a DJI 450 Flamewheel
frame. All computation was done onboard the UAV on an NVIDIA Jetson TX2 using
the Robot Operating System\footnote{Robot Operating System:
\url{www.ros.org}}. An ELP
USB Camera with a 2.1 mm lens was mounted to the bottom of the UAV such that the
camera faced downward during flight. The image from this camera was used for visual
feature tracking and fiducial marker detection.

% Throught the flight, the UAV was controlled based on the estimated state of the
% target vehicleto maintain a relative 
The multirotor UAV was manually flown until the fiducial landing marker was
detected.
Upon detection, a successive-loop PID control scheme took full control of the
UAV, closing the loop around the
estimated states. Throughout the flight, the UAV was controlled to maintain a
0.5 m altitude directly above the landing target such that $\hat{\vect{p}}_{g/b}^v =
\begin{bmatrix} 0 & 0 & 0.5 \end{bmatrix}$ m.
% constant altitude relative to the goal frame of 0.5 $m$ while attempting to
% drive the $x$ and $y$ components of $\vect{p}_{g/b}^v$ to zero.
The relative yaw
angle between the UAV and the goal frame was also controlled to zero.
Commands resulting from this control scheme were sent from
the onboard computer to a CC3D Revolution 32bit F4 flight controller running
the ROSflight firmware~\cite{jackson2016rosflight}.


% !TEX root=../root.tex

\subsection{Fiducial Landing Marker}
The fiducial landing marker used in our flight experiments was a 6.1 cm $\times$ 6.1 cm ArUco
marker~\cite{romero2018speeded}, as pictured in the top, right corner
of~\figref{fig:features_with_aruco}.
% We use an ArUco marker~\cite{romero2018speeded}, as pictured in the upper, right
% of~\figref{fig:features_with_aruco}, for the fiducial landing marker
When detected in a camera image, a relative translation and
rotation from the camera frame to the marker was estimated using the detected positions of
the corners of the marker in the camera image and the known size of the marker. 
% The ArUco library used only detects the marker when it is entirely visible, and
% unobstructed in the camera image.
% Due to these
% conditions, it is not uncommon that the marker may be undetected for significant
% periods of time during a landing maneuver.


% !TEX root=../root.tex

\subsection{Feature Tracking}
As mentioned in~\secref{sec:estimation},
the estimation algorithm uses measurements of visual features that are
rigidly attached to the landing target vehicle. It is important to note that
determining which visual features in a camera image are attached to
the target vehicle is not a trivial problem. We leave this problem as future work and
circumvent this problem by flying low enough to the target vehicle such that it
occupies the entire field of view of the camera.

Visual features were first detected using a FAST feature
detector~\cite{rosten2006machine}. The detected features were then tracked from one frame to
the next using optical flow~\cite{bouguet2001pyramidal}.
To remove features that had been poorly tracked, we periodically
estimated the essential matrix between the current camera image
and a stored keyframe.
Outliers to this estimated essential matrix were discarded, and new FAST
features were detected to replace them.
% to have been poorly tracked
% Periodically, we
% removed
% features that had been poorly tracked by estimating the essential matrix between
% the current camera image and a stored keyframe image. Outliers to the found
% essential matrix were thrown out, and new FAST features were detected to replace
% them. 
For our experiments, the feature tracker attempted to maintain 250
tracked features at all times. As the proposed estimator only used ten visual
features at a time,
% measurements from a few
% tracked features,
a subset of the tracked features which had persisted the longest
were provided to the estimator for each camera image.

Each visual feature that was acquired and tracked was assigned a unique integer
identitication number. The estimator used these identification numbers to
determine when visual features were no longer tracked and when new visual
features were acquired.
% a visual feature was no longer tracked and, therefore, should
% have been removed from
% the estimated state.
An example camera image from the UAV showing the
subset of tracked features provided to the estimator with the corresponding
identification
numbers is seen in~\figref{fig:features_with_aruco}.


\begin{figure}
  \centering
  \includegraphics[scale=0.5]{imgs/features_with_aruco.png}
  \caption[Visual Feature Tracking During Flight Experiment]{A processed camera
    image from the multirotor UAV's camera where the landing target vehicle
    occupies the entire image. The ArUco
  marker is pictured in the top, right corner of the image. Each green
circle shows the tracked location of a visual feature used by the
estimator. The red number associated with each visual feature is the unique
integer ID assigned by the feature tracker.}
  \label{fig:features_with_aruco}
\end{figure}

% !TEX root=../root.tex

\subsection{Indoor Motion Capture}
The hardware flight experiments were conducted in the indoor motion capture room
in the MAGICC Lab at Brigham Young University. An Optitrack motion capture
system provided meaurements of the global position and attitude of the UAV
throughout the flights. The motion capture system was also used as ground truth
for the position and attitude of the target vehicle.


% !TEX root=../root.tex

\subsection{Landing Target Vehicle}
As flight tests were conducted in a small indoor environment,
the landing target
vehicle was designed to be small in an attempt to better extend to outdoor
scenarios in which the UAV is to land on larger vehicles such as trucks or boats.
The target vehicle, pictured in~\figref{fig:landing_vehicle}, was manually
driven during the experiments,
roughly following an oval.
% following a rough oval.
% A ground vehicle was assembled as pictured in~\figref{fig:landing_vehicle}
% % with a
% % 6.1 $cm$ ArUco tag serving as the fiducial landing marker.
% % During the experiments, the landing vehicle was
% and manually driven around the
% perimeter of the room during the experiments.

\begin{figure}
  \centering
  \includegraphics[scale=0.5]{imgs/landing_vehicle.png}
  \caption[UAV Tracking the Target Vehicle During Flight Experiment]{Multirotor
    UAV shown autonomously tracking the landing target
  vehicle.}
  \label{fig:landing_vehicle}
\end{figure}

% !TEX root=../root.tex

\subsection{Experiment Results}
% The multirotor UAV was manually flown until the
% fiducial landing marker was detected. Upon detection, the closed-loop estimation
% and control system took full control of the UAV.

The results of the flight experiment are shown in~\figref{fig:est_hardware}.
The fiducial landing marker was first detected at $t = 20$ s, where the plots
begin.
To demonstrate the ability of the
system to maintain good tracking of a target vehicle when a fiducial
marker is not detected for long periods of time, the fiducial marker detection
was turned off ten seconds after initial detection, at $t = 30$ s.
% The fiducial landing marker was first detected at
% $t = 20$ s and last detected at $t = 30$ s.
It is clear that the estimates of the position, velocity, attitude, and angular
velocity of the target vehicle remained accurate and consistent for the duration
of the experiment.
% despite no measurements from the fiducial marker being used
% after $t = 30$ s.
% Even though no measurements from the fiducial marker are used, it is clear that
% the estimates of the position, velocitity, attitude and angular velocity of the
% target vehicle remain accurate and consistent for the duration of the
% experiment.
These accurate estimates allowed
the UAV to continue to control relative to the target vehicle, tracking closely
above the landing target as the target vehicle moved around the room. After the
target vehicle completed two full laps around the room,
at approximately $t=102$ s, manual control of the UAV was resumed, ending the experiment.
A video of the flight experiment can be found at
\url{https://youtu.be/3AyjCI0c1Nc}.
% \url{https://youtu.be/VU5sq6FuSL0}.

\begin{figure}
  \centering
  \includegraphics[width=6.5in]{plots/hardware_results}
  \caption[ESKF Hardware Results Using Ten Visual Features]{Hardware results when
    the ESKF estimated the positions of \emph{ten} visual
  features. The blue line represents the true state while the orange line
  represents the estimated state. The two grey lines show $\pm 2 \sigma$ bounds for
  the estimate based on the estimated covariance. Measurements from the fiducial
  marker were not used after $t = 30$ s to demonstrate the performance of the estimator.}
  \label{fig:est_hardware}
\end{figure}


\section{Conclusion} \label{sec:conclusion}
The proposed estimator provides a method for maintaining accurate and consistent
estimates of the state of a landing vehicle when a fiducial landing marker is
not detected for significant periods of time. This improvement is achieved by
tracking and estimating the locations of unknown visual features on the landing
vehicle. The simulation and hardware experiments show that by tracking and
estimating the locations of just 10 visual features, the multirotor UAV can
continue to reliably operate with respect to the landing vehicle for long
periods of time without detecting the fiducial landing marker.
