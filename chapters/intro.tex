% !TEX root=../master.tex

\chapter{Introduction}
\label{chp:introduction}

\section{Problem Statement}
As multirotor unmanned aerial vehicles (UAVs) have become popular
platforms for commercial and consumer products over the past decade, a variety
of new use cases have emerged that require autonomous operation from larger
vehicles acting as moving, mobile base stations.
Such applications include
maritime surveillance, package delivery, and convoy
support (see~\figref{fig:drone_convoy_support}).
While there are
many facets to operating from moving vehicles, this thesis works toward creating
a robust solution for autonomously landing multirotor UAVs onto moving vehicles
in challenging conditions.

\begin{figure}[htbp]
  \centering
  \includegraphics[width=4.5in]{figures/drone_convoy_support.png}
  \caption[UAV Convoy Support]{Multirotor UAVs used to increase
  situational awareness for a convoy of
vehicles~\cite{ground_vehicle_drone}.}
%
  \label{fig:drone_convoy_support}
\end{figure}


\section{Background}

Autonomous landing of multirotor UAVs onto moving vehicles has been previously
demonstrated in a variety of scenarios~\cite{wynn2019visual}; however, many problems still remain to
create a truly robust solution. Here we detail several of these outstanding
problems as they relate to the three principal problems that must be solved by
all autonomous UAVs: state estimation, motion planning, and control.

\subsection{State Estimation}
Most autonomous landing solutions require that the landing target vehicle is equipped
with a visual fiducial marker (e.g.~see~\figref{fig:aruco_tag})
that serves as
the designated landing platform for the UAV.
In these methods, a camera mounted to the UAV detects the fiducial landing marker, providing
information about the relative pose of the target
vehicle~\cite{borowczyk2017autonomous}. These measurements are often used in a
filtering framework, such as a Kalman filter~\cite{kalman}, to produce a
high-rate estimate of the relative state between the UAV and the target
vehicle. During landing, it is likely
that the fiducial marker is not detected for periods of time due to occlusion,
sun glare, or UAV motion. For this reason, it is important that the relative
motion between the target vehicle and the UAV be modeled and used to predict
the relative state when the fiducial marker is not detected.
However, even with a good model,
these methods quickly fail when the fiducial landing
marker is not detected for several seconds~\cite{ling2014precision}.

\begin{figure}[h]
  \centering
  \includegraphics[width=2.5in]{figures/aruco_104.png}
  \caption[Visual Fiducial Marker]{Visual fiducial markers such as this
    ArUco tag~\cite{garrido2016generation} are commonly used to assist in the
  landing of multirotor UAVs onto moving vehicles.}
  \label{fig:aruco_tag}
\end{figure}

When landing in challenging conditions such as strong winds, bright sun, or
dense fog, it is common that the fiducial marker is undetected by the UAV
for long periods of time.
Therefore, to create a truly robust landing solution, an accurate estimate of
the state of the target vehicle must be maintained when the fiducial marker is
not detected for significant durations.
Chapter~\ref{chp:estimation_paper}
describes an estimation algorithm, based on the error-state Kalman
filter~\cite{roumeliotis1999circumventing}, that tracks and estimates the locations of
visual features on the landing target vehicle to achieve this goal.

Visual feature
tracking and estimation is a common technique in the field of visual
odometry~\cite{qin2018vins}; however, almost all implementations assume the
tracked visual features are static with respect to an inertial reference frame.
In this landing scenario, the target vehicle occupies progressively more of the
field of view of the UAV's camera as the UAV descends. This makes it progressively
more difficult for typical visual odometry algorithms to track features,
resulting in a decreased
reliability of the estimates as the UAV approaches the landing target.
We also note that in most
applications, more precise state estimates are required as the UAV
approaches
the landing target to avoid obstacles and to ensure a gentle landing.
For these reasons, the described estimation algorithm
only tracks and estimates visual features that are rigidly attached to the
target vehicle.
Simulation and hardware tests of the proposed algorithm demonstrated accurate and consistent estimates
even when the fiducial marker was undetected
for long periods of time.

\subsection{Motion Planning}
The motion planning problem associated with autonomous vehicles is especially
important when 
navigating close to obstacles. In the case of landing on moving vehicles, it is
often desired that the UAV descends vertically onto the landing pad to avoid
potential obstacles such as buildings, terrain, or vehicle superstructure.
When landing on certain vehicles, such as a boat at sea, the specific time of
touchdown may also be important to minimize the risk of damage to the UAV.

Specific methods have been previously
presented to generate dynamically feasible trajectories for UAVs landing on moving
vehicles. One such method uses real-time model-predictive control techniques to
generate trajectories for a UAV landing on a moving car~\cite{baca2019autonomous}.
While improvements to these methods are not presented in this thesis, 
directions for future work related to motion planning are described
in~\secref{sec:future_motion_planning}.

\subsection{Control}
Many control algorithms have previously been presented that satisfy the
requirements for robust landing on moving vehicles.
However, a recent
movement in the robotics community aims to appropriately deal with the evolution of a
robot's state along a manifold using Lie theory~\cite{sola2018micro}. Though
Lie theory has been widely applied to the field of state
estimation~\cite{sola2017quaternion, koch2017relative}, there is little 
work applying Lie theory to optimal control of UAVs.

Chapter~\ref{chp:control_paper} derives a
linear-quadratic regulator (LQR) using Lie theory that computes control based on
the error-state dynamics of the system. Not only is this a more principled
approach than previous LQR methods, but it also achieves significant gains in computational efficiency.
Simulation and hardware results show that the derived controller can accurately
track a time-dependent trajectory, making it a good candidate for use in a
robust landing system.

\section{Summary of Contributions}
The research described in this thesis makes two significant contributions:
\begin{itemize}
\item A method of state estimation is developed that allows a multirotor UAV to
  continue to operate reliably with respect to a landing target vehicle even when a
  fiducial landing marker is not detected for significant periods of time.
\item An optimal control scheme for a multirotor UAV is derived using an error-state,
  LQR formulation that enables accurate tracking of any dynamically feasible,
  time-dependent trajectory.
\end{itemize}

These contributions are demonstrated in both simulation and hardware
experiments found in Chapters~\ref{chp:estimation_paper}
and~\ref{chp:control_paper}. These results 
give us reason to believe that a
complete and robust landing solution can be created by combining the proposed
state estimation and control schemes with a competent motion planner.

\section{Thesis Outline}

Chapter~\ref{chp:experimental_apparatus} details the hardware and software
systems used
in the experiments described in this thesis.
Chapter~\ref{chp:estimation_paper}
describes an improved method of state estimation for a multirotor UAV landing on
a moving vehicle.
Chapter~\ref{chp:control_paper} derives and demonstrates the performance of
an error-state LQR controller for a
multirotor UAV following a dynamically feasible, time-dependent trajectory.
Chapter~\ref{chp:conclusion} provides concluding remarks including
suggestions for future work that builds upon the work described in this
thesis.
