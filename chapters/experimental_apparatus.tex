% !TEX root=../master.tex

\chapter{Experimental Apparatus}
\label{chp:experimental_apparatus}

This chapter describes the software and hardware used in the experiments that
are detailed in Chapters~\ref{chp:estimation_paper} and~\ref{chp:control_paper}.

\section{Software}
\subsection {Robot Operating System}
The simulation and hardware experiments described in this thesis used the same
C++ implementations of the proposed algorithms.
This was made possible, in part, due to the use of the Robot Operating System\footnote{Robot Operating System:
\url{www.ros.org}}
(ROS) as a middleware. ROS provides a way for separate
programs, or nodes, to share information during runtime.

A network diagram of the software system can be seen in~\figref{fig:network_diagram}. The estimator and
controller for the UAV were implemented as two separate nodes. 
The estimator node computed the state estimate of the UAV and the controller
node computed the desired control action based on the estimated state.
The ROSflight flight control stack~\cite{jackson2016rosflight}
used this computed control action to actuate
either a simulated UAV or the UAV hardware platform.

\begin{figure}[htbp]
  \centering
  \includegraphics[width=3.5in]{figures/roscopter.png}
  \caption[Software Architecture Network Diagram]{Network diagram of the
  software running during simulation and hardware experiments.}
%
  \label{fig:network_diagram}
\end{figure}

\subsection {Gazebo}
The Gazebo\footnote{Gazebo:
\url{www.gazebosim.org}}
simulation environment was used
in conjunction
with the ROSflight software-in-the-loop
(SIL) simulation.
This setup provided an easier transition from software
experiments to
hardware experiments as all of the necessary software was first proven in
simulation.

\subsection {ROScopter}
The simulation and hardware experiments described in this thesis also used
the estimator and controller nodes of the ROScopter\footnote{ROScopter:
\url{www.github.com/byu-magicc/roscopter}}
project.
During testing of the estimator proposed in
Chapter~\ref{chp:estimation_paper}, the ROScopter controller node was used to
close the loop around the produced estimates.
While experimenting with the
controller proposed in Chapter~\ref{chp:control_paper}, the ROScopter estimator node
was used to provide state estimates of the UAV to the controller.

\section{Hardware}
The multirotor UAV used in hardware experimentation can be seen
in~\figref{f:drone_pic}. The UAV was built on a DJI Flamewheel 450 frame.
Specific components contained on the UAV are detailed in the following
subsections.

\begin{figure}[htbp]
  \centering
  \includegraphics[scale=0.15]{figures/hardware_platform.jpg}
  \caption[Multirotor UAV Used in Experiments]{Hardware platform used in experiments.}
  \label{f:drone_pic}
\end{figure}

\subsection{Flight Controller}
The multirotor UAV was equipped with an OpenPilot CC3D Revolution 32-bit F4
flight controller as shown in~\figref{fig:f4}. The flight controller ran the
ROSflight firmware that provided an easy interface
for the controller node on the onboard computer to control the UAV.

\begin{figure}[htbp]
  \centering
  \includegraphics[width=2.5in]{figures/f4.jpg}
  \caption[OpenPilot CC3D Revolution 32-bit F4]{OpenPilot CC3D Revolution 32-bit
    F4 flight controller~\cite{openpilotrevo}.
}
%
  \label{fig:f4}
\end{figure}

\subsection{Onboard Computer}
The computer onboard the UAV was an NVIDIA Jetson TX2 equipped with an Orbitty
carrier board. This configuration is shown in~\figref{fig:tx2_orbitty}. All
computation was done on the onboard computer during the hardware experiments.

\begin{figure}[h]
  \centering
  \includegraphics[width=2.5in]{figures/tx2_orbitty.jpg}
  \caption[NVIDIA Jetson TX2 with Orbitty Carrier Board]{NVIDIA Jetson TX2
  with an Orbitty carrier board attached~\cite{orbitty}.
  }
  \label{fig:tx2_orbitty}
\end{figure}

\subsection{Motion Capture}
Hardware flight experiments were performed in the motion capture room in the
MAGICC Lab at Brigham Young University. This room is equipped with an OptiTrack\footnote{OptiTrack:
\url{www.optitrack.com}}
motion tracking system that
provided high-rate measurements of the position and attitude of the multirotor
UAV during the experiments.

\subsection{Camera}
For the hardware experiments described in Chapter~\ref{chp:estimation_paper},
the multirotor UAV was outfitted with an ELP USB camera with a 2.1 mm lens as
shown in~\figref{fig:camera}. The intrinsic parameters of the sensor were
accurately calibrated using a software package provided by ROS\footnote{ROS
camera\_calibration:
\url{wiki.ros.org/camera_calibration}}.
However, the mounting position and attitude of the camera relative to the UAV were only roughly
approximated.

\begin{figure}[h]
  \centering
  \includegraphics[width=2.5in]{figures/camera.jpg}
  \caption[ELP USB Camera with 2.1 mm Lens]{ELP USB Camera with a 2.1 mm
    lens~\cite{webcam}.
}
%
  \label{fig:camera}
\end{figure}

