% !TEX root=../root.tex

\subsection{Traditional LQR}

A linear-quadratic regulator provides the optimal state-space controller gains
for an LTI  system given by
\begin{equation}
\dot{{\bf x}}=A{\bf x}+B{\bf u},
\end{equation}
assuming full-state feedback. We define the cost-to-go for the infinite-time solution as
\begin{equation}
  J(\vect{x},\vect{u})=\int_{0}^{\infty}\left(\vect{x}^{\top}Q\vect{x}+\vect{u}^{\top}R\vect{u}\right)dt
\label{eq:lqr_cost}
\end{equation}
with $Q$ and $R$ being positive definite matrices that
define the costs associated with the state and the input. The cost function
given in~\eqref{eq:lqr_cost} is minimized by the control input
\begin{equation}
  \vect{u} = -K\vect{x},
\end{equation}
where $K$ is given by
\begin{equation}
  K = R^{-1}B^{\top}P,
\end{equation}
and $P$ is the solution to the continuous-time algebraic Riccati equation (CARE),
\begin{equation}
  A^{\top}P + PA -PBR^{-1}B^{\top}P + Q = 0.
\end{equation}

It should be noted that in its basic form, an LQR controller is simply a
regulator and the control input $\vect{u}$ will only attempt to drive the state
to zero in an optimal way. If the desire is for the system to reach a desired
state, $\des{\vect{x}}$, one can start by defining the error-state as
\begin{equation}
  \tilde{\vect{x}}=\vect{x}-\des{\vect{x}}
  \label{eq:vector_error_state}
\end{equation}
and redefining the control input as
\begin{equation}
  \tilde{\vect{u}} = -K\tilde{\vect{x}}.
\end{equation}
This technique, however, will generally result in steady-state error between the
state $\vect{x}$ and the reference trajectory $\des{\vect{x}}$. The steady-state error can be removed by augmenting the state with an integrator or by applying a model-based feed-forward control input,
\begin{equation}
  \vect{u} = \tilde{\vect{u}} + \des{\vect{u}} = -K\tilde{\vect{x}} + \des{\vect{u}}.
\end{equation}

%In the following sections, we apply an error state LQR controller to
%control the position, velocity, and attitude of a multirotor vehicle. However, a
A direct application of~\eqref{eq:vector_error_state} in our case is not defined
because the multirotor state is not a vector quantity.
%In other words, any attempt
%to directly add or subtract an attitude representation will leave the manifold
To compensate for this, we propose to compute control based
on the error-state dynamics of the system, where the error-state is purely a vector
quantity.
