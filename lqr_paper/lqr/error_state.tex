% !TEX root=../root.tex

\subsection{Error-State LQR}

We can apply the same LQR approach to the error-state dynamics from
\eqref{eq:error_state_dynamics}. Since LQR control is a regulator, it will drive
the error-state to zero, or our current state to our desired state. Since LQR is
only defined for an LTI system, we can approximate the error-state system as an
LTI system by linearizing about the current state at each time step. This gives
us the system
%Heres the code that implements these
  %A_.setZero();
  %A_.block<3, 3>(dxPOS, dxVEL) = R_I_b.transpose();
  %A_.block<3, 3>(dxPOS, dxATT) = -R_I_b.transpose() * skew_vel;

  %A_.block<3, 3>(dxVEL, dxVEL) = -M - skew_omega;
  %A_.block<3, 3>(dxVEL, dxATT) = skew(R_I_b * grav_vec);

  %A_.block<3, 3>(dxATT, dxATT) = -skew_omega;

  %B_.setZero();
  %B_.block<3, 1>(dxVEL, uTHROTTLE) = -(grav_val_ / hover_throttle_) * e3;
  %B_.block<3, 3>(dxVEL, uOMEGA) = skew_vel;

  %B_.block<3, 3>(dxATT, uOMEGA) = Eigen::Matrix3d::Identity();

\begin{equation}
\dot{\tilde{{\bf x}}}=A\tilde{{\bf x}}+B\tilde{{\bf u}}
\end{equation}
with the matrices $A$ and $B$ given by
\begin{align}
A({\bf x},{\bf u}) & =\frac{\partial}{\partial\tilde{{\bf x}}}f({\bf x},\tilde{{\bf x}},{\bf u},\tilde{{\bf u}})\\
B({\bf x},{\bf u}) & =\frac{\partial}{\partial\tilde{{\bf u}}}f({\bf x},\tilde{{\bf x}},{\bf u},\tilde{{\bf u}}).
\end{align}
Using the error-state dynamics in \eqref{eq:error_state_dynamics} and dropping
the subscripts and superscripts for compactness it can
be seen that
\begin{align}
A({\bf x},{\bf u}) & =\frac{\partial}{\partial\tilde{{\bf x}}}f({\bf x},\tilde{{\bf x}},{\bf u},\tilde{{\bf u}})\\
 & =\begin{bmatrix}{\bf 0} & \frac{\partial\dot{\tilde{{\bf
   p}}}}{\partial\tilde{{\bf v}}} & \frac{\partial\dot{\tilde{{\bf
   p}}}}{\partial\tilde{\vect{r}}} \\[4pt]
{\bf 0} & \frac{\partial\dot{\tilde{{\bf v}}}}{\partial\tilde{{\bf v}}} &
\frac{\partial\dot{\tilde{{\bf v}}}}{\partial\tilde{\vect{r}}} \\[4pt]
{\bf 0} & {\bf 0} & \frac{\partial\dot{\tilde{\vect{r}}}}{\partial\tilde{\vect{r}}}
\end{bmatrix}
\end{align}
with the individual components given by
\begin{align}
  \frac{\partial\dot{\tilde{\vect{p}}}}{\partial\tilde{\vect{v}}} &
  =\left(R_{I}^{b}\right)^\transpose \\
  \frac{\partial\dot{\tilde{\vect{p}}}}{\partial\tilde{\vect{r}}} &
  =-\left(R_{I}^{b}\right)^\transpose \skewmat{\vect{v}_{b/I}^{b}} \\
  \frac{\partial\dot{\tilde{\vect{v}}}}{\partial\tilde{\vect{v}}} & =
-\drag\left(I - \e_3\e_3^\transpose\right) - \skewmat{\vect{\omega}_{b/I}^{b}} \\
  \frac{\partial\dot{\tilde{\vect{v}}}}{\partial\tilde{\vect{r}}} & =
  \skewmat{gR_{I}^{b}\e_3} \\
  \frac{\partial\dot{\tilde{\vect{r}}}}{\partial\tilde{\vect{r}}} & =
  -\skewmat{\vect{\omega}_{b/I}^{b}}.
\end{align}
It can similarly be seen that
\begin{align}
B({\bf x},{\bf u}) & =\frac{\partial}{\partial\tilde{{\bf u}}}f({\bf x},\tilde{{\bf x}},{\bf u},\tilde{{\bf u}})\\
                   & =\begin{bmatrix}{\bf 0} & {\bf 0}\\[4pt]
\frac{\partial\dot{\tilde{{\bf v}}}}{\partial\tilde{s}} & {\bf
\frac{\partial\dot{\tilde{{\bf v}}}}{\partial\tilde{\omega}}}\\[4pt]
{\bf 0} & \frac{\partial\dot{\tilde{\vect{r}}}}{\partial\tilde{\omega}}
\end{bmatrix}
\end{align}
with the individual components given by
\begin{align}
  \frac{\partial\dot{\tilde{\vect{v}}}}{\partial\tilde{s}} & =-\frac{g}{s_e}\e_{3}\\
  \frac{\partial\dot{\tilde{\vect{v}}}}{\partial\tilde{\vect{\omega}}} &
  =\skewmat{{\bf v}_{b/I}^{b}} \\
  \frac{\partial\dot{\tilde{\vect{r}}}}{\partial\tilde{\vect{\omega}}} & =I_{3\times3}.
\end{align}

By linearizing at every time step, the CARE
must be solved at each time step with the current $A$ and $B$ matrices. We use the
closed-form, Schur decomposition method described in~\cite{laub1979schur}. This
method allows us to relinearize and recompute the optimal control at full rate
in our experiments.

Although we relinearize and solve the CARE at each time step, the $Q$ and $R$
weighting matrices are fixed. We choose these gains based on Bryson's
rule~\cite{hespanha2018linear}. In addition, we have found that better results
are achieved by saturating the error-state in accordance with the maximum error
terms used to choose the gains with Bryson's rule.
