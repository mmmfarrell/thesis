% !TEX root=../root.tex

To test the proposed error-state LQR controller, we designed two experiments to be
performed in simulation and hardware: (i) tracking step inputs to the desired
position of the UAV and (ii) tracking time-dependent full-state trajectories. We
first explain how we generate these time-dependent trajectories for the
experiments and then detail the experimental setup for both simulation and
hardware.

%dynamically feasible trajectories. For simplicity in trajectory tracking, we use the temporal
%tracking scheme, where the desired state of the multirotor varies with time as
%opposed to the spatial tracking scheme where the multirotor must simply follow a
%trajectory in space~\cite{foehn2018onboard}.

\subsection{Trajectory Generation}
\label{subsec:trajectory_gen}

One important consideration in high-performance control of quadrotors is the
generation of smooth, feasible trajectories.  The quadrotor has the benefit of
being \emph{differentially flat} which means that the required inputs to the
quadrotor can be fully defined using derivatives of the outputs of the system,
the desired position and
heading~\cite{mellinger2011minimum}. If we are given some smooth, differentiable trajectory of our desired
position and heading then we can compute the full state and required inputs as
a function of time
\begin{equation}
	\begin{pmatrix} \des{\vect{p}}_{b/I}^I \left( t \right) \\
	                \des{\q}_I^b \left( t \right) \\
	                \des{\vect{v}}_{b/I}^I \left( t \right) \\
	                \des{s} \left( t \right) \\
	                \des{\vect{\omega}}_{b/I}^I \left( t \right)
	\end{pmatrix} = 
	\des{f} \begin{pmatrix} 
					\des{\vect{p}}_{b/I}^I\left(t\right) \\ 
					\dot{\des{\vect{p}}}_{b/I}^I\left(t\right) \\ 
					\ddot{\des{\vect{p}}}_{b/I}^I\left(t\right) \\ 
					\des{\psi}_{b/I}^I\left(t\right)
					%\dot{\des{\psi}}_{b/I}^I\left(t\right)
	\end{pmatrix}.
\end{equation}
We derive the general case where the desired yaw angle of the multirotor UAV is
a function of time. However, since it is well known that the yaw angle of a multirotor UAV is
easily controllable independent of the other states~\cite{mellinger2011minimum}, we simply command a constant zero yaw in our experiments. 

We now derive the differentially
flat outputs.  First, desired position is given to us directly
\begin{equation}
	\des{\vect{p}}_{b/I}^I = \des{\vect{p}}_{b/I}^I. 
\end{equation}
To derive desired attitude and throttle signal we start 
by applying Newton's second law, and consider the rotation from the heading-rotated, body-level
coordinate frame, $\ell$, to
the body frame, $b$. Note that to avoid the need for an iterative solution, we neglect the forces due to drag
that are accounted for in the quadrotor dynamics in \eqref{eq:dynamics}.
Newton's second law is given by
%force balance is given by
\begin{align}
	\sum F^I &= m\ddot{\des{\vect{p}}}_{b/I}^I\\
    -T \left(\des{R}_\ell^b\right)^\transpose + mg\e_3 &= m\ddot{\des{\vect{p}}}_{b/I}^I\\
    \frac{T}{m}\left(\des{R}_\ell^b\right)^\transpose\e_3 &= g\e_3 -
    \ddot{\des{\vect{p}}}_{b/I}^I.
\end{align}
If we then define $\des{\vect{a}} = g\e_3 - \ddot{\des{\vect{p}}}_{b/I}^I$,
then we get the following expression
\begin{equation}
	\frac{T}{m}\left(\des{R}_\ell^b\right)^\transpose\e_3 = \des{\vect{a}}
\end{equation}
where $T$ and $\des{R}_\ell^b$ must satisfy the following conditions:
\begin{align}
	\frac{T}{m} &= \norm{\des{\vect{a}}} \\
	\quad \des{R}_\ell^I\e_3 &= \frac{\des{\vect{a}}}{\norm{\des{\vect{a}}}}. \label{eq:rot_diff_flat}
\end{align}
Because $T=\tfrac{s}{s_e}gm$, then
\begin{equation}
	\des{s} = \frac{s_e}{g}\norm{\des{\vect{a}}}
\end{equation}
and \eqref{eq:rot_diff_flat} can be solved with
\begin{equation}
	%\des{R}_\ell^b = \exp \left( \skewmat{\theta \vect{\delta}} \right) 
  \des{\q}_\ell^b = \exp_{\q} \left( \theta \vect{\delta} \right)
\end{equation}
where
\begin{align}
  \theta &= \mathrm{cos}^{-1}\left(\vect{e}_3^\transpose\frac{\des{\vect{a}}}{\norm{\des{\vect{a}}}}\right) \\
	\vect{\delta} &= \skewmat{\vect{e}_3}\frac{\des{\vect{a}}}{\norm{\des{\vect{a}}}}.
\end{align}
The heading portion of attitude can now be applied to give us our full desired
attitude
\begin{equation}
	%\des{R}_I^b = \des{R}_\ell^b \exp \left( \skewmat{\des{\psi} \e_3} \right).
  \des{\q}_I^b = \exp_{\q} \left( \des{\psi} \e_3 \right) \otimes \des{\q}_\ell^b.
\end{equation}
Desired velocity can be found using the desired attitude
\begin{equation}
	\des{\vect{v}}_{b/I}^b = \des{R}_I^b \dot{\des{\vect{p}}}_{b/I}^I,
\end{equation}
and the required angular rate is found by taking the time derivative of our desired attitude
\begin{equation}
	\des{\vect{\omega}}_{b/I}^b = \frac{d}{dt} \des{\q}_I^b.
\end{equation}
In our implementation, we do this numerically with central differencing on the manifold.

In summary, the differentially flat output of a quadrotor is given as
follows:
\begin{align}
        \label{eq:traj_p}
	\des{\vect{p}}_{b/I}^I &= \des{\vect{p}}_{b/I}^I \\
	%\des{R}_I^b &= \des{R}_\ell^b \exp \left( \skewmat{\des{\psi} \e_3} \right) \\
	\des{\q}_I^b &= \exp_{\q} \left( \des{\psi} \e_3 \right) \otimes
        \des{\q}_\ell^b \\
	\des{\vect{v}}_{b/I}^b &= \des{R}_I^b \dot{\des{\vect{p}}}_{b/I}^I \\
	\des{s} &= \frac{s_e}{g}\norm{g\e_3 - \ddot{\des{\vect{p}}}_{b/I}^I} \\
	%\des{\vect{\omega}}_{b/I}^b &= \frac{d}{dt} \des{R}_I^b.
	\des{\vect{\omega}}_{b/I}^b &= \frac{d}{dt} \des{\q}_I^b.
        \label{eq:traj_omega}
\end{align}


The examples in this work all reference the same figure-eight trajectory defined by 
\begin{align}
	\des{\vect{p}}_{b/I}^I\left(t\right) &= \vect{p}_{b/I}^I\left(t_0\right)
        + \begin{bmatrix}\delta_x \sin\left(\tfrac{T}{2\pi} t\right) \\[4pt]
        \delta_y \sin\left(\tfrac{T}{\pi} t\right) \\[4pt] \delta_z \sin\left(\tfrac{T}{2\pi} t\right)\end{bmatrix} \\
	\des{\psi}_{b/I}^I\left(t\right) &= 0
	%\des{\psi}_{b/I}^I\left(t\right) &= \psi_{b/I}^I\left(t\right) + \delta_\psi \sin\left(\tfrac{T}{2\pi} t\right)
\end{align}
where the $\delta_{\left(\cdot\right)}$ parameters define the dimensions of the
trajectory, and $T$ defines the period.  While a trajectory defined by periodic
functions is useful for simple demonstrations such as what we perform in this
work, we direct the reader to more sophisticated methods of differentiable
trajectory generation such as~\cite{mellinger2011minimum} for practical application.
