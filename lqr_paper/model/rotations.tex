% !TEX root=../root.tex

\subsection{Rotations}
\MF{If we use quaternions throughout (because thats part of the point) maybe we
don't need this. But we can take the first talk about passive rotations}

We use passive rotation matrices, meaning that the rotation matrix $R_a^b$ acts on a vector $\vect{r}^a$, expressed in frame $a$, to express it in frame $b$ as
\begin{equation}
\vect{r}^b = R_a^b \vect{r}^a .
\end{equation}
By representing orientation states directly as rotation matrices, as opposed to other representations such as Euler angles or axis-angle, the group structure of the rotation is preserved.
While unit-quaternions also preserve the group structure in a more concise representation, we choose rotation matrices for clarity of presentation in the following derivations.
However, the companion software library implements techniques described in this paper using quaternions, and the mapping between the rotation-matrix and quaternion representations is described in \appxref{appx:quaternions}.

Rotation matrices are elements of the special orthogonal group $\SO(3)$, where the group operator is matrix multiplication.
Incremental changes or perturbations happen in the tangent space.
The tangent space at the identity element $R=I$ is the Lie algebra $\so(3)$.
The exponential and logarithmic mappings map between the group and the algebra:
\begin{align}
\exp &: \so(3) \to \SO(3) \\
\log &: \SO(3) \to \so(3)
\end{align}
For $\SO(3)$ these mappings are the matrix exponential and matrix logarithm.

The Lie algebra is a vector space, and so is isomorphic to $\mathbb{R}^3$.
The hat $^\wedge$ and vee $^\vee$ operators describe this invertible mapping:
\begin{align}
^\wedge &: \mathbb{R}^3 \to \so(3) \\
^\vee &: \so(3) \to \mathbb{R}^3 .
\end{align}
For $\so(3)$ the $^\wedge$ operator is the skew-symmetric operator:
\begin{equation}
^\wedge : \vect{v} \mapsto \skewmat{\vect{v}} ,
\end{equation}
and the $^\vee$ operator is simply the inverse operation.

For convenience in writing differential quantities as vectors in $\mathbb{R}^3$, we define the capitalized exponential and logarithmic mappings as
\begin{align}
\Exp &: \mathbb{R}^3 \to \SO(3) \\
\Exp &: \vect{\omega} \mapsto \exp(\vect{\omega}^\wedge) ,
\end{align}
and
\begin{align}
\Log &: \SO(3) \to \mathbb{R}^3 \\
\Log &: R \mapsto \log(R)^\vee .
\end{align}
For $\SO(3)$, a closed-form solution for the $\Exp$ mapping is given by the Rodriques formula.
These operators are related to the $\boxplus$/$\boxminus$ notation originally described in~\cite{hertzberg2013integrating} as
\begin{align}
R \boxplus \vect{\omega} &\triangleq R \Exp(\vect{\omega}) , \\
R_2 \boxminus R_1 &\triangleq \Log({R_1}^\transpose R_2) .
\end{align}
\DK{Do we want a symbol for the group operator? (e.g. $R_1 \circ R_2$ or $R_1 \cdot R_2$ vs $R_1 R_2$)}
