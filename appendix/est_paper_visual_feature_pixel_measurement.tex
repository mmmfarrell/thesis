\chapter{Derivation of Visual Feature Pixel Measurement Model for the Error-State Kalman Filter}
\label{apdx:estimation_pixel_meas_model}

To derive the measurement Jacobian for the visual feature pixel measurements, we
first start with the residual model from~\eqref{eq:rpix}
\begin{align}
  \vect{r}_{\text{pix}}
  &= \frac{1}{\e_3^\transpose \vect{p}_{i/c}^c} I_{2 \times 3} K
  \vect{p}_{i/c}^c + \vect{\eta}_{\text{pix}} - \frac{1}{\e_3^\transpose \hat{\vect{p}}_{i/c}^c} I_{2 \times 3} K
  \hat{\vect{p}}_{i/c}^c.
\end{align}
We then note that
\begin{align}
  H_{\text{pix}} &= \frac{ \partial \vect{r}_{\text{pix}} }{ \partial \tilde{\x}
  } \\
         &= \frac{ \partial }{ \partial \tilde{\x}}
         \left( \frac{1}{\e_3^\transpose \vect{p}_{i/c}^c} I_{2 \times 3} K
         \vect{p}_{i/c}^c \right) + \frac{ \partial }{ \partial \tilde{\x}}
         \left( \vect{\eta}_{\text{pix}} \right) - \frac{ \partial }{ \partial
           \tilde{\x}} \left( \frac{1}{\e_3^\transpose \hat{\vect{p}}_{i/c}^c} I_{2 \times 3} K
         \hat{\vect{p}}_{i/c}^c \right) \\
         &= \frac{ \partial }{ \partial \tilde{\x}}
         \left( \frac{1}{\e_3^\transpose \vect{p}_{i/c}^c} I_{2 \times 3} K
         \vect{p}_{i/c}^c \right) + \vect{0} + \vect{0} \\
          &= 
         \frac{1}{\e_3^\transpose \vect{p}_{i/c}^c} I_{2 \times 3} K
         \frac{\partial}{\partial \tilde{\x}} \vect{p}_{i/c}^c 
         - \frac{\e_3^\transpose \frac{\partial}{\partial \tilde{\x}} 
         \vect{p}_{i/c}^c}{\left( \e_3^\transpose
         \vect{p}_{i/c}^c \right)^2 } I_{2 \times 3} K
          \vect{p}_{i/c}^c. 
          \label{eq:H_pix_expanded}
\end{align}
As $\vect{p}_{i/c}^c$ appears in~\eqref{eq:H_pix_expanded}, we substitute in our
best possible estimate for this vector, $\hat{\vect{p}}_{i/c}^c$, such that
\begin{align}
  H_{\text{pix}} 
  &\approx
  \frac{1}{\e_3^\transpose \hat{\vect{p}}_{i/c}^c} I_{2 \times 3} K
 \frac{\partial}{\partial \tilde{\x}} \vect{p}_{i/c}^c 
 - \frac{\e_3^\transpose \frac{\partial}{\partial \tilde{\x}} 
 \vect{p}_{i/c}^c}{\left( \e_3^\transpose
 \hat{\vect{p}}_{i/c}^c \right)^2 } I_{2 \times 3} K
 \hat{\vect{p}}_{i/c}^c. 
\end{align}

To solve for $\frac{\partial}{\partial \tilde{\x}} \vect{p}_{i/c}^c$, we must
first establish a few approximations related to the landing vehicle attitude
state, $\psi_I^g$. In~\eqref{eq:2d_err_rot_approx} we established that the 2D
rotation matrix created from $\tilde{\psi}_I^g$ can be approximated as
\begin{equation}
  \left( R_{\text{2D}} \left( \tilde{\psi}_I^g \right) \right)^\transpose \approx I -
  \skewmat{\tilde{\psi}_I^g}.
\end{equation}
We extend this to the 3D rotation matrix created from $\tilde{\psi}_I^g$ noting
that
\begin{equation}
  \left( R_{\text{3D}} \left( \tilde{\psi}_I^g \right) \right)^\transpose \approx I -
  \skewmat{
    \begin{matrix}
  0 \\ 0 \\ \tilde{\psi}_I^g  \end{matrix}
  }.
  \label{eq:3d_psi_rot_approx}
\end{equation}

We now expand~\eqref{eq:p_i_c_c} using the error-state definitions along with
the approximations given in~\eqref{eq:3d_psi_rot_approx}
and~\eqref{eq:est_paper_Rapprox}:
\begin{align}
  \vect{p}_{i/c}^c &= R_b^c \left( R_I^b \left( R_I^g \right)^\transpose
  \vect{r}_{i/g}^g + R_I^b \vect{p}_{g/b}^v - \vect{p}_{c/b}^b \right) \\
  &= R_b^c \left( \left( I - \skewmat{\tilde{\vect{\theta}}_I^b
    } \right) \hat{R}_I^b \left( I +
    \skewmat{ \begin{matrix} 0 \\ 0 \\ \tilde{\psi}_I^g \end{matrix} }
  \right) \left( \hat{R}_I^g \right)^\transpose
  \left( \hat{\vect{r}}_{i/g}^g + \tilde{\vect{r}}_{i/g}^g \right) \right. \\
  & \qquad \left. + \left( I -
  \skewmat{\tilde{\vect{\theta}}_I^b} \right) \hat{R}_I^b
  \left( \hat{\vect{p}}_{g/b}^v + \tilde{\vect{p}}_{g/b}^v \right) -
  \vect{p}_{c/b}^b \right). \nonumber \\
\end{align}
By simplifying and ignoring higher-order terms
\begin{align}
  \vect{p}_{i/c}^c 
  &= R_b^c \left( \hat{R}_I^b \left( \hat{R}_I^g \right)^\transpose
  \hat{\vect{r}}_{i/g}^g + \hat{R}_I^b \hat{\vect{p}}_{g/b}^v - \vect{p}_{c/b}^b
  - \skewmat{\tilde{\vect{\theta}}_I^b} \hat{R}_I^b \left( \hat{R}_I^g
    \right)^\transpose \hat{\vect{r}}_{i/g}^g \right. \\
  & \qquad \left. + \hat{R}_I^b
    \skewmat{\begin{matrix} 0 \\ 0 \\ \tilde{\psi}_I^g \end{matrix}} \left( \hat{R}_I^g
  \right)^\transpose \hat{\vect{r}}_{i/g}^g +
\hat{R}_I^b \left( \hat{R}_I^g
    \right)^\transpose \tilde{\vect{r}}_{i/g}^g
    - \skewmat{\tilde{\vect{\theta}}_I^b} \hat{R}_I^b \hat{\vect{p}}_{g/b}^v
    + \hat{R}_I^b \tilde{\vect{p}}_{g/b}^v
   \right). \nonumber \\
    &= \hat{\vect{p}}_{i/c}^c
    + R_b^c \left( 
  - \skewmat{\tilde{\vect{\theta}}_I^b} \hat{R}_I^b \left( \hat{R}_I^g
    \right)^\transpose \hat{\vect{r}}_{i/g}^g + \hat{R}_I^b
    \skewmat{\begin{matrix} 0 \\ 0 \\ \tilde{\psi}_I^g \end{matrix}} \left( \hat{R}_I^g
  \right)^\transpose \hat{\vect{r}}_{i/g}^g \right. \label{eq:p_i_c_c_simplified} \\
    & \qquad \left. +
\hat{R}_I^b \left( \hat{R}_I^g
    \right)^\transpose \tilde{\vect{r}}_{i/g}^g
                   - \skewmat{\tilde{\vect{\theta}}_I^b} \hat{R}_I^b \hat{\vect{p}}_{g/b}^v
    + \hat{R}_I^b \tilde{\vect{p}}_{g/b}^v
   \right) \nonumber. 
\end{align}
Using~\eqref{eq:p_i_c_c_simplified}, we define $\frac{\partial}{\partial \tilde{\x}} \vect{p}_{i/c}^c$ by its
non-zero components,
\begin{align}
  \frac{\partial}{\partial \tilde{\vect{\theta}}_I^b } \vect{p}_{i/c}^c
  &=
  R_b^c \skewmat{ \hat{R}_I^b \left( \left( \hat{R}_I^g \right)^\transpose
  \hat{\vect{r}}_{i/g}^g + \hat{\vect{p}}_{g/b}^v \right) } \\
  \frac{\partial}{\partial \tilde{\vect{p}}_{g/b}^v} \vect{p}_{i/c}^c
  &=
  R_b^c \hat{R}_I^b \\
  \frac{\partial}{\partial \tilde{\psi}_I^g} \vect{p}_{i/c}^c
  &=
  R_b^c \hat{R}_I^b \left( \hat{R}_I^g \right)^\transpose
  \skewmat{\begin{matrix} 0 \\ 0 \\ 1 \end{matrix}} \hat{\vect{r}}_{i/g}^g \\
  \frac{\partial}{\partial \tilde{\vect{r}}_{i/g}^g} \vect{p}_{i/c}^c
  &=
  R_b^c \hat{R}_I^b \left( \hat{R}_I^g \right)^\transpose .
\end{align}


% \MF{TODO put this stuff in another appendix}
% To compute the measurement Jacobian, we first expand~\eqref{eq:p_i_c_c}
% \begin{align}
  % \vect{p}_{i/c}^c &= R_b^c \left( \left( I - \skewmat{\tilde{\vect{\theta}}_I^b
    % } \right) \hat{R}_I^b \left( I + \skewmat{\tilde{\vect{\theta}}_I^g }
  % \right) \left( \hat{R}_I^g \right)^\transpose
% \left( \hat{\vect{r}}_{i/g}^g  + \tilde{\vect{r}}_{i/g}^g \right) + \left( I - \skewmat{\tilde{\vect{\theta}}_I^b
% } \right) \hat{R}_I^b \left( \hat{\vect{p}}_{g/b}^v + \tilde{\vect{p}}_{g/b}^v \right) - \vect{p}_{c/b}^b \right).
% \end{align}
% By simplifying and ignoring secord-order terms,
% \begin{align*}
  % \vect{p}_{i/c}^c &= R_b^c \left( \hat{R}_I^b \left( \hat{R}_I^g \right)^\transpose
  % \hat{\vect{r}}_{i/g}^g + \hat{R}_I^b \hat{\vect{p}}_{g/b}^v - \vect{p}_{c/b}^b
  % - \skewmat{\tilde{\vect{\theta}}_I^b} \hat{R}_I^b \left( \hat{R}_I^g
    % \right)^\transpose \hat{\vect{r}}_{i/g}^g + \hat{R}_I^b
    % \skewmat{\tilde{\vect{\theta}}_I^g} \left( \hat{R}_I^g
  % \right)^\transpose \hat{\vect{r}}_{i/g}^g +
% \hat{R}_I^b \left( \hat{R}_I^g
    % \right)^\transpose \tilde{\vect{r}}_{i/g}^g \right. \\
                   % &- \left. \skewmat{\tilde{\vect{\theta}}_I^b} \hat{R}_I^b \hat{\vect{p}}_{g/b}^v
    % + \hat{R}_I^b \tilde{\vect{p}}_{g/b}^v
   % \right). \\
    % &= \hat{\vect{p}}_{i/c}^c
    % + R_b^c \left( 
  % - \skewmat{\tilde{\vect{\theta}}_I^b} \hat{R}_I^b \left( \hat{R}_I^g
    % \right)^\transpose \hat{\vect{r}}_{i/g}^g + \hat{R}_I^b
    % \skewmat{\tilde{\vect{\theta}}_I^g} \left( \hat{R}_I^g
  % \right)^\transpose \hat{\vect{r}}_{i/g}^g +
% \hat{R}_I^b \left( \hat{R}_I^g
    % \right)^\transpose \tilde{\vect{r}}_{i/g}^g
                   % - \skewmat{\tilde{\vect{\theta}}_I^b} \hat{R}_I^b \hat{\vect{p}}_{g/b}^v
    % + \hat{R}_I^b \tilde{\vect{p}}_{g/b}^v
   % \right), 
% \end{align*}
% which we define as
% \begin{equation*}
  % \vect{p}_{i/c}^c = \hat{\vect{p}}_{i/c}^c + \tilde{\vect{p}}_{i/c}^c .
% \end{equation*}
% For simplicity we note that 
% \begin{equation}
  % H_{\text{pix}} = \frac{\partial}{\partial \tilde{\x}} \vect{r}_{\text{pix}}
% \end{equation}
% and that
% \begin{align*}
  % \frac{\partial}{\partial \tilde{\x}} \left( \frac{1}{\e_3^\transpose \hat{\vect{p}}_{i/c}^c} I_{2 \times 3} K
  % \hat{\vect{p}}_{i/c}^c \right) &= \vect{0}
% \\
  % \frac{\partial}{\partial \tilde{\x}} \vect{\eta}_{\text{pix}} &= \vect{0}
% \end{align*}
% such that
% \begin{equation}
  % H_{\text{pix}} = \frac{\partial}{\partial \tilde{\x}}
 % \left( \frac{1}{\e_3^\transpose \vect{p}_{i/c}^c} I_{2 \times 3} K
 % \vect{p}_{i/c}^c \right).
% \end{equation}
% We expand this using the chain rule to show that
% \begin{align}
  % H_{\text{pix}} &= \frac{\partial}{\partial \tilde{\x}}
 % \left( \frac{1}{\e_3^\transpose \vect{p}_{i/c}^c} I_{2 \times 3} K
 % \vect{p}_{i/c}^c \right) \\
  % &= 
 % \frac{1}{\e_3^\transpose \vect{p}_{i/c}^c} I_{2 \times 3} K
 % \frac{\partial}{\partial \tilde{\x}} \vect{p}_{i/c}^c 
 % - \frac{\e_3^\transpose \frac{\partial}{\partial \tilde{\x}} 
 % \vect{p}_{i/c}^c}{\left( \e_3^\transpose
 % \vect{p}_{i/c}^c \right)^2 } I_{2 \times 3} K
  % \vect{p}_{i/c}^c. 
  % % \label{eq:H_pix_expanded}
% \end{align}
% As $\vect{p}_{i/c}^c$ appears in~\eqref{eq:H_pix_expanded}, we substitute in our
% best possible estimate for this vector, $\hat{\vect{p}}_{i/c}^c$, such that
% \begin{align}
  % H_{\text{pix}} 
  % &\approx
  % \frac{1}{\e_3^\transpose \hat{\vect{p}}_{i/c}^c} I_{2 \times 3} K
 % \frac{\partial}{\partial \tilde{\x}} \vect{p}_{i/c}^c 
 % - \frac{\e_3^\transpose \frac{\partial}{\partial \tilde{\x}} 
 % \vect{p}_{i/c}^c}{\left( \e_3^\transpose
 % \hat{\vect{p}}_{i/c}^c \right)^2 } I_{2 \times 3} K
 % \hat{\vect{p}}_{i/c}^c. 
% \end{align}
% We define $\frac{\partial}{\partial \tilde{\x}} \vect{p}_{i/c}^c$ by its
% non-zero components,
% \begin{align*}
  % \frac{\partial}{\partial \tilde{\vect{\theta}}_I^b } \vect{p}_{i/c}^c
  % &=
  % R_b^c \skewmat{ \hat{R}_I^b \left( \left( \hat{R}_I^g \right)^\transpose
  % \hat{\vect{r}}_{i/g}^g + \hat{\vect{p}}_{g/b}^v \right) } \\
  % \frac{\partial}{\partial \tilde{\vect{p}}_{g/b}^v} \vect{p}_{i/c}^c
  % &=
  % R_b^c \hat{R}_I^b \\
  % \frac{\partial}{\partial \tilde{\theta}_I^g} \vect{p}_{i/c}^c
  % &=
  % R_b^c \hat{R}_I^b \left( \hat{R}_I^g \right)^\transpose
  % \skewmat{\begin{matrix} 0 \\ 0 \\ 1 \end{matrix}} \hat{\vect{r}}_{i/g}^g \\
  % \frac{\partial}{\partial \tilde{\vect{r}}_{i/c}^c} \vect{p}_{i/c}^c
  % &=
  % R_b^c \hat{R}_I^b \left( \hat{R}_I^g \right)^\transpose .
% \end{align*}

